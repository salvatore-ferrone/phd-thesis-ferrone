\usepackage[nomaketitle]{config/psl-cover/psl-cover}

\sujet{Le titre du manuscrit de thèse pour obtenir le titre de docteur}
\author{Prénom Nom}
\encadrant{Directeur \textsc{De Recherche}}

\institute{Observatoire de Paris}
\doctoralschool{Titre de l'école doctorale}{xxx}
\specialty{Acoustique}
\laboratory{Laboratoire de Recherche}
\date{xx mois 202x}

\pslassetspath{config/psl-cover}

\jurymember{1}{Prénom \textsc{Nom 1}}{Affiliation & \emph{Examinateur}}{} % le président doit être en premier
\jurymember{2}{Prenom \textsc{Nom 2}}{Affiliation}{Rapporteur}
\jurymember{3}{Prenom \textsc{Nom 3}}{Affiliation}{Rapporteur}
\jurymember{4}{Prenom \textsc{Nom 4}}{Affiliation}{Rapporteur}
\jurymember{5}{Prenom \textsc{Nom 5}}{Affiliation}{Rapporteur}
\jurymember{6}{Prenom \textsc{Nom 6}}{DR Affiliation}{Directeur de thèse}
% \jurymember{7}{Prénom NOM}{Titre, établissement}{Directeur de thèse}

% entre 1700 et 4000 signes ? 
\frabstract{
\lipsum[10-11]
}

\enabstract{
\lipsum[13-14]
}

\frkeywords{mot clé 1, mot clé 2, mot clé 3, mot clé 4}
\enkeywords{keyword 1, keyword 2, keyword 3, keyword 4}


%%%%%%% Paramètres PDF
\title{\sujet}
\hypersetup{
pdftitle={Thesis Year},
pdfsubject={\sujet},
pdfauthor={\theauthor},
pdfkeywords={\thefrkeywords},
}
