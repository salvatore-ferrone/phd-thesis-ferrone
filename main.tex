\documentclass[a4paper,12pt,twoside]{book}
% Encodage des caractères et langue du document
\usepackage{setspace}
\usepackage[OT1]{fontenc}
\usepackage[utf8]{inputenc}
\usepackage{lmodern}
\usepackage[english,french]{babel}
\usepackage{textgreek}
% \usepackage{config/psl-cover/aas_macros}
\setcounter{tocdepth}{3} % Pour que les subsubsections n'apparaissent pas dans la TOC
\setcounter{secnumdepth}{3} % Pour que les subsubsections ne soient pas numérotées
\usepackage{fixltx2e}

%%%%%%%%%% Gestions des marges %%%%%%%%%% 
\usepackage{geometry} % Si on a besoin d'une configuration plus précise des marges
\geometry{a4paper,                % format de papier
% Définition des marges :
  left= 3cm,right = 2cm,  % marge intérieure extérieure à la page
  top = 3cm,bottom = 3cm,
% En-tête et pied de page :
  headheight=6mm,         % espace réservé à l'en-tête dans la marge top
  %headsep=3mm,            % espace entre le corps et l'en-tête
  %footskip=9mm            % espace entre le corps et le pied de page
  marginparwidth = 16mm
}

\raggedbottom
\reversemarginpar
% \usepackage{showframe} % pour afficher les traits des marges

%%%%%%%%%%%%%%%%%%%%%%%%%%%%%%%%%%%%%%%%%%%%%%%%%%%%%%%%%%%%%%%%%%%%
%%%%%%%%%%% Gestion maths %%%%%%%%%% 
\usepackage{amsmath,amssymb,amsfonts,amsthm}
%\usepackage{mathtools} % version modifiée de amsmath, ajoute des symboles, etc.
\usepackage{mathrsfs}% pour rajouter un format de lettres façon calligraphie en math mode.
\DeclareMathOperator{\sinc}{sinc}
\DeclareMathOperator{\e}{e}

\usepackage[locale = FR]{siunitx} % Pour gérer les unités
\sisetup{inter-unit-product=\ensuremath{{}\cdot{}}} % pour mettre des points médians entre les unités quand il y en a plusieurs
\sisetup{separate-uncertainty=true,multi-part-units=single} % pour faire des incertitudes en écrivant \SI{valeur(incertitude)}{unité}
\DeclareSIUnit\vitesse{\meter\per\second}
\usepackage{eurosym}
\DeclareSIUnit{\octet}{o}

%%%%%%%%%%%%%%%%%%%%%%%%%%%%%%%%%%%%%%%%%%%%%%%%%%%%%%%%%%%%%%%%%%%%
%%%%%%%%%%%  Graphics / Table / List %%%%%%%%%%%
\usepackage{graphicx,array,tikz,multirow}
\usepackage{caption,subcaption} % permet de faire des subfigures (remplace le package subfig)
\usepackage{svg,float}
\usepackage{booktabs,paralist}
\newcolumntype{x}[1]{>{\centering\arraybackslash\hspace{0pt}}p{#1}}
\usepackage[section]{placeins}
% \usepackage{hanging}

%%%%%%%%%%%%%%%%%%%%%%%%%%%%%%%%%%%%%%%%%%%%%%%%%%%%%%%%%%%%%%%%%%%%
%%%%%%%%%%% Header / Foot %%%%%%%%%%% 
\usepackage{fancyhdr,emptypage} % garantit que les pages blanches avant les débuts de chapitres soient vraiment blanches (pas d'en-tête ni de pied de page)
\let\cleardoublepage\clearpage
 
\fancypagestyle{plain}{ %% Page chapitre, toc ...
    \fancyhead{}\fancyfoot[C]{\thepage}
    \renewcommand{\headrulewidth}{0pt}
    \renewcommand{\footrulewidth}{0pt}
}

%%%%%%%%%%% Page normale
\pagestyle{fancy}
    % \renewcommand{\chaptermark}[1]{\markboth{\chaptername \ \thechapter.\ #1}{}} % sert à personnaliser l'affichage de \leftmark (ici : le mot "Chapitre", le numéro, un point, et le titre du chapitre, sans écrire en majuscules)
    % \renewcommand{\chaptermark}[1]{\markleft{\chaptername \ \thechapter.\ #1}{}}
    % \renewcommand{\sectionmark}[1]{\markright{\thesection.\ #1}} % sert à personnaliser l'affichage de \rightmark (ici : le numéro et le titre de la section en cours, sans écrire en majuscules)
    \fancyhf{} % assure que les entête et pieds de page sont vides au départ
    \fancyhead[LE]{\selectfont\nouppercase{\leftmark}}
    \fancyhead[RO]{\selectfont\nouppercase{\rightmark}}
    \fancyfoot[C]{\thepage}
% Explications :
% L = left, R = right, C = center, E = even pages, O = odd pages
%\leftmark : adds name and number of the current top-level structure (for example, Chapter for reports and books classes; Section for articles ) in uppercase letters.
%\rightmark : adds name and number of the current next to top-level structure (Section for reports and books; Subsection for articles) in uppercase letters.

%%%%%%%%%%% Personnaliser les premières pages des chapitres
\usepackage[Lenny]{fncychap}
\ChNameVar{\fontsize{25}{25}\usefont{OT1}{phv}{m}{n}\selectfont}
\ChRuleWidth{0pt}
\ChNumVar{\fontsize{60}{62}\selectfont\textcolor{curcolor}}

\makeatletter
\ChTitleVar{\Huge\rm}
\renewcommand{\DOCH}{%
\setlength{\fboxrule}{\RW} % Let fbox lines be controlled by
\fbox{\CNV\FmN{\@chapapp}\space \CNoV\thechapter}\par\nobreak
\vskip 20\p@}
\renewcommand{\DOTIS}[1]{%
\CTV\bfseries\FmTi{#1}\par\nobreak
\vskip 20\p@}
\makeatother

\renewcommand{\thesection}{\arabic{section}}

% Pour la table des matières
\usepackage[francais,nohints,tight]{minitoc}		% Mini table des matières, en français
\setcounter{minitocdepth}{2} % Mini-toc détaillées (sections/sous-sections)
\setlength{\mtcindent}{-1em} % décalage des minitoc à gauche
\dominitoc

\usepackage[nottoc]{tocbibind} % pour que la bibliographie apparaisse dans la table des matières (avec l'option pour que la table des matières elle-même n'apparaisse pas dans la table des matières).
% \usepackage{tocloft}% pour pouvoir modifier les tailles d'espacement dans la table des matières
\usepackage[titles]{tocloft}

%%%%%%%%%%%%%%%%%%%%%%%%%%%%%%%%%%%%%%%%%%%%%%%%%%%%%%%%%%%%%%%%%%%%
%%%%%%%%%%% Divers %%%%%%%%%%% 
\usepackage{textcomp} % rajoute des symboles 
\usepackage{xcolor} % pour ajouter de la couleur (si besoin)
\usepackage{epigraph} % pour rajouter des citations en début de chapitre  \epigraph{Citation}}{Auteur}
\usepackage{titling}
\usepackage{lipsum} 
\usepackage{csquotes} % added 07/09/21

\usepackage{xspace}
\usepackage{afterpage}
\renewcommand{\baselinestretch}{1.2} % interligne

\usepackage[textsize=footnotesize]{todonotes}

%%%%%%%%%%%%%%%%%%%%%%%%%%%%%%%%%%%%%%%%%%%%%%%%%%%%%%%%%%%%%%%%%%%%
%%%%%%%%%%% Links ref  %%%%%%%%%%% 
\usepackage{bookmark}
\usepackage{acronym}
\usepackage[nameinlink,french]{cleveref} % noabbrev
\Crefname{figure}{Fig.}{Figs.} % traduction des références aux figures/tables/équations
\crefname{figure}{fig.}{figs.}
\Crefname{equation}{Eq.}{Eqs.}
\crefname{equation}{eq.}{eqs.}
\Crefname{table}{Table.}{Tables.}
\crefname{table}{table.}{tables.}

% Configuration de hyperref
\definecolor{color_ref}{rgb}{0.18, 0.31, 0.31} % couleur cite
\definecolor{color_link}{RGB}{36, 56, 141}
\definecolor{curcolor}{RGB}{113,127,184} % couleur des liens (bleu clair)

\hypersetup{
	colorlinks=true, % colore les liens au lieu de les encadrer
	pdfstartview=FitV, % ouvre le PDF de façon à ce qu'il prenne la taille verticale de l'écran
	urlcolor=color_link, % choix de la couleur des liens URL
	linkcolor= color_link, % choix de la couleur des liens internes (table des matières, etc.)
	citecolor=color_ref % choix de la couleur des liens de citations
}

%%%%%%%%%%%%%%%%%%%%%%%%%%%%%%%%%%%%%%%%%%%%%%%%%%%%%%%%%%%%%%%%%%%%
%%%%%%%%%%% Bibliography ref  %%%%%%%%%%%
\usepackage[
    backend=biber,
    style=authoryear,
    sorting=nyt,
    maxcitenames=2,
    mincitenames=1,
    maxbibnames=6,
    minbibnames=6,
    natbib=true,
    hyperref=true,
    backref=true,
    url=false,
    doi=false,
    isbn=false
]{biblatex}

\renewcommand*{\bibfont}{\footnotesize}
\setlength\bibitemsep{\itemsep}
\renewbibmacro{in:}{} % remove "In:"

\renewcommand*{\labelalphaothers}{}
\DeclareLabelalphaTemplate{
  \labelelement{
    \field[final]{shorthand}
    \field{labelname}
    \field{label}
  }
  \labelelement{\literal{,\addhighpenspace}}
  \labelelement{\field{year}}
}

\AtEveryBibitem{%
    \clearfield{note} % Remove note
    \clearlist{language} % Remove doi
}

\definecolor{sapred}{rgb}{0.5098039,0.1411765,0.2}
\newcommand{\sapred}[1]{\textcolor{sapred}{#1}}

\newcommand{\sujet}[1]{\renewcommand{\sujet}{#1}}
\newcommand{\auteur}[1]{\renewcommand{\auteur}{#1}}
\newcommand{\encadrant}[1]{\renewcommand{\encadrant}{#1}}

\newcommand{\reference}[1]{%
\vspace{.1cm}
\begin{singlespace}
\tikzstyle{titlebox}=[rectangle,inner sep=10pt,inner ysep=10pt,draw=curcolor,draw]%
\tikzstyle{title}=[fill=white]%
\bigskip\noindent\begin{tikzpicture}
\node[titlebox] (box){%
    \begin{minipage}{0.88\textwidth}
#1
    \end{minipage}
};
\node[title] at (box.north west) {\color{curcolor}  reference};
\end{tikzpicture}\bigskip%
\vspace{.1cm}
% \minitoc
\end{singlespace}
\newpage
}

\newcommand{\myparagraph}[1]{\paragraph{#1}\mbox{} \vskip .5\baselineskip \par}

\newcommand{\T}[1]{T\textsubscript{#1}}


%%% Macro pour la mise en forme de la liste des publiations
\newcommand{\Myprod}[3]{ % sans DOI
\begin{singlespace}{#1}. ``\textit{{#2}}''. {#3}.
\end{singlespace}}

\newcommand{\MyprodwithDOI}[4]{ % avec DOI
\begin{singlespace}{#1}. ``\textit{{#2}}''. {#3}. {DOI: \href{https://doi.org/#4}{#4}}
\end{singlespace}}

% Chapitre sans mise en forme particulière (remerciements, conclusion
\newcommand{\addchapnonumber}[1]{
\phantomsection
\addtocounter{chapter}{1}
\chapter*{#1}
\addcontentsline{toc}{chapter}{#1}
\markboth{#1}{#1}
\setcounter{section}{0}
}

%%% Macro pour mise en forme de l'abstract et résumé avec mots clés
\newcommand{\AddResumeAbstract}{
\chapter{Résumé}
% L'avènement de l'astrométrie \textit{Gaia} a permis des études à grande échelle des étoiles et des sous-structures galactiques. Les $\sim 160$ amas globulaires (GCs) de la Voie lactée, chacun contenant de centaines de milliers à plusieurs millions d'étoiles, peuvent désormais être caractérisés par des distances, des mouvements propres, des vitesses radiales, des masses et des tailles précises. Parallèlement, le nombre de courants stellaires connus est passé d'environ 60 au début de cette thèse à plus de 120 aujourd'hui. Comme les courants tracent les orbites de leurs progéniteurs, ils constituent d'excellents sondages du potentiel gravitationnel de la Voie lactée et des perturbations de matière noire. Grâce à ces données, nous avons étudié le système des courants stellaires et des amas globulaires galactiques. J'ai développé le code open-source \texttt{tstrippy}, qui modélise le détachement tidal des étoiles d'amas via le problème restreint des trois corps. Avec ce cadre, nous avons prédit la distribution des débris tidaux de tous les GCs galactiques et réalisé des simulations ciblées du courant de Palomar~5, introduisant des « gaps » dues aux survols d'amas globulaires. Nous présentons les premières prédictions globales des débris tidaux de tous les GCs de la Voie lactée \citep{2023A&A...673A..44F}. Ces simulations sont accessibles publiquement et utilisées par la communauté. Dans une étude de suivi \citep{2025A&A...699A.289F}, nous avons quantifié la fréquence et l'ampleur des rencontres d'GC perturbant Palomar~5, montrant que ces interactions doivent être prises en compte pour éviter les faux positifs dans les recherches de sous-halos de matière noire. Nous avons également montré qu'une dispersion interne de vitesse accrue et des surdensités épicycliques peuvent réduire la persistance des gaps, diminuant la sensibilité des courants aux perturbations externes. Ce travail établit la distribution attendue des courants tidaux des GCs galactiques et quantifie le taux auquel les GCs perturbent Palomar~5. Il fournit une base pour distinguer l'origine des perturbations des courants, étudier la dynamique interne et l'évolution des amas globulaires, et contraindre le potentiel gravitationnel de la Voie lactée en matière visible et noire.
\thefrabstract
\markboth{}{}
\vskip 2em \noindent\makebox[\linewidth]{\rule{.5\linewidth}{0.4pt}}
\vfill
% \noindent \textbf{Mots clés :} Voie Lactée ; Amas globulaires ; courants stellaires ; Matière noire ; Dynamique galactique ; Disruption gravitationnelle
\noindent \textbf{Mots clés :} \thefrkeywords

\chapter{Abstract}
% The advent of \textit{Gaia} astrometry has enabled large-scale studies of stars and Galactic substructures. The Milky Way's $\sim 160$ globular clusters (GCs), each with hundreds of thousands to millions of stars, can now be characterized via accurate distances, proper motions, radial velocities, masses, and sizes. Meanwhile, known stellar streams have grown from about 60 at the start of this thesis to over 120 today. Because streams trace progenitor orbits, they provide excellent probes of the Milky Way's gravitational potential and dark matter perturbations. Enabled by this data, we studied the Galactic stellar stream and globular cluster system. I developed the open-source code \texttt{tstrippy}, which models tidal stripping of cluster stars via the restricted three-body problem. Using this framework, we predicted the distribution of tidal debris from all Galactic GCs and ran targeted simulations of Palomar~5's stream, introducing ``gaps'' from GC flybys. We present the first global predictions of tidal debris from all Milky Way GCs \citep{2023A&A...673A..44F}. These simulations are publicly available and used by the community. In a follow-up study \citep{2025A&A...699A.289F}, we quantified the frequency and range of GC encounters perturbing Palomar~5, showing such interactions must be considered to avoid false positives in dark matter subhalo searches. We also showed that increased internal velocity dispersion and epicyclic overdensities can reduce gap persistence, lowering stream sensitivity to external perturbations. This work establishes the expected distribution of tidal streams from Galactic GCs and quantifies the rate at which GCs perturb Palomar~5. It provides a foundation for disentangling the origins of stream perturbations, studying the internal dynamics and evolution of globular clusters, and constraining the Milky Way's gravitational potential in both visible and dark matter.
\theenabstract
\markboth{}{}
\vskip 2em \noindent\makebox[\linewidth]{\rule{.5\linewidth}{0.4pt}}
\vfill
% \noindent\textbf{Keywords :} Milky Way; Globular Clusters; Stellar Streams; Dark Matter; Galactic Dynamics; Tidal Disruption
\noindent\textbf{Keywords:} \theenkeywords
}
\input{config/journals.tex}
\usepackage{config/psl-cover/psl-cover}
\pslassetspath{config/psl-cover/}
\newcommand{\paola}[1]{\textcolor{purple}{#1}}
\usepackage{listings}
\title{Stellar Streams for Galactic and Cosmic Archaeology: when gravity meets complexity}
\institute{Observatoire de Paris-PSL, LIRA, UMR 8254 CNRS}
\doctoralschool{Astronomie et Astrophysique d'Ile de France}{127}
\specialty{Astronomy and Astrophysics}
\author{Salvatore Ferrone}
\date{13 October 2025}
\laboratory{LIRA, Observatoire de Paris}
\jurymember{1}{Françoise Combes}{Paris Sciences et Lettres}{Presidente}
\jurymember{2}{Jorge Penarrubia}{University of Edinburgh}{Referee}
\jurymember{3}{Anna Lisa Varri}{University of Edinburgh}{Referee}
\jurymember{4}{Chrstian Boily}{Université de Strasbourg}{Examinateur}
\jurymember{5}{Eugene Vasiliev}{University of Surrey}{Examinateur}
\jurymember{6}{Raffaella Schneider}{University of Rome ``La Sapienza''}{Examinatrice}
\jurymember{7}{Paola Di Matteo}{Paris Sciences et Lettres}{Supervisor}
\jurymember{8}{Marco Montuori}{Centro Nazionale Delle Ricerche}{Supervisor}
\frkeywords{Voie Lactée ; Amas globulaires ; courants stellaires ; Matière noire ; Dynamique galactique ; Disruption gravitationnelle}
\enkeywords{Milky Way; Globular Clusters; Stellar Streams; Dark Matter; Galactic Dynamics; Tidal Disruption}
\frabstract{L'avènement de l'astrométrie \textit{Gaia} a permis des études à grande échelle des étoiles et des sous-structures galactiques. Les $\sim 160$ amas globulaires (GCs) de la Voie lactée, chacun contenant de centaines de milliers à plusieurs millions d'étoiles, peuvent désormais être caractérisés par des distances, des mouvements propres, des vitesses radiales, des masses et des tailles précises. Parallèlement, le nombre de courants stellaires connus est passé d'environ 60 au début de cette thèse à plus de 120 aujourd'hui. Comme les courants tracent les orbites de leurs progéniteurs, ils constituent d'excellents sondages du potentiel gravitationnel de la Voie lactée et des perturbations de matière noire. Grâce à ces données, nous avons étudié le système des courants stellaires et des amas globulaires galactiques. J'ai développé le code open-source \texttt{tstrippy}, qui modélise le détachement tidal des étoiles d'amas via le problème restreint des trois corps. Avec ce cadre, nous avons prédit la distribution des débris tidaux de tous les GCs galactiques et réalisé des simulations ciblées du courant de Palomar~5, introduisant des « gaps » dues aux survols d'amas globulaires. Nous présentons les premières prédictions globales des débris tidaux de tous les GCs de la Voie lactée \citep{2023A&A...673A..44F}. Ces simulations sont accessibles publiquement et utilisées par la communauté. Dans une étude de suivi \citep{2025A&A...699A.289F}, nous avons quantifié la fréquence et l'ampleur des rencontres d'GC perturbant Palomar~5, montrant que ces interactions doivent être prises en compte pour éviter les faux positifs dans les recherches de sous-halos de matière noire. Nous avons également montré qu'une dispersion interne de vitesse accrue et des surdensités épicycliques peuvent réduire la persistance des gaps, diminuant la sensibilité des courants aux perturbations externes. Ce travail établit la distribution attendue des courants tidaux des GCs galactiques et quantifie le taux auquel les GCs perturbent Palomar~5. Il fournit une base pour distinguer l'origine des perturbations des courants, étudier la dynamique interne et l'évolution des amas globulaires, et contraindre le potentiel gravitationnel de la Voie lactée en matière visible et noire.}
\enabstract{The advent of \textit{Gaia} astrometry has enabled large-scale studies of stars and Galactic substructures. The Milky Way's $\sim 160$ globular clusters (GCs), each with hundreds of thousands to millions of stars, can now be characterized via accurate distances, proper motions, radial velocities, masses, and sizes. Meanwhile, known stellar streams have grown from about 60 at the start of this thesis to over 120 today. Because streams trace progenitor orbits, they provide excellent probes of the Milky Way's gravitational potential and dark matter perturbations. Enabled by this data, we studied the Galactic stellar stream and globular cluster system. I developed the open-source code \texttt{tstrippy}, which models tidal stripping of cluster stars via the restricted three-body problem. Using this framework, we predicted the distribution of tidal debris from all Galactic GCs and ran targeted simulations of Palomar~5's stream, introducing ``gaps'' from GC flybys. We present the first global predictions of tidal debris from all Milky Way GCs \citep{2023A&A...673A..44F}. These simulations are publicly available and used by the community. In a follow-up study \citep{2025A&A...699A.289F}, we quantified the frequency and range of GC encounters perturbing Palomar~5, showing such interactions must be considered to avoid false positives in dark matter subhalo searches. We also showed that increased internal velocity dispersion and epicyclic overdensities can reduce gap persistence, lowering stream sensitivity to external perturbations. This work establishes the expected distribution of tidal streams from Galactic GCs and quantifies the rate at which GCs perturb Palomar~5. It provides a foundation for disentangling the origins of stream perturbations, studying the internal dynamics and evolution of globular clusters, and constraining the Milky Way's gravitational potential in both visible and dark matter.}
\addbibresource{mybib.bib}

\begin{document}
\selectlanguage{english}
\pslcover{}
% filepath: /Users/sferrone/repos/phd-thesis-ferrone/config/sapienza-cover.tex
\begin{titlepage}
\linespread{1}\selectfont
\definecolor{sapred}{rgb}{0.5098039,0.1411765,0.2}
\sffamily
\setlength{\parindent}{0pt}

\vspace*{-10mm}

\includegraphics[width=5cm]{sapienzalogo}

\vspace{1cm}

% Indent all content with respect to logo
\vfill

\vfill
\hspace{1.5cm}
\begin{minipage}{0.75\textwidth}
    {\LARGE\linespread{1.1}\selectfont\textcolor{sapred}{Les courants stellaires pour l'archéologie galactique et cosmique}\par}
\end{minipage}

\hspace{1.5cm}
\begin{minipage}{0.75\textwidth}
    {\large\linespread{1.1}\selectfont\textcolor{sapred}{quand la gravité rencontre la complexité}\par}
\end{minipage}

\vfill
\vfill
\vfill
\hspace{1.5cm}
\textcolor{sapred}{Sapienza Università di Roma}

\hspace{1.5cm}
\textcolor{sapred}{PhD program in Astronomy, Astrophysics and Space Science (XXXVIII cycle)}

\vfill
\hspace{1.5cm}
{\bfseries Salvatore Ferrone}

\hspace{1.5cm}
ID number: 1935299

\vfill
\hspace{1.5cm}
Advisors

\hspace{1.5cm}
Prof. Paola Di Matteo

\hspace{1.5cm}
Prof. Marco Montuori


\vfill
\hspace{1.5cm}
Coordinator

\hspace{1.5cm}
Prof. Francesco Piacentini

\vfill 

\hspace{1.5cm}
Academic Year: 2024/2025


\end{titlepage}
\frontmatter
\AddResumeAbstract % macro pour résumé et abstract

\begin{singlespace} % réduction de l'interligne pour condenser
% Utiliser \adjustmtc si un décalage apparaît dans les petites tables des matières
\phantomsection \addcontentsline{toc}{chapter}{Table of contents}\adjustmtc
\tableofcontents\newpage
\renewcommand{\listfigurename}{List of Figures}
\phantomsection\listoffigures\adjustmtc % Liste des figures
\phantomsection\listoftables\adjustmtc % Liste des tableaux
% {\printglossary[type=\acronymtype,title=Acronymes et anglicismes]}\adjustmtc
\end{singlespace}

\mainmatter
\setcounter{page}{1}
\chapter{Introduction}
This thesis investigates stellar streams originating from globular clusters in the Milky Way. Stellar streams are long, thin structures composed of stars that have escaped their host stellar system, forming coherent tails that can span large regions of the sky (Fig.~\ref{fig:S5MilkywayStreams}). Globular clusters were first systematically cataloged by Charles Messier in the late 1700s--not for their scientific interest at the time, but to help comet hunters avoid mistaking these nebula-like objects for new comets \citep{1781cote.rept..227M}. Many of the most prominent globular clusters, along with diffuse nebulae and external galaxies, are included in the Messier catalog (Fig.~\ref{fig:All_messier_objects}). Physically, globular clusters are dense, gravitationally bound systems containing hundreds of thousands to millions of stars, and are an important source of stellar streams.
\begin{figure}
    \centering
    \includegraphics[width=\linewidth]{images/S5MilkywayStreams.jpg}
    \caption[Artist Rendition of Stellar Streams]{An artist's rendition of a galaxy surrounded by stellar streams. Credit: James Josephides and S$^5$ Collaboration \citep{2019MNRAS.490.3508L}.}
    \label{fig:S5MilkywayStreams}
\end{figure}
In Chapter 4, I present our study in which we simulated the expected distribution of stellar streams originating from the entire Milky Way globular cluster system. This study was the first of its kind. Stellar streams have high hopes for being inferential tools for fine details of the gravitational field of the Milky Way, notably for constraining the presence of \textit{dark mater subhalos}. These halos may perturb the stellar streams leaving ``gaps'' in their wake. Since structure of stellar streams are sensitive to local and global gravitational field, all phenomena that can influence the streams must be well calibrated to ensure proper inference of the distribution of dark matter within the Milky Way. In Chapter~5, we explored how the collective gravitational influence of all globular clusters can perturb streams to produce such gaps.
\begin{figure}
    \centering
    \includegraphics[width=\linewidth]{images/All_messier_objects.jpg}
    \caption[Messer objects]{The Messier catalog containing ``nebulae-like'' objects: planetary nebulae, diffuse nebulae, supernovae remenants, open clusters, globular clusters, and foreign galaxies. By Michael A. Phillips, an amateur astronomer. - http://astromaphilli14.blogspot.com.br/p/m.html official blog, CC BY 4.0, https://commons.wikimedia.org/w/index.php?curid=38121043}
    \label{fig:All_messier_objects}
\end{figure}
The structure of the thesis is as follows. The remainder of the introduction provides background on stellar streams, globular clusters, and their astrophysical context, as well as an overview of the current state of the field. Chapter~2 outlines the physical framework used to model stream formation and interpret their morphology. Chapter~3 describes the numerical methods employed in the simulations, including convergence tests and estimates of computational cost. Chapters~4 and~5 present two published studies, with the final chapter discussing these results in the broader context of the literature and highlighting future directions.

\section{General context}
    Before explaining why Milky Way globular clusters and stellar streams are scientifically interesting, it is important to first set the scene. This thesis fits neatly within the field of galactic astronomy, which, at its core, studies the current state of our galaxy and the processes that shaped its formation within the broader context of the universe.

    If we start this narrative from the beginning and place stellar streams and globular clusters in the timeline of cosmic history, their importance becomes clear. The story begins best at the very start. The Lambda Cold Dark Matter ($\Lambda$CDM) cosmological model is currently the leading theory of the universe, successfully unifying a variety of observational evidence from the Cosmic Microwave Background Radiation and the large-scale distribution of galaxies to the accelerating expansion of the universe, etc. \citep{2001LRR.....4....1C,2022NewAR..9501659P}.

    Shortly after the Big Bang, conditions allowed protons, neutrons, and electrons to form and interact. For a few minutes, these particles collided and fused into heavier elements in a process known as Big Bang Nucleosynthesis \citep{2007ARNPS..57..463S}. When this phase ended, the universe's composition was mostly hydrogen, deuterium ($^2$H), helium-4 ($^4$He), and trace amounts of helium-3 ($^3$He) and lithium-7 ($^7$Li). By mass, hydrogen made up roughly 75\%, and helium about 25\% of the primordial universe \citep{1966ApJ...146..542P,2016RvMP...88a5004C}.

    $\Lambda$CDM positis the existence of Dark matter and that it has about five times more mass than ordinary matter \citep{2020A&A...641A...6P}. Dark plays a critical role in galaxy formation. In the early universe, dark matter was distributed nearly uniformly. Over dense regions caused it to collapse into a cosmic web of filaments and nodes. These massive nodes created deep gravitational wells that attracted ordinary matter \citep{1974ApJ...187..425P}. The infalling gas subsequently cooled and formed stars. The resulting complex of stars, gas, and dark matter constitutes a galaxy \citep{2008LNP...740.....P,2010gfe..book.....M}.

    Galaxies contain stars that are born, fuse hydrogen and helium into heavier elements, and eventually die, often as supernovae, enriching the interstellar medium with these heavier elements \citep{2019A&ARv..27....3M}. The stellar formation process is a strong function of cosmic time as the chemical evolution of the universe becomes more metal rich. For example, the first generation of stars, known as Population III (Pop III) stars\footnote{Stellar populations are named in reverse order of discovery. Population I stars are the youngest, metal-rich stars (including the Sun), Population II stars are older and metal-poor, and Population III stars are the very first, metal-free stars.}, formed in a very different environment, one devoid of metals \citep{2002Sci...295...93A,2005SSRv..117..445G,2013RPPh...76k2901B}. 

    The chemical composition of the gas is crucial since it influences the initial mass function (IMF) of stars. Stars formed from pristine, metal-free gas tend to have a top-heavy IMF, favoring the formation of massive stars \citep{2002ApJ...571...30S,2006MNRAS.369..825S}. In contrast, even small amounts of metals introduced into the gas can dramatically shift the IMF toward the favoring the formation of lighter stars \citep{2021MNRAS.508.4175C}.

    Globular clusters are, to first approximation, single-stellar populations: their constituent stars formed from the same molecular cloud over a timescale shorter than the internal dynamical time \citep{1988ApJ...324..288A,2009MNRAS.397..954F,2014PhR...539...49K}. Owing to their uniform chemical composition, stellar evolution within a cluster proceeds along the color-magnitude diagram as a function solely of initial mass \citep{2013sse..book.....K}. This property enables precise age determinations from photometric observations, a method that historically provided the first robust lower limits on the age of the Universe \citep{1959MNRAS.119..124H,1970ApJ...162..841S,1985A&A...147..169G,1992ApJ...400..265M}. With their low metallicities and advanced ages, globular clusters are dominated by Population~II stars, indicating that they formed in environments enriched exclusively by the earliest generations of stars \citep{2022A&A...668A.191C}. They thus constitute valuable fossil records of the initial phases of star formation and chemical evolution in the Universe.
    
    Globular clusters are among the most ubiquitous stellar systems in the Universe \citep{2006ARA&A..44..193B,2019ARA&A..57..227K}. They are found in virtually all galaxy types, from dwarfs to giant ellipticals, and their total number correlates with global properties of their host galaxies such as mass and luminosity \citep[e.g.,][]{2013ApJ...772...82H,2018MNRAS.481.5592F}. In the local Universe, we can directly observe the formation of massive bound stellar systems—so-called young massive star clusters—which may represent present-day analogs of the GC formation process \citep[e.g.,][]{2010ARA&A..48..431P,2020SSRv..216...69A}, though whether all such systems will evolve into globular clusters remains an open question.

    Globular clusters serve as unique astrophysical laboratories, offering insights into stellar dynamics, stellar evolution, and galaxy assembly. Their evolution involves stellar structure and evolution, gravitational dynamics, and relativistic effects in dense environments. For instance, close stellar encounters and binary interactions—such as mass transfer or mergers—are common in such environments and significantly shape cluster evolution \citep{2004MNRAS.349..129D,2016MNRAS.458.1450W,2024MNRAS.528.5119A}. Additionally, globular clusters are suspected to be a pathway for forming intermediate-mass black holes (IMBHs) \citep{2013MNRAS.432.2779B,2015MNRAS.454.3150G}. The origin of IMBHs remains uncertain, as their masses are too large to be explained by isolated stellar evolution, yet too small to fall into the supermassive category \citep{2020ARA&A..58..257G}. 
    
    Moreover, while traditionally considered as single stellar population that are homogeneous in age and chemical composition, decades of spectroscopic and photometric evidence have revealed the presence of multiple stellar populations in most GCs \citep{2008MNRAS.391..825D,2012A&ARv..20...50G,2018ARA&A..56...83B}. The origin of these multiple populations remains debated, with proposed explanations ranging from self-enrichment to accretion of external material, but no consensus has yet emerged.

    \citet{2025arXiv250116438K} proposes that the main formation of globular cluster formation is at high redshift galaxies in high pressure environments. However, other mechanisms can exist as well and contribute to this.
    

    \citet{2018RSPSA.47470616F} also discusses two different formation scenarios. That GCs form naturally at high redshift. Or that they form within dark matter sub halos. 
    
    \citep{2016ApJ...823...52K} talks about  the formation of globular clusters within dark matter sub-halos. it's awesome! they discuss prestine gas and also evoke enrichment from pop III and pop II stars. They ran some hydro dynamical simultions. 

    A striking property of many GC systems is the bimodality in their metallicity distribution, with one population of metal-poor GCs and another of metal-rich GCs \citep[e.g.,][]{2006ARA&A..44..193B, 2015ApJ...806...36H}. This bimodality is now widely interpreted as evidence for multiple formation channels: metal-rich GCs likely formed in situ within the main progenitor galaxy during intense star-formation episodes, whereas metal-poor GCs were predominantly accreted from lower-mass satellites. The persistence of this bimodality across a wide range of galaxies suggests that hierarchical accretion has been a fundamental process shaping GC systems.

    From the perspective of galaxy formation, the hierarchical model of structure growth posits that galaxies assemble through repeated mergers and accretion of smaller systems \citep{2015ARA&A..53...51S}. In this framework, GCs can form in situ within the main progenitor or be accreted from satellite galaxies \citep[e.g.,][]{2018MNRAS.479.4760F,2020MNRAS.498.2472K,2023A&A...673A..86P,2024MNRAS.528.3198B,2025A&A...693A.155P}. Consequently, the spatial distribution, kinematics, chemical abundances, and metallicity substructure of a galaxy's GC system retain valuable information about its merger history and assembly pathways. This makes GCs not only tracers of early star formation, but also fossil records of the build-up of their host galaxies.

    In essence, globular clusters are involved in many astrophysical processes such as the hierarchical formation of galaxies, formation of intermediate black holes, and being products of barely enriched gas in the early states of the universe \citep{2016ApJ...823...52K,2025arXiv250116438K}. Fully understanding their formation and evolution ties neatly into a variety of astrophysical problems.

    Not only are globular clusters fascinating for stellar formation, evolution, rich stellar dynamics, black hole formation, and galactic build up, but also for how they form \textit{stellar streams}. Globular clusters can evaporated and eject low mass stars due to their own internal dynamics \citep{2003gmbp.book.....H}. Additionally, fact the galactic tidal froces strip stars from the globular cluster as well. \citep{2007ApJ...659.1212M}
    
    \textit{the galactic gravitaitonal field is stronger on the close side cluster than on the far. This differential force is known as tidal froces and rips the body apart. the stars that get stripped from a the cluster trace roughly trace out the orbit of the cluster (cite all the papers that support this and say tricky tricky). To give an idea on why this is incredibly useful, consider the long time scales involved with astronomy. For instance, the Sun takes roughly 220 million years to complete a single orbit around the Galaxy. Stellar streams, by encoding orbital trajectories of their progenitor clusters, provide a unique window into millions of years of dynamical history. By characterizing their shapes and extents, we gain precise constraints on the Galactic gravitational potential.}



    \citep{1992ApJ...386..519O} did the Fokker-planck equations \citet{1999A&A...352..149C} also used N body simulations to predict tidal tails.  

    Stellar streams, formed by stars gradually escaping from globular clusters due to tidal forces, are especially valuable. These streams preserve dynamical information about both their progenitor clusters and the gravitational potential of the host galaxy. The timing and properties of escaping stars can reveal the recent internal evolution of the cluster itself \citep{1972ApJ...178..623T}.  \citet{1995AJ....109.2553G} did the first discovery.

    Moreover, as stellar streams trace arcs across the sky, they serve as natural probes of the Galactic gravitational field. Their coherent kinematics allow constraints on the mass distribution and substructure of the Milky Way \citep{2011MNRAS.417..198V}. This dual insight into cluster dynamics and galactic environment makes streams a uniquely powerful tool for modern galactic archaeology and dynamics.

    \textit{now transition into gaia}

\section{The state of the art}
    The current era of Galactic astronomy is undoubtedly defined by the \emph{Gaia} space mission \citep{2016A&A...595A...1G,2016A&A...595A...2G,2018A&A...616A...1G,2021A&A...650C...3G,2023A&A...674A...1G}. The European Space Agency's space-based observatory has been conducting continuous observations of the sky to perform precise astrometric measurements. Operating since 2014, Gaia improves its data volume and quality each year through repeated parallax measurements.  As of Data Release~3, the mission has provided distances and proper motions for nearly two billion stars in the Galaxy, along with millions of radial velocity measurements \citep{2023A&A...674A...1G}. This yields five-dimensional phase-space information (positions and on-sky velocities) for about 1-2\% of the Galactic stellar population. Although the radial velocities are limited to relatively bright stars, and therefore a smaller spatial coverage, the resulting dataset is still unprecedented in scope.

    When additional measurements are required, \emph{Gaia} data are often complemented with observations from other facilities.  For example, \emph{Gaia} struggles to determine parallaxes in the crowded interiors of globular clusters, where high stellar densities pose challenges for its instruments \citep{2017MNRAS.467..412P}. To address this, \citet{2021MNRAS.505.5957B} combined \emph{Gaia}~EDR3 astrometry with \emph{Hubble Space Telescope} observations, enabling precise distance measurements for 162 globular clusters.

    Several ground-based spectroscopic surveys are specifically designed to extend \emph{Gaia}'s radial velocity measurements to fainter magnitudes or more crowded regions.  Examples include the Apache Point Observatory Galactic Evolution Experiment (APOGEE; \citealt{Majewski2017}), the Gaia-ESO Survey \citep{2023A&A...676A.129H}, and upcoming wide-field facilities such as 4MOST \citep{2019Msngr.175....3D} and WEAVE \citep{2014SPIE.9147E..0LD}.  Other large surveys primarily focus on providing detailed chemical abundances for a wide range of stellar populations, including GALAH \citep{2012ASPC..458..421Z}, the Gaia-ESO Survey \citep{2023A&A...676A.129H}, and APOGEE \citep{2017AJ....154...94M}, often delivering both radial velocities and multi-element abundance measurements in the same observations.



\citet{2021ApJ...909L..26B} showed that streams can also be used to get at the accretion history of the Milky Way. 

\textbf{cite Malhan papers} particularly the one showing how two streams were the same but one got twisted. 

\textbf{cite the bonaca papers}

\textbf{cite the gaia papers}

\textbf{cite ibata papers}

\textbf{other stellar stream papers}


\chapter{Theory}
In this chapter, I present the theoretical background necessary to understand the modeling of globular clusters and stellar streams performed in this thesis. Much of the content draws from two comprehensive introductions to galactic dynamics: Galactic Dynamics by Binney and Tremaine, and Galaxiesbook.org by Bovy. These references provide a solid foundation for the physics and mathematical tools used throughout this work.

A common assumption in galactic dynamics is the so-called fluid limit, in which the orbit of a star is determined by the smooth gravitational potential generated by the galaxy as a whole. In this approximation, interactions between individual stars are negligible. This assumption holds well in many contexts—but not in all.

Globular clusters are a notable exception. Their relatively small number of stars makes them too “grainy” for the fluid approximation to hold, yet they contain far too many stars to be treated as simple few-body systems. This intermediate regime is the subject of the aptly named Million Body Problem, explored in detail by Heggie and Hut. Their textbook provides a thorough survey of methods to address this challenge, and their preface offers an insightful summary of the central difficulty: globular clusters inhabit an awkward middle ground where neither the fluid limit nor simplified few-body interactions apply cleanly. As a result, no analytical theory fully captures their dynamics.

While this thesis focuses primarily on stellar streams—specifically, how stars escape from globular clusters and evolve under the influence of the galactic potential—it is important to acknowledge that the internal evolution of the progenitor clusters still affects the properties of the streams. Although the internal cluster dynamics lie outside the scope of this work, they place important constraints on the interpretation of our results.

The remainder of this chapter is structured to clarify the theoretical framework supporting this thesis. I divide the discussion into three main parts:

\begin{itemize}

    \item \textbf{Explicit physics} - the physical laws and initial conditions implemented in the simulations;

    \item \textbf{Implicit physics} - the emergent behavior of these systems, the assumptions involved, and the mathematical tools used to interpret the results;

    \item \textbf{Ignored physics} - relevant aspects of the problem that are beyond the scope of this thesis, but which impact the interpretation of our results. Where appropriate, I cite works that pursue these directions and discuss how future work could incorporate them to improve upon the current modeling.

\end{itemize}




\section{The Explicit Physics}
    My simulations solve the \textit{restricted three body problem}. In essence
    \begin{figure}
        \centering
        \includegraphics[width=\linewidth]{images/restricted_three_body_set_up.png}
        \caption{Little sketch of my equations of motion. }
    \end{figure}
    
    \subsection{Equations of Motion} \label{subsec:myEquationsOfMotion}
        I like to start with the \textit{Lagrangian}, which comes from the variational principle which states that particles with move along trajectories that minimize the difference, $ L = T-U $, which $L$ is the Lagrangian, $T$ is the kinetic energy and $U$ is the potential energy. Also, as is almost always the case in gravitational dynamics, we normalize by the mass and use \textit{specific} energy: 
        \begin{equation}
            \mathcal{L} = \frac{L}{m} = \frac{1}{2}\left(\dot{x}^2+\dot{y}^2+\dot{z}^2\right) - \Phi(x,y,z).
        \end{equation}
        However, Lagrange's equations give a system of three second order coupled ordinary differential equations. If we switch to Hamiltonain dynamics, we can object a set of six \textit{first} order ordinary differential equations, which is easier to implement computationally. Also, since we are using the specific energy, the momentum coordinates for Hamilton's equations are the same as the velocities from the Lagrangian: $ p_i = \frac{\partial \mathcal{L}}{\partial \dot{q}_i}$. Therefore, $p_i = \dot{q}_i$, where $i \in \left(x,y,z\right)$. The Hamilton is derived through the Legendre transform: $ \mathcal{H}=\sum_i p_i\dot{q}_i - \mathcal{L}$. Then, we can apply Hamilton's equations to obtain the set of equations: 
        \begin{align}
            \dot{p}_i &= -\frac{\partial \mathcal{H}}{\partial q_i} \\
            \dot{q}_i &= \frac{\partial \mathcal{H}}{\partial p_i}
        \end{align}

        And when written explicity become: 
        \begin{align}
            \dot{p}_x &= -\frac{\partial \Phi}{\partial x} \\
            \dot{p}_y &= -\frac{\partial \Phi}{\partial y} \\
            \dot{p}_z &= -\frac{\partial \Phi}{\partial z} \\
            \dot{x} &= p_x \\ 
            \dot{y} &= p_y \\ 
            \dot{z} &= p_z \\ 
        \end{align}        


        \subsubsection*{The Globular Cluster}
            In the case of the Globular Clusters, they only feel the Galaxy. So their Hamiltonian becomes: 
            % \begin{equation}

            % \end{equation}

        \subsubsection*{The Star Particles}

    \subsection{Potential density pairs}

        \begin{figure}
            \centering
            \includegraphics[width=\linewidth]{images/figure_pouliasis2017pii_potential_-8_8.png}
            \caption{The main potential used throughought the thesis}
        \end{figure}        
    

\section{The Implicit Physics}

    \subsection{The circular restricted three body problem}
        \begin{itemize}
            \item The Lagrange points
            \item Allowed regions 
        \end{itemize}

    \subsection{The tidal tensor}
        Tidal forces arise due to spatial variations in the gravitational field and are especially apparent when comparing the accelerations experienced by nearby particles. To explore this, consider a Taylor expansion of the gravitational potential of the primary, \(\Phi_g\), evaluated at the star's position \(\vec{x}_s\), relative to the secondary's position \(\vec{x}_c\):
        \begin{equation}
            \Phi_g\left(\vec{x}_s\right) \approx \Phi_g\left(\vec{x}_c\right) + \left[\nabla \Phi_g (\vec{x}_c)\cdot \Delta \vec{x}\right] + \left[\Delta \vec{x} \cdot \mathcal{D}^2\left(\Phi_g\right) \cdot \Delta\vec{x}\right],
        \end{equation}
        where \(\Delta \vec{x} = \vec{x}_s - \vec{x}_c\), and \(\mathcal{D}^2 \Phi_g\) is the Hessian matrix of second derivatives of the potential: \(\partial^2 \Phi/\partial x_i \partial x_j\).

        An equivalent expression can be derived by linearizing the gravitational force in a non-inertial frame co-moving with the secondary. Let us write Newton's second law for the star-particle and the secondary in an inertial frame:
        \begin{eqnarray}
            \vec{F}_s &= \nabla \Phi_c\left(\Delta \vec{x}\right) + \nabla \Phi_g\left(\vec{x}_s\right),\\
            \vec{F}_c &= \nabla \Phi_g\left(\vec{x}_c\right).
        \end{eqnarray}
        Then the relative acceleration of the star in the non-inertial frame is:
        \begin{eqnarray}
            \vec{f}_s &= \vec{F}_s - \vec{F}_c + \vec{F}_\mathrm{fictitious} \\
                    &= \nabla \Phi_c\left(\Delta \vec{x}\right) + \nabla \Phi_g\left(\vec{x}_s\right) - \nabla \Phi_g\left(\vec{x}_c\right) + \vec{F}_\mathrm{fictitious} \\
                    &\approx \nabla \Phi_c\left(\Delta \vec{x}\right) + \mathrm{Jac}\left(\nabla \Phi_g(\vec{x}_c)\right) \cdot \Delta \vec{x} + \vec{F}_\mathrm{fictitious},
        \end{eqnarray}
        where the last line uses a first-order Taylor expansion of the gravitational force field, valid under the assumption that \(|\Delta \vec{x}| \ll |\vec{x}_c|\). 

        The Jacobian of the gravitational field is equal to the Hessian of the potential, owing to the symmetry of second derivatives and the fact that \(\vec{g} = -\nabla \Phi_g\). This matrix, known as the \textit{tidal tensor} \(\mathcal{T}\), describes the linearized spatial variation of the gravitational field:
        \begin{equation}
            \mathcal{T} = -\mathcal{D}^2\Phi_g = \mathrm{Jac}(\nabla \Phi_g) = \left(\begin{matrix}
                \partial_x g_x & \partial_y g_x & \partial_z g_x \\
                \partial_x g_y & \partial_y g_y & \partial_z g_y \\
                \partial_x g_z & \partial_y g_z & \partial_z g_z 
            \end{matrix}\right).
        \end{equation}

        While the Hessian and Jacobian are formally equivalent, the Jacobian viewpoint offers a more geometric interpretation: it acts as a linear transformation on nearby displacements, mapping them to differences in acceleration. Diagonalizing the tidal tensor reveals the principal axes of tidal deformation. A positive eigenvalue corresponds to stretching along the associated eigenvector; a negative eigenvalue indicates compression. The magnitude gives the rate of stretching or compression.

        Finally, we note that although many relevant potentials exhibit spherical or cylindrical symmetry, Cartesian coordinates are preferred here. In curvilinear systems, computing the Jacobian or Hessian requires accounting for Christoffel symbols, which complicates the interpretation and computation.


        
        \subsubsection*{The Moon}
            Nothing clarifies the concept of tides like the most familiar example: the Moon. Tidal forces are invoked to explain a wide range of phenomena in the Earth-Moon system. The most relatable effect is, of course, the periodic variation in sea level on Earth. While accurately modeling these changes requires fluid dynamics—beyond the scope of this thesis—NASA provides several accessible explanations and visualizations at \href{https://science.nasa.gov/moon/tides/}{https://science.nasa.gov/moon/tides/}, including daily high and low tides, as well as spring and neap tides.

            Another key example is the tidal deformation of the Moon, which ultimately led to its tidal locking—explaining why we always see the same side of the Moon from Earth.

            A particularly insightful illustration is the angular offset between the Earth's tidal bulge and the Moon's position, caused by the Earth's rotation. This offset results in a torque that transfers angular momentum from the Earth's rotation to the Moon's orbit. As a consequence, Earth's rotation gradually slows while the Moon slowly recedes from Earth. Much of this behavior can be understood qualitatively using the tidal tensor for a Keplerian potential:
            \begin{equation}
                \mathcal{T}= -\frac{GM}{r^3}\left(\begin{matrix}
                    1-\frac{3x^2}{r^2} & -\frac{3xy}{r^2} & -\frac{3xz}{r^2} \\
                    -\frac{3yx}{r^2} & 1-\frac{3y^2}{r^2} & -\frac{3yz}{r^2} \\
                    -\frac{3zx}{r^2} & -\frac{3zy}{r^2} & 1-\frac{3z^2}{r^2}
                \end{matrix}\right),
            \end{equation}
            which has eigenvalues $2\frac{GM}{r^3}$, $-\frac{GM}{r^3}$, and $-\frac{GM}{r^3}$, with corresponding eigenvectors:
            \begin{equation}
                \vec{v}_1,\vec{v}_2,\vec{v}_3 = \dfrac{1}{r}\begin{bmatrix} x \\ y \\ z \end{bmatrix}, \dfrac{1}{r}\begin{bmatrix} -x \\ y \\ 0 \end{bmatrix}, \dfrac{1}{r}\begin{bmatrix} -x \\ 0 \\ z \end{bmatrix}.
            \end{equation}

            Notably, the first eigenvalue is positive and corresponds to a stretching deformation along the position vector. The other two are negative, representing compression in directions perpendicular to the stretching axis. These directions define a plane orthogonal to the Earth-Moon line. From this, several tidal effects become evident. For instance, the Earth's oceans stretch along the Earth-Moon axis due to the Moon's tidal forces. While the Sun also exerts tidal forces on Earth, their magnitude is weaker due to the $r^{-3}$ scaling with distance.

            When the Moon is either full or new, the Sun and Moon's tidal forces act constructively, leading to spring tides. At first and third quarters, they interfere destructively, causing neap tides. Additionally, Earth's tidal influence distorts the Moon from spherical symmetry into an ellipsoid. The Moon's most stable orientation is one where its longest axis aligns with the Earth-Moon line—resulting in tidal locking.

            A more quantitative treatment of these phenomena would require modeling the Moon's internal structure and Earth's ocean dynamics—well beyond the gravity-only scope of this thesis. However, we can still explore one instructive effect: how solar tidal forces \textit{perturb} the Moon's orbit away from the idealized two-body Earth-Moon configuration. Figure~\ref{fig:moon_tidal_simulation} shows a toy model comparing two scenarios. In both, I used initial conditions based on JPL NASA ephemerides (citation needed) and integrated two sets of equations of motion.

            In the first scenario, the Moon's motion is governed by the two-body Earth-Moon problem with a rotating reference frame correction:
            \begin{equation}
                \ddot{\vec{r}} = -\frac{GM_\oplus}{r^3}\vec{r} - \omega_\oplus \times \left(\omega_\oplus \times \vec{r}_\oplus\right),
            \end{equation}
            while in the second, we include the effect of solar tidal forces:
            \begin{equation}
                \ddot{\vec{r}} = -\frac{GM_\oplus}{r^3}\vec{r} - \omega_\oplus \times \left(\omega_\oplus \times \vec{r}_\oplus\right) -\frac{GM_\odot}{r_\oplus^3}
                \left(\begin{matrix}
                    1-\frac{3x^2}{r_\oplus^2} & -\frac{3xy}{r_\oplus^2} & -\frac{3xz}{r_\oplus^2} \\
                    -\frac{3yx}{r_\oplus^2} & 1-\frac{3y^2}{r_\oplus^2} & -\frac{3yz}{r_\oplus^2} \\
                    -\frac{3zx}{r_\oplus^2} & -\frac{3zy}{r_\oplus^2} & 1-\frac{3z^2}{r_\oplus^2}
                \end{matrix}  \right) \cdot \vec{r},
            \end{equation}
            where $r_\oplus$ is the Earth's position relative to the Sun, $\vec{r}$ is the Moon's position relative to Earth, $M_\odot$ is the mass of the Sun, and $M_\oplus$ is the mass of the Earth. The coordinates $x, y, z$ refer to the components of Earth's heliocentric position.

            
            
\begin{verbatim}
VIDEO: moon_tidal_simulation.mp4
\end{verbatim}

            \begin{figure}
                \centering
                \includegraphics[width=\linewidth]{images/moon_tidal_simulation.png}
                \caption{An illustrative experiment demonstrating the effect of the Sun's tidal field. The left panels show the tidal field, and the right panels show two snapshots of the Moon's orbital trajectory. The green curve corresponds to the solution of Eq.~\textbf{XXX}, which includes the Sun's tidal effects, while the orange curve corresponds to the simpler two-body problem that neglects them. The black ellipse represents the tidal ellipsoid, whose major axis remains aligned with the Sun's position vector relative to the Earth. The bottom panel shows the accumulated phase difference between the two solutions. Neglecting solar tides causes the predicted Moon orbit to drift ahead of the more accurate trajectory. With about three to four years, the two body predicted solution would off by half a moon phase.
 }\label{fig:moon_tidal_simulation}
            \end{figure}

        \subsubsection*{Tides in the Galaxy}
            \begin{itemize}
                \item Show some positions of the tidal tensor and how it can change orientation based on it's altitude 
                \item The tidal forces add together linearly, so show the Halo and the Disc example together. 
            \end{itemize}

            The Miyamoto Nagai potential is: 
            \begin{eqnarray}
                \Phi'   &= \frac{1}{D}\\
                D       &= \sqrt{x^2 + y^2 + \beta^2(z)}\\
                \beta(z)   &= 1 + \sqrt{z^2 + b^2}\\
                \beta'(z) &= \frac{z}{\sqrt{z^2 + b^2}}\\
                \beta''(z)  &= \frac{b^2}{\left(z^2 + b^2\right)^{3/2}}
            \end{eqnarray}

            The dimensionless tidal tensor then becomes: 


            \begin{figure}
                \includegraphics[width=\linewidth]{images/miyamoto_disc_shocks_ab_rp_e_i_0.20_4.0_0.01_50.0.png}
                \caption{No disk shocks on a planar orbit.}
                \label{fig:miyamoto_disc_shocks_circular_inclined_orbit}
            \end{figure}



            \begin{figure}
                \includegraphics[width=\linewidth]{images/miyamoto_disc_shocks_ab_rp_e_i_0.20_4.0_0.50_0.0.png}
                \caption{Disk shocks on a circular inclined orbit.}
                \label{fig:miyamoto_disc_shocks_circular_inclined_orbit}
            \end{figure}

            \begin{figure}
                \includegraphics[width=\linewidth]{images/miyamoto_disc_shocks_ab_rp_e_i_0.20_4.0_0.50_30.0.png}
                \caption{Disk shocks on an orbit with resonant R and z.}
                \label{fig:miyamoto_disc_shocks_responant_R_z}
            \end{figure}
            
            \begin{figure}
                \includegraphics[width=\linewidth]{images/miyamoto_disc_shocks_ab_rp_e_i_0.20_31.6_0.60_50.0.png}
                \caption{Big apocenter}
                \label{fig:miyamoto_disc_shocks_big_apocenter}
            \end{figure}
            
            \begin{figure}
                \includegraphics[width=\linewidth]{images/miyamoto_disc_shocks_ab_rp_e_i_0.88_4.0_0.50_45.0.png}
                \caption{Not very disky with this axis ratio}
                \label{fig:miyamoto_disc_shocks_weak_shocks}
            \end{figure}


            \begin{equation}
                \mathcal{T}_{i,j}=-\frac{1}{D^3}\left(\begin{matrix}
                    1-\frac{3x^2}{D^2} & -\frac{3xy}{D^2} & -\frac{3x\beta \beta'}{D^2} \\
                    \dots & 1-\frac{3y^2}{D^2} & -\frac{3y\beta \beta'}{D^2} \\
                    \dots & \dots & \beta'^2 + \beta \beta'' -\frac{3\left(\beta\beta'\right)^2}{D^2}
                \end{matrix}\right)
            \end{equation}    
            You can see immediately that multiplying the matrix by $\vec{v}=\lambda\left[x,y,z\right]$ does not return a vector that is parallel to the position vector, as it does for the spherical mass distribution. 

            The Marto's halo has this mass distribution:
            
            \begin{equation}
                M'_\text{enc}(s) = \frac{s^\gamma}{1+s^{\gamma-1}}
            \end{equation}
            The dimensionless tidal tensor is thus: 
            \begin{equation}
                \mathcal{T'}_{i,j}= -\frac{M'_\text{enc}(s)}{s^3}\left(\begin{matrix}
                    1-\frac{x^2}{s^2}f(s) & -\frac{xy}{s^2}f(s) & -\frac{xz}{s^2}f(s) \\
                    -\frac{yx}{s^2}f(s) & 1-\frac{y^2}{s^2}f(s) & -\frac{yz}{s^2}f(s) \\
                    -\frac{zx}{s^2}f(s) & -\frac{zy}{s^2}f(s) & 1-\frac{z^2}{s^2}f(s)
                \end{matrix}\right)
            \end{equation}  
            where 
            \begin{equation}\label{eq:martos_f_s}
                f(s) = 2-\frac{\gamma-1}{1+s^{\gamma-1}}
            \end{equation}

        \subsubsection*{Interesting case}
            There is an interesting area in the parameter space where the tidal forces would impede creating stellar streams instead of making them, as shown in Fig.~\ref{fig:martos_tidal_field_small_r}.

            Taking the Martos tidal tensor in Eq.~\ref{eq:martos_f_s}, we can see that for $\gamma > 3$ and $s \ll  1$, then $f(s)< 0$. Physically, this would be a sphere whose density increases with distance. This is not natural, as, in general, gravity sends the more massive objects towards the center. However, it's fun to indulge in this situation to learn some insight about the flexibility of tidal fields. The consequence of $f(s)< 0$ is that all terms in the tidal tensor are negative, which means that the force is compressive everywhere. Consequently, no stars escape from the cluster. 

            In Fig.~\ref{fig:martos_tidal_field_small_r}, I present a small experiment demonstrating the consequence of such a tidal force on a globular cluster, which is that no tidal stream forms. Briefly, I created a plummer sphere of $10^6$~M$_\odot$ and half mass radius 20~pc and evolved it in a Martos halo potential of mass parameter $10^{12}$~M$_\odot$ a characteristic radius of $30$~kpc. Each cluster was placed at the same initial conditions, a distance of 1/4 the scale radius from the center of the potential. The initial velocity was made perpendicular to the position vector with a speed of $(1-e)v_\textrm{c}$. This is a pseudo-eccentricity, which was added to have a non-circular orbit to demonstrate how the trajectories change in the two cases. The top panel uses a $\gamma$ of 2.02, which is the same value in the model where the halo was originally presented, and the value I employ in this thesis. Next, the bottom panel uses $\gamma$ of 4.5, which corresponds to a density profile where $\rho(r) \propto r^{1.5}$. 

            To get a feel for the strength of the tidal stretching and compression, I show a circle and the resulting ellipse after applying the tidal deformation. I computed the coordinates of the ellipse by adding $\vec{Ell} = \vec{C} + \frac{1}{2} t_\textrm{char}^2 \mathcal{T}\cdot \left(\vec{C} - \vec{r_o}\right)$. This way, force can be mapped to position space, and the strengths of the tidal forces can be seen visually. The characteristic time, $t_\textrm{char}$, was set to $\frac{1}{10} 2\pi r_\textrm{halo} / v_\textrm{c}$. 
            
            \begin{figure}
                \includegraphics[width=\linewidth]{images/martos_tidal_field_202_10_25.png}
                \includegraphics[width=\linewidth]{images/martos_tidal_field_450_10_25.png}
                \caption{The plots show two low-resolution streams (N = 1000) created by dissolving a Plummer sphere in the Martos halo potential. The units are scaled to the halo's characteristic radius. Gamma is the mass exponent and is the sole variable between the two simulations. The panels on the left show the orbit in gray and the stars in black. The black arrow points towards the center of the potential. The panels on the right show the tidal field, which is the tidal tensor evaluated at each position in space. The gray dotted circle is plotted with an arbitrary radius and is deformed by the tidal field into a black ellipse.}
                \label{fig:martos_tidal_field_small_r}
            \end{figure}

            \begin{figure}
                \includegraphics[width=\linewidth]{images/martos_tidal_field_202_10_400.png}
                \includegraphics[width=\linewidth]{images/martos_tidal_field_450_10_400.png}
                \caption{The same experiment as Fig.~\ref{fig:martos_tidal_field_small_r}, but the cluster was placed at a larger distance of 4 $r_\textrm{halo}$, since we are beyond the characteristic radius, the tidal fields are the same, despite the different exponents $\gamma$.}
                \label{fig:martos_tidal_field_big_r}
            \end{figure}

            In the case of Fig.~\ref{fig:martos_tidal_field_big_r}, the tidal field returns to the typical situation where one axis is compressive and the other stretches. Notice how the deformations are similar in magnitude, while in the case of $\gamma=2.02$ for the top panel of Fig.~\ref{fig:martos_tidal_field_small_r}, the compression is stronger than the expansion. Both of these are different than the keplerian tidal deformation where the stronger deformation is stretching and whose axis is parallel to the position vector. 






    \subsection{Phase mixing}
        \begin{itemize}
            \item The Luiville theorem
            \item How it is slower with tidal tails 
            \item also the monte-carlo approach with phase mixing is what causes vast differences in orbital solutions after a certain period of time 
        \end{itemize}
    
    \subsection{Shocks}
        I started this section with the circular and planar restricted three body problem. It really simplifies the problem instead of looking for a general solution. All three bodies are point masses, in fact the tertiary body has no mass. The secondary is on a circular orbit about the primary, and the tertiary is in the same orbital plane as the secondary. This simplified problem is already quite complex but solvable and rich with physics. However, by restricting the orbits of the tertiary and secondary, we lose a lot of physics that affects our system. Additionally, for the globular clustres, most of them are not on circular orbits. Additionally, the galaxy is not a point mass, but rather a mass distribution with cylindrical symmetry.        



\section{The Ignored Physics}
    \subsection{Collisional dynamics}
        \begin{itemize}
            \item not nbody
            \item no mass segregation
            \item no three body encounters 
            \item no soft or hard binaries 
            \item show some results from Corespray 
        \end{itemize}
    
    \subsection{Stellar evolution}
        \begin{itemize}
            \item They're all point masses 
            \item No salpeter's 
            \item No strong initial mass loss 
            \item No accurate model for the colors 
            \item No multiple stellar populations 
        \end{itemize}
    
    \subsection{Time evolution}
        In someways, we take time evolution into account, and in someways, we ignore and this has already been covered in the previous sections. i.e., the orbit of the star-particles depend on the position of the host globular cluster, which I do not solve for simoltaneously but instead opt to load it into the computation, as shown in Section~\ref{subsec:myEquationsOfMotion}. Also, things like mass segregation and stellar evolution are time-dependent which is completely ignored in my simulations. 



\chapter{Numerical methods}
In line with a thesis in computational astrophysics, the equations of motion presented here do not admit closed-form analytical solutions. As such, we must rely on numerical methods and computer simulations to solve them. Specifically, I solve $N$ independent sets of Hamilton's equations, each consisting of six first-order, coupled differential equations.

Numerical integration presents several challenges. First, there is the practical question: how do we actually solve these equations? Most students first encounter a simple update scheme, such as Euler's method: $y_{i+1} = y_i + \frac{dy}{dt}\Delta t$. However, this quickly becomes insufficient for coupled systems or for problems where precision and conservation laws are essential. How can we be confident that our numerical solutions faithfully approximate the true dynamics, especially when the true solution is unknown? How do we handle performance, data volume, or the trade-offs between speed and accuracy?

Although many software packages already exist to solve such problems, I chose to write my own code. Some senior researchers warn, ``Don't reinvent the wheel.'' Others lamet that ``the problem with the youth today is that no one knows how the packages they use work.'' Caught between damned if I do and damned if I don't, I decided to write my own solver anyway. This led to the development of \texttt{tstrippy}, a code designed to solve the restricted three-body problem. My motivation was part practical: I wanted to avoid installation headaches, steep learning curves, and uncertainty over whether existing tools could handle my specific setup. But above all else, I wanted to create a reliable product that works, allows me to reproduce my results, and run my simulations at scale. 

Developing my own package gave me a deeper understanding of code structure, numerical algorithms, and the subtleties of scientific programming. It also gave me the confidence to later use other packages more effectively. The \texttt{tstrippy} code is available on GitHub and runs on macOS and Linux.

This chapter documents how I numerically solve the equations of motion, how I validate the accuracy of the solutions, and how the code is organized under the hood.



\section{Astronomical units and scaling}

    When writing any code, the choice of units is important. Astronomical units are rarely the same as SI units. In general, the choice of units is observationally and historically motivated, resulting in a system that uses multiple units for the same physical quantity, which can be confusing at first.

    For instance, sky positions are typically reported in spherical coordinates. Right ascension (analogous to longitude) is often expressed in either degrees or hours, while declination (similar to latitude) is given in degrees. Distances are reported in parsecs when derived from parallax measurements. Line-of-sight velocities, obtained via spectroscopic Doppler shifts, are reported in kilometers per second. Proper motions describe angular displacements over time and are usually reported in milliarcseconds per year. Already, we encounter several different units for angles (degrees, hours, arcseconds), time (years, seconds), and distance (km, kpc), none of which align with SI’s standard units of radians, seconds, or meters, as summarized in Table~\ref{tab:units}.

    \begin{table}[]
        \caption{Units for various astronomical quantities in Galactic and SI systems.}
        \label{tab:units}
        \begin{tabular}{l|l|l|l|l|l|l|}
                            & Distance  & RA                     & DEC                    & \textbf{$\mathrm{v}_\mathrm{LOS}$} & $\mu_\alpha$ & $\mu_\delta$ \\ \hline
            Galactic: & {[}kpc{]} & {[}deg{]} {[}HHMMSS{]} & {[}deg{]}              & km/s                      & {[}mas/yr{]} & {[}mas/yr{]} \\ \hline
            S.I.       & {[}m{]}   & {[}rad{]}              & {[}rad{]}              & m/s                       & {[}rad/s{]}  & {[}rad/s{]}  \\ 
        \end{tabular}
    \end{table}

    This raises practical concerns—for example, what would be the unit of acceleration? km/s$^2$? parsec/year/second? To systematically manage units, we turn to dimensional analysis, notably the Buckingham Pi theorem. In classical mechanics, physical quantities are typically expressed in terms of three fundamental dimensions: length, time, and mass. Any quantity can then be represented as a product of powers of these base units:

    \begin{equation}
        \left[\mathrm{Quantity}\right] = \mathcal{l}^a t^b m^c =
            \begin{bmatrix}
                a\\
                b\\
                c 
            \end{bmatrix}
    \end{equation}

    For example, velocity has dimensions $[1, -1, 0]$, momentum is $[1, -1, 1]$, and acceleration is $[1, -2, 0]$.

    It is not strictly necessary to adopt length-time-mass as the fundamental basis, as long as the three chosen base units are linearly independent. In stellar dynamics, it is often more natural to use distance, velocity, and mass as the base units. In this thesis, I adopt:
    \begin{itemize}
        \item Distance: 1~kpc
        \item Velocity: 1~km/s 
        \item Mass: 1~solar mass $\mathrm{M}_\odot$
    \end{itemize}

    In this system, time has derived units of:
    \begin{equation}
        \left[t\right] = \frac{\mathrm{distance}}{\mathrm{velocity}} = \frac{\mathrm{kpc}}{\mathrm{km/s}}.
    \end{equation}
    While not immediately intuitive, this unit of time is convenient because:
    \begin{equation}
        1\frac{\mathrm{kpc}}{\mathrm{Gyr}} \approx 1\frac{\mathrm{km}}{\mathrm{s}}.
    \end{equation}

    The gravitational constant has dimensions:
    \begin{equation}    
        \left[G\right]=\frac{v^2 \cdot l}{m},
    \end{equation} 
    which evaluates numerically in these units as: 
    \begin{equation}
        G = 4.301 \times 10^{-6} \left(\mathrm{km}/\mathrm{s}\right)^2 \cdot \mathrm{kpc} \cdot \mathrm{M}_\odot^{-1}.
    \end{equation}

    Once the base units are defined, derived quantities such as acceleration follow directly. Whether considering acceleration as $v^2~l^{-1}$ or $l \cdot t^{-2}$, they are equivalent and yield: $\left(\mathrm{kpc}/\mathrm{s}\right)^2 \cdot \mathrm{kpc}^{-1}$.

    It is worth mentioning that $N$-body codes often simplify further by selecting distance, velocity, and the gravitational constant as the base units, setting $G = 1$. While this simplifies force computations, it introduces less intuitive units for mass. For instance, by choosing 1~kpc for distance and 1~km/s for velocity, and setting $G = 1$, the derived mass unit becomes:
    \begin{equation}
        \left[\mathrm{mass}\right] = \frac{l \cdot v^2}{G} = 232509~\mathrm{M}_\odot.
    \end{equation}

    This approach was used in our first paper (see Chapter~4). Another example is the case of \texttt{Galpy}. \citet{2015ApJS..216...29B} introduced \textit{natural units} motivated by the galaxy's rotation curve, embodied in:
    \begin{equation}
        \frac{v_\mathrm{circ}^2}{R_0} = \nabla  \Phi \left(R_0, z=0\right),
        \label{eq:vcirc}
    \end{equation}
    which is the circular velocity in a cylindrically symmetric potential evaluated in the plane. In this normalization, $R$ is the cylindrical scale length of the galaxy, and $v_\mathrm{circ}$ is the circular velocity at this radius, both of which are set to 1. The gravitational constant is also set to 1. Whatever the form of the potential, the scale lengths must be normalized to $R$, and the mass parameter is subsequently determined through Eq.~\ref{eq:vcirc}. The total potential is a linear combination of individual components, with the user selecting the contribution of each component to the force at the characteristic radius. For example, $\Phi = \sum_i a_i\Phi_i$, where $a_i$ are weights such that $\nabla \Phi_i(R_0, z=0) = a_i$ in normalized units.

    In this system of units, emphasis is placed on the rotation curve and how much each component contributes to it at the reference radius of the galaxy. Note that $v_\mathrm{circ}(R_0)$ is not necessarily the maximum value of the rotation curve.

    In short, each code presents its own preferred units and normalization. It is the job of a computational astrophysicist to understand these conventions and be able to convert between them and observable quantities. $N$-body codes work best when setting $G = 1$, while \texttt{Galpy} uses natural units that emphasize the rotational properties of a galaxy. \texttt{Tstrippy}, by contrast, expects the user to pass masses in solar masses, velocities in kilometers per second, and distances in kiloparsecs.  However, physical constants are not hard-coded, so the user may pass any numerical values to the code as long as they are based on a self-consistent unit system. The code includes the \texttt{Pouliasis2017pii} potential and a catalog of galactic globular cluster properties, both reported in the aforementioned units.

    A valid general strategy when developing numerical codes is to implement a module that converts user-defined units to the internal units of the code. This functionality also exists in \texttt{Galpy} and a similar system is implemented in \texttt{Agama} \citep{2018arXiv180208255V}. I chose not to add such a layer to \texttt{Tstrippy}, since my goal was not to develop the most robust tool in the field, but rather to answer specific scientific questions. That said, \texttt{Astropy} provides an excellent unit-handling module that allows users to convert between units easily \citep{2013A&A...558A..33A}, and I recommend its use in the documentation. 

    In the end, I chose this unit convention because I believe it is the most intuitive for users and aligns with how quantities are typically reported in galactic astronomy.


\section{Solving the equations of motion}

    Long before the advent of computers, Euler proposed a simple method for numerically solving differential equations. However, it is inefficient and inaccurate for coupled systems, especially when long-term behavior matters. In the case of Hamiltonian systems, we can exploit specific geometric properties to design numerical schemes that better preserve the qualitative features of the true solution.

    In this thesis, I implemented the Leapfrog integrator and the Forest-Ruth scheme. These schemes are derived from the structure of Hamiltonian mechanics and are known as \textit{symplectic integrators}. Before continuing, I would like to quote \citep{bovy_inprep}:

    \begin{quote}
        Hamiltonian integrators are often called symplectic. This name comes from the fact that these integrators are Hamiltonian maps, whose mathematical structure is that of a vector flow on a symplectic manifold. Many fine dynamicists have made great contributions to the field without delving deeply into the meaning of the previous sentence and we do not discuss this further.
    \end{quote}

    However, my curiosity about linguistics pushed me to delve further: What does \textit{symplectic} mean? \citet{weyl1946classical} coined the term because \textit{complex} was already taken. The prefix \textit{com} refers to together, and ``plexus'' comes from greek meaning ``woven'' or ``braided''. So from its roots, “complex” means something composed of interwoven parts. The image of tangled braids captures the idea well. Similarly, ``sym-'' is a Greek prefix meaning “together.” The idea remains the same: in Hamiltonian dynamics, the evolution of position and momentum are interdependent. This becomes clearer in matrix form:
    \begin{equation}
        \begin{bmatrix}
            \dot{\bf{q}}\\
            \dot{\bf{p}}
        \end{bmatrix}
         = 
        \begin{bmatrix}
            0 & I_n \\
            -I_n & 0 
        \end{bmatrix}
                \begin{bmatrix}
            \frac{\partial \mathcal{H}}{\partial \bf{q}} \\
            \frac{\partial \mathcal{H}}{\partial \bf{p}}
        \end{bmatrix}
        \label{eq:symplectic}
    \end{equation}
    Here, the skew-symmetric matrix “weaves” the positions and momenta together.

    Although the equations of motion do not admit analytical solutions, they possess several known properties. First, trajectories governed solely by gravity are time-reversible. This property is important for my methodology, where I integrate the equations of motion backward in time and then forward again to the present-day position. Secondly, the total orbital energy is conserved. Moreover, according to Liouville's theorem, Hamiltonian flows preserve the local phase space volume. A corollary of this is that the determinant of the Jacobian matrix of the transformation from $\left(q,p\right)\rightarrow \left(q',p'\right)$ must be one, which means that the transformation only rotates or translates an infinitesimal volume but does not shrink or expand the volume. We can view the transform as: 
    \begin{eqnarray}
        q' &= q + \frac{\partial \mathcal{H}}{\partial p}\Delta t, \\
        p' &= -\frac{\partial \mathcal{H}}{\partial q}\Delta t + p,
    \end{eqnarray}
    The Jacobian matrix is given by $\left(\frac{\partial x_i'}{\partial x_j }\right)$: 
    \begin{equation}
        \begin{bmatrix}
            1 & \Delta t \frac{\partial^2 \mathcal{H}}{\partial p^2} \\  
            -\Delta t \frac{\partial^2 \mathcal{H}}{\partial q^2} & 1 \\  
        \end{bmatrix}
    \end{equation}
    and the subsequent determinant is: 
    \begin{equation}
        \mathrm{det}\left(J\right) = 1 - \Delta t^2 \frac{\partial^2 \mathcal{H}}{\partial q^2} \frac{\partial^2 \mathcal{H}}{\partial p^2}.
    \end{equation}
    In general, neither $\frac{\partial^2 \mathcal{H}}{\partial q^2}$ or $\frac{\partial^2 \mathcal{H}}{\partial p^2}$ are zero. Therefore, to preserve phase-space volume, we update positions and momenta in alternating steps. The transformation order becomes: $(q,p) \rightarrow (q',p) \rightarrow (q',p')$. This is commonly referred to as a sequence of \textit{drifts} and \textit{kicks}. A \textit{drift} updates the position while holding the momentum fixed, and a \textit{kick} updates the momentum while holding the position fixed. Symplectic integrators alternate these operations in a specific sequence to preserve the structure of Hamiltonian flow.

    The scheme outlined above is essentially a first-order method and is closely related to Euler's method. More sophisticated integrators use values from multiple time steps to construct higher-order estimates of the system's evolution. For example, some schemes temporarily evolve the position to an intermediate value $q_\mathrm{temp}$, use this to compute a momentum $p_\mathrm{temp}$, and then adjust both using weighted averages or predictor-corrector steps to reach the final state. These methods carefully balance forward and backward steps to optimize accuracy while preserving the symplectic structure.

    One of the most commonly used integrators in galactic dynamics is the leapfrog method. It works by interleaving updates of positions and momenta using time-centered averages. Specifically, the average momentum between $q_i$ and $q_{i+1}$ (denoted $p_{i+1/2}$) is used to advance the position, and then the average force (derived from the potential) is used to update the momentum. In Cartesian coordinates—used throughout this thesis—the leapfrog algorithm can be rewritten in terms of grid-aligned quantities:


    \begin{eqnarray}
        x_{i+1} &= x_i + \dot{x}_i \Delta t + \frac{1}{2}\ddot{x}_i \Delta t^2, \\
        \ddot{x}_{i+1} &= -\nabla \Phi(x_{i+1}), \\
        \dot{x}_{i+1} &= \dot{x}_i + \frac{1}{2}\left(\ddot{x}_i + \ddot{x}_{i+1}\right)\Delta t.
    \end{eqnarray}
    This formulation highlights how the leapfrog method naturally incorporates the second-order accuracy and time-reversibility essential for modeling gravitational systems over long timescales. As I will show in the next section, the leapfrog algorithm is sufficient for the problems that I solve. However, the question of computational efficiency and numerical accuracy is ever present. Leapfrog uses uses the two local points about the position and momenta to evolve them. Other schema can use more points to have more accurate estimations for the local derivatives. 
    
    \citet{1990PhyD...43..105F} proposed one such method for symplectic integrations, which is our situation here. The method is complicated and involves finding roots of high order polynomials, and the roots of which determine the weights and distances about the local point for finding the best estimate of the derivative for evoling the system. The method involves solving a cubic polynomial to determine the optimal coefficients. While the derivation is mathematically involved, the final scheme is straightforward to implement. Nonetheless, I implemented this method and tested its efficiency against the leapfrog. There are eight coefficients in this method, which are presented in table~\ref{tab:forest_ruth_coeffs}.
    \begin{table}[h]
        \centering
        \caption{Velocity (\(c_n\)) and acceleration (\(d_n\)) coefficients for the Forest-Ruth symplectic integrator.}
        \label{tab:forest_ruth_coeffs}
        \begin{tabular}{ccccc|cccc}
            \multicolumn{4}{c|}{Velocity coefficients (\(c_n\))} & \multicolumn{4}{c}{Acceleration coefficients (\(d_n\))} \\
            $c_1$ & $c_2$ & $c_3$ & $c_4$ & $d_1$ & $d_2$ & $d_3$ & $d_4$ \\
            \hline
            $w + \frac{1}{2}$ & $-w$ & $-w$ & $w + \frac{1}{2}$ & $2w + 1$ & $-4w - 1$ & $2w + 1$ & $0$ \\
        \end{tabular}
    \end{table} 
    The coefficients are all based on the solution to the cubic polynomial: $48 w^3 + 24 w^2 - 1 = 0 $. For a single step, the positions and velocities are updated as follows:
    \begin{align} 
        x' &= x + c_n v \Delta t \\ 
        t' &= t + c_n \Delta t \\ 
        \ddot{x} &= \nabla \Phi (x') \\ 
        \dot{x}' &= \dot{x} + d_n \ddot{x} \Delta t,
    \end{align}
    where $n$ is the \textit{mini-step}. Notice that the sum of $\sum_n^4 c_n$ and $\sum_n^4 d_n$ both equal 1, which is a full time step $\Delta t$. 

    Lastly, it is important to note that the leapfrog algorithm is symplectic and time-reversible only for Hamiltonians that are both time-independent and separable—that is, where the Hamiltonian can be written as a sum of a kinetic term depending only on momenta, $T(p)$ and whose potential depends only on position $\Phi(q)$. These conditions are satisfied for systems in an inertial frame with conservative forces. This is true when I integrate the motion for the center of mass of the globular clusters. However, the Hamiltonian for the integration of the particles does depend on time. So the leapfrog algorithm may introduce systematic integration errors due to the violation of its underlying assumptions, beyond ordinary rounding errors.

    Similarly, when we integrate the orbits of either the particles or the globular clusters in the galaxy containing a galactic bar, we are faced with a choice: we can either work in a time-dependent inertial frame, where the potential rotates and the Hamiltonian explicitly depends on time, or we can transform to a rotating frame, in which case the kinetic energy becomes position-dependent due to Coriolis and centrifugal forces, which breaks the necessary criteron of separability: $\mathcal{H}(q,p) = T(p)+\Phi(q)$. In both cases, the standard assumptions of the leapfrog algorithm are violated.

    Nonetheless, we will continue to use leapfrog as it remains a robust and efficient integrator for a wide range of astrophysical systems. Its good long-term energy behavior makes it a reasonable approximation even when the ideal assumptions are not strictly met. However, this highlights the need for careful validation: we must verify that the integration errors remain within acceptable bounds, especially in systems with non-separable or time-dependent dynamics. This validation is the subject of the next section.







\section{Tstrippy}

    \citet{2018ComAC...5....2V} talks about a series of papers between a larger collaboration of people who specialize in collisional dynamics and who have performed a series of workshops together. The introduction stated that the collaboration wants to tackle many open questions regarding stellar clusters and build the necessary codes to interprete the future large quantity of data that was destined to come. It has now come since the review was 2018. An interesting point was that in general globular clusters are approximated as being orderless, i.e. isotropic but order does present itself within these stelalr systems. Another large problem is no one knows what a good set of initial conditiosn is. Unresolved binaries pose a problem because you can overestiamte the total mass of the system. If I talk about this review, I should probably discuss some of the results from the papers that is builds on or at least their techinques.

    The MODEST review led me to discover AMUSE, which is an framework for integrating various astrophysical codes for solving 4 types of problems: gravitational dynamics, radiative transfer, hydrodynamics, and stellar evolution. The codes are written by the community and are interfaced together with Amuse. The user end is python. I have spent some time reading the book, which is instructive and well written. Steve McMillian is one of the authors. The code has a large support on GitHub and is still being developped. I have had trouble trying to install the code. It seems as though their documentation is incoherrent. At one place, it said `pip' is the easiest way to install. It didn't work. In another place, I was instructed to install a zipped up tarball. The setup failed becuase it expected there to be a .git file in the directory. I successfully downloaded the code by cloneing the repository, despite the fact that this was not recommended. I can use some aspects of the code but not all of them. For instance, my memory tells me that about 80\% of the test suite passed, thus many scripts failed. This was when I only installed the frame-work, which was advised since installing the whole package is huge and unnecessary since I am not solving all astrophysical problems. However, I wasn't able to use one of the gravity solvers that was presented in the textbook `AstrophysicalRecipes The art of AMUSE'. The install still has some codes that failed for instance: amuse-adaptb, amuse-hermite-grx, amuse-mi6. However, I'm hoping that this isn't necessary. I want to educate myself and make some examples. 

    Installing other codes and figuring out their functionalities to me has never been trivial. This is similar to galpy when I tried to figure out particle spray method and got less than good results. Agama also confused me a bit. The main point is that for each package, at the end of the day I decided that it was easier and better if I solved the problem myself with my own code. Because, even with the other packages, I know that they can be used to solve other astrophysical problems and it wasn't clear to me how to make the codes solve my specific set of of the restricted three body problems in a potential with other perturbers flying around. 

    In this search, I also discovered another review called \textit{Computational methods for collisional stellar systems} by Spurzem and Kamlah 2023. It is also interesting and instructive. I found it insightful when they called NBody an industry. I think the story of GRAPE and Makino is really interesting, how he build dedicated hardware for the nbody problem which were great for 10 years but were quickly replaced by GPU technology. 
    \begin{itemize}
        \item f2py, and why did we choose to use Fortran? 
        \item Bovy's guide for making a public python package
        \item migrating going from setuptools to meson
        \item a brief overview of how it works. 
        \item how I can either save orbits or snapshots
    \end{itemize}

    \section{Computation time and Data Volume}

    \begin{verbatim}
    VIDEO: cluster_showing_scale_and_dynamical_time.mp4
    \end{verbatim}


\chapter{Extra tidal debris of Milky Way Globular Clusters}
\section{General intro}

How should I introduce this paper??

\section{\bf{Numerical method}}\label{methods}

    To model the formation and evolution of extra-tidal features around Galactic globular clusters, we use a set of codes, called Globular Clusters' Tidal Tails (GCsTT)  developed by our group. It comprises two python codes, for the backward and forward integration of a stellar system, made of N test-particles (see Sect.~\ref{numerical}). These codes are separated for data organization and management, while the (computationally) most expensive part, namely, the calculation of the accelerations acting on the N particles and the orbits integration, is realized by means of a Fortran module written by our group. This module is interfaced to python by means of f2py directives from NumPy. The use of test-particle methods for modeling the tidal stripping process is widespread in the literature, where these methods are  usually applied to one or few clusters at a time \citep[see, e.g., ][]{lane12, mastrobuono12, palau19, piatti21a, grillmair22}. In this work, we apply a test-particle methodology to the whole set (159) of Galactic globular clusters for which this is currently possible, also taking into account, for each cluster, the errors on astrometry, line-of-sight velocities\footnote{Note: the term ``line-of-sight velocities''  adopted in this paper corresponds to the term ``radial velocities'' often used in the literature, as well as in the Gaia catalogues. We prefer the use of the first term, since the second is usually used also to indicate the (Galactocentric) radial velocities and can introduce some ambiguity, especially when different coordinate systems are used. We emphasize that the choice to use the term ``line-of-sight velocity'' is not new \citep[see, e.g., ][]{vasiliev21}.} and distances. In the following, we describe the two main steps of the procedure used by GCsTT  to simulate the tidal stripping process (Sect.~\ref{numerical}), the initial conditions adopted for the clusters' parameters and their mass distribution (Sect.~\ref{initialconds}), as well as the Galactic potentials (Sect.~\ref{galmod}).
    \subsection{Simulations of the tidal stripping process:  Two-step procedure}\label{numerical}
        To model the formation and evolution of extra-tidal features around Galactic globular clusters, and predict their current properties, we proceed as follows: 

        \textit{Step i: Backward integration. Reconstructing the globular cluster orbit over the last 5~Gyr}: First, for each Galactic globular cluster for which the distances from the Sun, proper motions, line-of-sight velocities, and structural parameters are  available (see Sect.~\ref{initialconds}), we determine their current positions and velocities in a Galactocentric reference frame, in which the Sun is at $(x_\odot, y_\odot, z_\odot) = (-8.34, 0., 0.027)$~kpc \citep{chen01, reid14} and at a given velocity for the local standard of rest, $v_{LSR}= 240$~km/s \citep{reid14}, and a peculiar velocity of the Sun with respect to the LSR, $(U_{\odot}, V_{\odot}, W_{\odot})  = (11.1, 12.24, 7.25)$~km/s \citep{schonrich10}. We then integrate the orbit of a single point mass, representing the cluster barycenter, backwards in time for 5~Gyr, and in this way, we retrieve its position and velocity at that time in the chosen Galactic potential (see Sect.~\ref{galmod}). We notice that other choices for the Sun's position or velocity with respect to the Galactocentric frame would have been possible. For example, \citet{piatti21a} adopted the same values as ours for the $v_{LSR}$ and for the peculiar velocity of the Sun but with a different distance to the Galactic center \citep[8.1~kpc in their work, see][]{gravity18}. The difference in the adopted position of the Sun is, however, generally smaller than the uncertainties affecting our knowledge of the distance of Galactic globular clusters to the Sun. For this reason, we do not to explore the dependency of the results presented in this paper with regard to these choices. 

        \textit{Step ii: Forward integration. Test-particle streams from the past to the present day}: Once the positions and velocities of the barycenter of each cluster, 5~Gyr ago, have been determined, we build  the corresponding $N$-body system, with N = 100 000 particles.  The phase-space coordinates of these particles are generated following a Plummer distribution, with the total mass and half-mass radius as described in Sect.~\ref{initialconds}. The barycenter of this $N$-body cluster is then assigned initial positions and velocities in the Galactic model, as those retrieved at step $(i),$ and the cluster is then integrated forward in time until the present day. Particles in this $N$-body system are modeled as test-particles, that is, they experience the gravitational field exerted by the globular cluster itself (see Sect.~\ref{initialconds}) and by the Galaxy (see Sect.~\ref{galmod}), but do not generate any gravitational field themselves. This allows us to maintain a computational time which scales as $O(N)$ and not as $O(N^2)$, as would be the case for a direct $N$-body self-consistent computation.

        In the following, we refer to these simulations, made by using the most probable values on distances, proper motions, and line-of-sight velocities, as the ``reference simulations.'' In addition, for each globular cluster, we also take into account the errors on its distance, proper motions, and line-of-sight velocity,  assuming Gaussian distributions of the errors, treated independently, and by generating 50 random realizations of these parameters.  For each of these realizations, we repeat the steps  described above, that is: (\textit{step i}) we determine the associated current positions and velocities in the chosen Galactocentric reference frame, we integrate the orbit of the single-point mass (representing the cluster barycenter) backwards in time, retrieving the corresponding values 5~Gyr ago, (\textit{step ii}) we build an $N$-body cluster containing $N$= 100~000 particles, with total mass and half-mass radius as those used for the reference simulation, and then we integrate the $N$-body cluster forwards in time until the present-day position. 

        To summarize, for a given Galactic potential, we run $159\times (50+1)=8109$ simulations, where 159 is the total number of clusters for which we currently have both 6D phase-space information and structural parameters. As we discuss in the following section, the whole set of globular clusters has been evolved in three different Galactic potentials, which implies that a total of 24 327 simulations have been run.

        For the orbit integration, a leap-frog algorithm is used, with a fixed time-step, $\Delta t$, and a total number of steps, $N_{steps}$, such that the total simulated time is  $\Delta t \times N_{steps}=5$~Gyr. The choice of the value of $\Delta t$ adopted to simulate each cluster in the Galactic potential has been based on the energy conservation of the corresponding cluster evolved in isolation (i.e., without the effect of the Galactic gravitational field for 5~Gyr). For the majority of the clusters (109/159), this value was set to $\Delta t = 10^5$~yr (for a corresponding value of $N_{steps}=50\,000$), while for the remaining clusters (50/159) a  $\Delta t = 10^4$~yr (for a corresponding value of $N_{steps}=500\,000$) was used. We refer to Appendix~\ref{deltat} (and in particular to Table~\ref{tcross-energy}) for additional details on the choice of $\Delta t$ for the whole set of clusters. As for the total simulated time, while globular clusters are much older than 5~Gyr, we chose this time limit because the longer back in time we could go, the less certain we would be of the Galactic environment. In addition, the last significant mergers in the Galaxy happened between 9 and 11~Gyr ago  \citep[see][]{belokurov18, helmi18, dimatteo19, gallart19, kruijssen20} -- well before the time interval simulated in this study. Other more recent interactions, such as the accretion of Sagittarius and of the Magellanic Clouds, may perturb the Galactic potential as well \citep[see, e.g.,][]{vasiliev21b} and we plan to investigate their impact on the properties of globular cluster streams in the future.

        For each realization, we generate an output file in an hdf5 format\footnote{\url{https://www.hdfgroup.org/solutions/hdf5/}} containing the values for the right ascension ($\alpha$), declination ($\delta$), distance from the Sun ($D$), along with the components for proper motion in the equatorial coordinate system ($\rm \mu_{\alpha}\cos(\delta)$ and $\rm \mu_\delta$), the line-of-sight velocity ($\rm v_{\ell os}$), longitude ($\ell$), latitude ($ b$), as well as the components for proper motion in the Galactic coordinate system ($\rm \mu_{\ell}\cos(\mathit{b})$ and $\rm \mu_b$) and the Galactocentric positions ($x, y, z$), velocities ($v_x, v_y, v_z$) and energy, $E$, of each particle in the simulated system. We used Astropy \citep{astropy13, astropy18} to convert the Galactocentric positions and velocities in the equatorial and Galactic quantities $\alpha, \delta, D, \rm \mu_{\alpha}\cos(\delta), \rm \mu_\delta, \rm v_{\ell os}, \ell, b, \rm \mu_{\ell}\cos(\mathit{b})$, and $\rm \mu_b$.

        For each particle, we also save its escape time $t_{\rm esc}$,  defined as the time at which the particle escapes from the cluster, that is, the time, $t,$ at which the particle satisfies the relation\footnote{If the particle is gravitationally bound to the cluster until the end of the simulation, $t_{\rm esc}$ is set equal to $-9999$.}:
        \begin{equation}
            E_{GC}= 0.5 \times \left( (v_x-v_{x, GC})^2+(v_y-v_{y,GC})^2+(v_z-v_{z,GC})^2\right)+\Phi_{GC} > 0,
        \end{equation}
        with $E_{GC}$ being the total specific energy of the particle relative to the cluster, that is, the sum of the potential energy, $\Phi_{GC}$, due to the gravitational field of the cluster (see Eq.~\ref{gcpot}), and of the kinetic energy, relative to the cluster barycenter, $T_{GC}=0.5 \times \left( \left(v_x-v_{x, GC}\right)^2+\left(vy-v_{y,GC}\right)^2+\left(vz-v_{z,GC}\right)^2\right)$, where $v_x, v_y$, and $v_z$ are its velocity components at time, $t$, and $v_{x,GC}, v_{y,GC}$, and $v_{z,GC}$  of the cluster barycenter at the same time. A positive value of $E_{GC}$ implies that the particle is no longer gravitationally bound to the cluster and, hence, it is lost in the field. Overall, the total volume of the whole set of  24 327 simulations, saved in hdf5 format, amounts to about 370 Gb.


\chapter{Gaps in stellar streams}
\section{Published results}

\section{Velocity distribution within the stream}

    \subsection{Self-Segregation and Stream Chilling}

    The classic collisionless boltzmann equation:
    \begin{equation}
        \frac{df}{dt} = 0 = \frac{\partial f}{\partial x} \frac{dx}{dt} + \frac{\partial f}{\partial v} \frac{dv}{dt}+ \frac{\partial f}{\partial t} 
    \end{equation}

    we are saying that the pusles drift with the same velocities thus $\frac{dv}{dt}=0$. I impart more assumptions, namely that: $\rho(x,t=0)=\delta(x)$ and that velocity is defined as a normal distribution that does not change over time. This means that I can write the initial distribution function as:
    
    \begin{equation}
        f(x,v,t=0) = \delta(x)\frac{1}{\sigma\sqrt{2\pi}}\textrm{exp}\left(-\frac{1}{2}\left(\frac{v-\langle v \rangle}{\sigma}\right)^2\right)
    \end{equation}    

    the solution to the evolution of the density is $\rho = \int f dv$, and you also need to perform this variable substitution $f(x,v,t) = f(x-vt,v,0)$

    \begin{equation}
        \rho(x,t) = \frac{1}{\sigma t \sqrt{2\pi} }\textrm{exp}\left(-\frac{1}{2}\left(\frac{x-\langle v \rangle t}{\sigma_v t}\right)^2\right)
    \end{equation}

    it's also useful to know the relative velocity of the impact site between one group and another. These things drift apart. How much faster does one group go ahead of another? 
    \begin{equation}
        \delta v_{ij} = \frac{x\prime}{t-iT} - \frac{x\prime}{t-jT}
    \end{equation}
    where $i,j$ are the indexes for the packet. Note that this only works for $t > nT$ where $T$ is the orbital period, or spacing between the impacts, $n$ is the number of pericenter passages. $t$ is the total simulation time. $x\prime$ is the position of the impact. It makes sense that the velocity of the particle is the position where the impact occured, divided by the time since it left the origin. 

\chapter{Conclusions}
\citet{2022A&A...664A..31C} Laia's paper that used the first version of the code.

In this chapter, I summarize the works undertaken in this thesis and reconnect it to the broader literature with proposing more avenues for investigation. 

\section{Summary}
    In Chapter 1, I outlined the importance of stellar streams from globular clusters, highlighting both their intrinsic scientific interest and their role as probes for a wide range of astrophysical questions. These include the chemical enrichment of the Universe, the assembly history of the Milky Way, pathways of black hole formation, and constraints on the global and local properties of the Galactic gravitational field—particularly in relation to the detection of dark matter.

    In Chapter 2, I presented the exact equations of motion used to model the formation of stellar streams from globular clusters, formulated as a restricted three-body problem. I then explained how these equations can be interpreted to describe the escape of stars through the Lagrange points under tidal forces, followed by their evolution via phase mixing to form the streams. I summarized the literature on the mechanisms by which perturbations can create gaps. I also discussed the limitations of our models: specifically, that without $N$-body simulations or analytical prescriptions for internal cluster dynamics, certain questions cannot be addressed.

    In Chapter 3, I introduced my simulation code, \texttt{tstrippy}, describing its numerical solution of the equations of motion, its performance in terms of accuracy and computation time, and a comparison of two numerical schemes. I explained how the code is written in Fortran and interfaced with Python via \texttt{f2py}, and provided a minimal, {\tiny non}-working example illustrating the workflow from the user's perspective.

    In Chapter 4, I presented \citet{2023A&A...673A..44F}, in which we simulated the expected tidal debris from the entire globular cluster catalogue. These simulations have been made publicly available and are now used by the community, including in searches for additional tidal debris beyond some of the clusters in our sample \citep{2025arXiv250705590K,2025ApJ...988...39W}.
    
    In Chapter 5, I present \citet{2025A&A...699A.289F}, where we investigated how the Palomar~5 stellar stream responds to the granularity of the Milky Way's gravitational field—specifically, the presence of other globular clusters. We showed that the internal dynamics of globular clusters can strongly influence the morphology of their stellar streams. In particular, gap formation from the fly-by of a massive perturber requires very specific stream conditions: regions close to the progenitor, where the morphology is still dominated by epicyclic overdensities, are unfavorable for gap formation, as different stellar packets respond differently to the perturbation and subsequently drift out of phase. This is a new contribution, as previous studies primarily considered the effects of random motions, such as velocity dispersion.

\section{Prospectives}
    One of the main strengths of this work is its focus on Milky Way streams and globular clusters. While idealised toy models are invaluable for testing new ideas and exploring a wide range of theoretical phenomena, our emphasis on the Milky Way allows us to ask more grounded questions: how often should these phenomena occur in our Galaxy, with what significance, and how do they connect directly to what we observe? By anchoring our models to the Milky Way, we can close the loop between theory and data.

    The ubiquity and sensitivity of stellar streams make them exceptional probes for inferring the properties of various mass distributions. This opens multiple avenues for future investigation. Some phenomena still require detailed numerical forward modelling—either to understand the underlying physical mechanisms or to obtain robust predictions for their expected occurrence rates. In other cases, where the theoretical picture is mature, we can turn directly to observations and begin the task of inference.
    
    \subsection{Dark Matter Subhalos}
        The prospect of detecting dark matter (DM) on scales smaller than those accessible in the extragalactic context is particularly compelling. Our goal is to perform a study similar to that in Chapter~5, but rather than limiting ourselves to a population of Galactic globular clusters, we will also include a $\Lambda$CDM-motivated population of dark matter subhalos (Boldrini et al., in prep). Figure~\ref{fig:mollweide-density-with-haloes.png} shows an example of such a population from a cosmological simulation of a Milky Way analogue. \citet{2024arXiv241213144A} recently carried out a similar analysis for the GD-1 stream and found that, over the last few billion years, GD-1 has on average one detectable gap from a DM subhalo encounter. \citet{2025arXiv250207781L} adopted this rate of $\sim1$ gap per stream to discuss the detectability of gaps in $\sim 50$ known Milky Way streams.

        \begin{figure}
            \includegraphics[width=\linewidth]{images/mollweide-density-with-haloes.png}
            \caption[Plausible $\Lambda$CDM dark matter subhalo population as seen from the Sun]{Plausible $\Lambda$CDM dark matter subhalo population as seen from the Sun. The subhalos are modelled as Plummer spheres, with their light integrated along the line of sight from the Sun and scaled according to the total halo mass. Overlaid is the integrated surface light density profile of the Milky Way's stellar disk, modelled with a Miyamoto--Nagai potential.}
            \label{fig:mollweide-density-with-haloes.png}
        \end{figure}

        There is considerable modelling work on gap formation in stellar streams. While we aim to make theoretical predictions for the expected number of gaps, these must be set in the context of a realistic Milky Way potential that includes known time-dependent and non-axisymmetric features. To date, most studies have examined either subhalo encounters in isolation \citep{2013ApJ...775...90C,2015MNRAS.450.1136E,2016MNRAS.463..102E,2016MNRAS.457.3817S,2024arXiv241213144A,2025arXiv250207781L} or the effect of the Galactic bar alone \citep{2016MNRAS.460..497H,2016ApJ...824..104P,2017NatAs...1..633P,2023A&A...678A.180T}. However, it is known that perturbations from baryonic structures can also generate gap-like signals, potentially mimicking the effects of DM subhalos \citep{2020ApJ...891..161I}. To correctly frame the inference problem, the false positive rate must therefore be quantified and calibrated.

        Ultimately, the inversion of this problem is a case study in Bayesian hierarchical modelling \citep{2020sdmm.book.....I}. Each gap detection is underdetermined \citep{2015MNRAS.450.1136E}: even in the absence of measurement uncertainties, a single gap cannot uniquely constrain the perturber's mass, size, time of impact, and relative velocity simultaneously. Nevertheless, with a sufficiently large sample of gaps and posterior distributions for each encounter, we can begin to infer statistical properties of the DM subhalo population.

        This is a challenging problem. We must account for other astrophysical processes that can erase gap signatures or create false positives, quantify their rates, and embed them in a hierarchical inference framework. The application of such an analysis to the Milky Way will require extremely high-quality data, which may be achievable with future LSST observations and the final Gaia data releases.

        As highlighted in Chapter~5, the coherence of a stream is crucial for gap survival, and this must be modelled accurately to avoid overestimating gap counts. In order to obtain proper predictions, we must accurate model stream generation. While full $N$-body simulations would be the most physically accurate way to model internal cluster dynamics and their mapping into streams, they are computationally prohibitive for the parameter space we must explore. Capturing variations in the Galaxy's potential, cluster internal dynamics, orbital initial conditions, DM subhalo populations, and the properties of multiple streams could require computing tens of thousands of stream realisations. $N$-body simulations have long been at the avant-garde of computational astrophysics, often driving innovations in both hardware and software. A notable example is the GRAPE (GRAvity PipE) series of special-purpose computers developed for gravitational $N$-body problems \citep{1991PASJ...43..841F,1997ApJ...480..432M}. For over a decade, GRAPE systems enabled simulations that would have been prohibitively slow on general-purpose hardware, pushing the limits of the field. Eventually, advances in general-purpose graphics processing units (GPUs) offered comparable performance with greater flexibility, leading to the widespread adoption of GPU-accelerated $N$-body codes \citep{2012MNRAS.424..545N,2015MNRAS.450.4070W}. Despite these hardware revolutions, the computational cost of the large ensembles of simulations required for our purposes remains high.

        Several methods have been developed to avoid the need for full $N$-body modelling. The ``streak-line'' method \citep{2012MNRAS.420.2700K} approximates streams as having the same orbital parameters as their progenitor. The action-angle formalism of \citet{2011MNRAS.413.1852E} describes each stream star as having a small offset from the progenitor's Hamiltonian, expressible via a second-order Taylor expansion. \citet{2014ApJ...795...95B} used this to develop the ``particle-spray'' method, which improves upon the streak-line model by introducing velocity dispersion into the streams. As noted by \citet{2015MNRAS.452..301F}, this is one of the most elegant stream modelling approaches to date.

        \citet{2015MNRAS.452..301F} also developed a prescriptive stream-generative model, fitting analytic functions to escape rates measured from $N$-body simulations. This method operates on the dynamical timescale of the Galaxy rather than the cluster's internal timescale, greatly speeding up computations.  

        The internal dynamics of globular clusters involve a rich range of processes \citep{1997A&ARv...8....1M}. While particle-spray and semi-analytic methods can be extremely efficient, extrapolating beyond the regime covered by their $N$-body calibrations can be risky. However, machine learning techniques \citep{2023ApJ...959...99T} offer a promising way to emulate $N$-body simulations without assuming a specific parametric form, providing both flexibility and speed.

        In summary, with upcoming improvements in both data quality and modelling techniques, there is strong potential to make significant progress in this field. Depending on computational constraints, I may either adopt an existing stream simulation method in place of \texttt{tstrippy} or implement a particle-spray approach within it.

    \subsection{The bulk gravitational field of the MW}

            \begin{enumerate}
                \item we wanted to explore many aspects of the MW that are time-dependent and see what features they leave on streams. 
                \item there are many of these and many studies that have investigated this stuff, as spoke about in Bonaca conroy review such as caotic orbits, the stellar bar, spiral arms, etc,. 
                \item however, we have run more simulations with the galactic bar and recognize that there is more theoretical understanding to be made about how the bars effect the stellar stream that what is yet reported in the literature. 
                \item Prelim result shown in Fig.~\ref{fig:pal5_with_bar}. Indeed, we know that the bar can create long streams, and that it can deflect them. But we have seen the bar stund the growth before\dots Malhan showed one of the twisting streams 
                \item Ibata 2024 made a good model. However it's still axis symmetric. A Malhan article showed that the structure of the MW can twist a stream. I see that too in my simulations. This mecanism needs to be understood theoretically better and also employed in the MW\dots
            \end{enumerate}


            \begin{verbatim}
                VIDEO: Pal5_longmurali_5000_monte_carlo_002_white_bg.mp4
            \end{verbatim}

            \begin{figure}
                \centering
                \begin{tabular}{ccc}
                    \includegraphics[width=.32\linewidth]{images/frame_0002.png}&
                    \includegraphics[width=.32\linewidth]{images/frame_0004.png}&
                    \includegraphics[width=.32\linewidth]{images/frame_0008.png}\\
                    
                    \includegraphics[width=.32\linewidth]{images/frame_0016.png}&
                    \includegraphics[width=.32\linewidth]{images/frame_0019.png}&
                    \includegraphics[width=.32\linewidth]{images/frame_0023.png}\\
                    
                    \includegraphics[width=.32\linewidth]{images/frame_0038.png}&
                    \includegraphics[width=.32\linewidth]{images/frame_0048.png}&
                    \includegraphics[width=.32\linewidth]{images/frame_0062.png}\\
                    
                    \includegraphics[width=.32\linewidth]{images/frame_0065.png}&
                    \includegraphics[width=.32\linewidth]{images/frame_0066.png}&
                    \includegraphics[width=.32\linewidth]{images/frame_0104.png}\\
                \end{tabular}
                \caption[The presence of a stellar bar with different rotational speeds damaging the Palomar~5 stream]{Various simulations of producing the Palomar~5 stream using \texttt{tstrippy} with 5000 particles. The galaxy was modeled using \citet{2017A&A...598A..66P} and the bar was modeled with \citet{1997MNRAS.291..717M}, as describe in chapter~4. However, we vary the bar pattern speed. We run the simulation for the same initial conditions but with the different bar pattern speeds. We varied the bar pattern speed between 25-61~km/(s~kpc) with a 150 samples. There is an animated version of this graph availble on the online version of this thesis.}
                \label{fig:pal5_with_bar}
            \end{figure}

    \subsection{Multiple-stellar populations, Stellar evolution, and globular cluster formation and internal dynamics, stellar populations}

        What if we wanted to go further? Indeed at somepoint, this is just considered other work or future considerations, or other research questions entirely. 

        What if we wanted to further? this is certainly outside the scope of our work, but we could consider the globular cluster formation. 

        The complexity doesn't end here. For instance, we are unsure how globular clusters form. There are some mechanisms proposed (cite the extra galactic GC review). That review summarizes them as: (1), (2), (3). However, to completely understand them would involve understanding the chemical enrichment history of the universe. Starting from big-bang nucleosysntehsis, to first generation stars (POP III), and how these first stars chemically enrich the environment, and how the clusters form from said environment. \citet{2022A&A...668A.191C} studied cluster formation in environments that were enriched by just pop II stars or a mixture of pop III/II stars and showed that pop III stars were necessary to enrich the environment enough to reproduce current cluster qualtities. 

        However, from my literature review there does seem to be a gap linking the chemical environment that could be enriched from a combination of pop II/popIII stars and how cluster formation. At least we know clusters are at least second generation (pop II) stars, since the first stars (pop III) form massive and die young \citep{2002ApJ...571...30S}.

        Ideally, we would like to know about the stars within the cluster to then infer the environment that it must have been born in. Extra-galactic census of globular clusters show that there isa bi-modality in color, making people believe that there are at least two generations or two different mechanisms for forming them. 

        \citet{2025arXiv250507491V} is looking at black hole formation inside of globular clusters that are being born. 

        \citet{2024A&A...681A..45L} Elena and Alessandra investigate a scenario that considers multiple populations in globular clusters where the second population is much less massive than the first, but evaporation and tidal stripping processes can mean that there present day contributions are similar to one another. 

        \citet{2025MNRAS.537.2342C} is looking at the dynamics of the different populations within 47 Tuc.

        \citet{2024MNRAS.529.2413U} said that there's evidence of multiple populations within the stellar streams. Something expected and now observationally confirmed     

        \citet{2025MNRAS.540.1235C} presents a cool scenario for investigating globular cluster formation during galaxy formation and mergers. They discuss the chemical environment the clusters form in as an enrichment process from POP III and POP II stars. I must read this more in detail. It is fascinating work. 

        \citet{2004AJ....127.2753D} presented a great work on Palomar~5 and found the parameters of a king model that best reproduce Palomar~5. Perhaps I can take form this study to have the best description of the tidal tails that corresponds to Palomar~5, and iterate over this to have the best description of of the tails as possible. 

        \citet{2006ARA&A..44..193B} made a review on extra galactic globular clusters. They discuss things like the observation evidence, the globular cluster mass function (all clusters), and find it to have an equal power spectrum, but the issue is that you don't know where to truncate the power law. maybe at lower end you can argue $10^4 M_\odot$ from two body relaxation and a tidal field. However, the upper limit is hard since you already don't expect that many, and the inclusion of a couple more high mass members can significantly alter the ``mean'' mass GC. They also note how a lot of cluster properties match the galactic properties. They talk about a bi-modal distribution. It really isn't that bimodal, there's a ton of overlap. There's a bluer one and a redder one. One of them might form during the collisions between galaxies while the other could be formed deeped in the potential wells. They also said that as of 2014 there is evidence that GCs are still forming. They also discuss the chemical abundance patterns within the clusters. It's complicated stuff, I wish I was better at stellar physics and knowing all the different sequences\dots

        Something about the chemical abundance patterns of the halo stars that can be consistent with the globular clusters \dots

        Schiavon (2017) showing that there exist many N-rich stars which have chemical abundance patterns similar to stars within the clusters all over the galactic field. 

        Fernández-Trincado, José G (2021) says that there are some metal rich cluster debris, which is interesting 

        Fernández-Trincado, J. G (2017) said they found 11 stars with the chemical abundance patterms of second general stars within globular cluster \dots 

\section{Future modeling techinques}

    I want to integrate these discussions in smoothly with the other scientific questions and not separate the modelling techinques. 

    \subsection{N-body codes}
    \citep{2012MNRAS.424..545N} showed the GPU's can replace GRAPE

    \citet{2018ComAC...5....2V} talks about a series of papers between a larger collaboration of people who specialize in collisional dynamics and who have performed a series of workshops together. The introduction stated that the collaboration wants to tackle many open questions regarding stellar clusters and build the necessary codes to interprete the future large quantity of data that was destined to come. It has now come since the review was 2018. An interesting point was that in general globular clusters are approximated as being orderless, i.e. isotropic but order does present itself within these stelalr systems. Another large problem is no one knows what a good set of initial conditiosn is. Unresolved binaries pose a problem because you can overestiamte the total mass of the system. If I talk about this review, I should probably discuss some of the results from the papers that is builds on or at least their techinques.

    The MODEST review led me to discover AMUSE, which is an framework for integrating various astrophysical codes for solving 4 types of problems: gravitational dynamics, radiative transfer, hydrodynamics, and stellar evolution. The codes are written by the community and are interfaced together with Amuse. The user end is python. I have spent some time reading the book, which is instructive and well written. Steve McMillian is one of the authors. The code has a large support on GitHub and is still being developped. I have had trouble trying to install the code. It seems as though their documentation is incoherrent. At one place, it said `pip' is the easiest way to install. It didn't work. In another place, I was instructed to install a zipped up tarball. The setup failed becuase it expected there to be a .git file in the directory. I successfully downloaded the code by cloneing the repository, despite the fact that this was not recommended. I can use some aspects of the code but not all of them. For instance, my memory tells me that about 80\% of the test suite passed, thus many scripts failed. This was when I only installed the frame-work, which was advised since installing the whole package is huge and unnecessary since I am not solving all astrophysical problems. However, I wasn't able to use one of the gravity solvers that was presented in the textbook `AstrophysicalRecipes The art of AMUSE'. The install still has some codes that failed for instance: amuse-adaptb, amuse-hermite-grx, amuse-mi6. However, I'm hoping that this isn't necessary. I want to educate myself and make some examples. 

    Installing other codes and figuring out their functionalities to me has never been trivial. This is similar to galpy when I tried to figure out particle spray method and got less than good results. Agama also confused me a bit. The main point is that for each package, at the end of the day I decided that it was easier and better if I solved the problem myself with my own code. Because, even with the other packages, I know that they can be used to solve other astrophysical problems and it wasn't clear to me how to make the codes solve my specific set of of the restricted three body problems in a potential with other perturbers flying around. 

    In this search, I also discovered another review called \textit{Computational methods for collisional stellar systems} by Spurzem and Kamlah 2023. It is also interesting and instructive. I found it insightful when they called NBody an industry. I think the story of GRAPE and Makino is really interesting, how he build dedicated hardware for the nbody problem which were great for 10 years but were quickly replaced by GPU technology. 









\backmatter
\appendix
\addchapnonumber{Appendices}
\hypertarget{Appendices}{You're my wonderwall}
\section{Galactic Astronomy}
\subsection*{Unravelling UBC 274: A morphological, kinematical, and chemical analysis of a disrupting open cluster}
\reference{\citep{2022A&A...664A..31C} Casamiquela}
\subsection*{Charting the Galactic Acceleration Field. II. A Global Mass Model of the Milky Way from the STREAMFINDER Atlas of Stellar Streams Detected in Gaia DR3}
\citet{2024ApJ...967...89I}

\section{Asteroid science}
\reference{\citep{2023A&A...676A...5F}\\ \citep{2024A&A...682A..64B}}
\citet{2023A&A...676A...5F}. \citet{2024A&A...682A..64B}. These two papers changed my life. 

\addchapnonumber{List of Publications}

\section*{Peer-reviewed papers}

\MyprodwithDOI{\textbf{Ferrone, S.}, Montuori, M., Di Matteo, P., Mastrobuono-Battisti, A., Ibata, R., Bianchini, P., Khoperskov, S., Leclerc, N., Hottier, C., Stein, E., Valls-Gabaud, D., Snaith, O. N., \& Haywood M.}{Gaps in stellar streams as a result of globular cluster flybys: The case of Palomar 5}{Astronomy \& Astrophysics. Volume 699, July 2025}{10.1051/0004-6361/202553923}

\MyprodwithDOI{Ibata, R., Malhan, K., Tenachi, W., \dots, \textbf{Ferrone, S.}, \dots, \& Yuan, Z.}{Charting the Galactic Acceleration Field. II. A Global Mass Model of the Milky Way from the STREAMFINDER Atlas of Stellar Streams Detected in Gaia DR3}{The Astrophysical Journal, 2024, 967.2: 89}{10.3847/1538-4357/ad382d}

\MyprodwithDOI{\textbf{Ferrone, S.}, Delbo, M., Avdellidou, C., Melikyan, R., Morbidelli, A., Walsh, K., \& Deienno, R.}{Identification of a 4.3 billion year old asteroid family and planetesimal population in the Inner Main Belt}{Astronomy \& Astrophysics, 2023, 676: A5.}{10.1051/0004-6361/202245594}

\section*{International conferences (first author)}



\begin{singlespace}
\setlength\labelalphawidth{0em}
\small\printbibliography[heading=bibintoc,title=Bibliography]
\end{singlespace}

\end{document}