This thesis investigates stellar streams originating from globular clusters in the Milky Way. Stellar streams are long, thin structures composed of stars that have escaped a host stellar system, forming coherent tails that can span large regions of the sky. Globular clusters are dense gravitationally bound systems containing hundreds of thousands to millions of stars. They are one source of stellar streams. In Chapter 4, I present our study in which we simulated the expected distribution of stellar streams originating from the entire Milky Way globular cluster system. This study was the first of its kind. Stellar streams have high hopes for being inferential tools for inferreing the presence of dark matter sub halos within the Galaxy. These halos may perturb the stellar streams leaving ``gaps'' in their wake. In Chapter~5, we explored how the collective gravitational influence of all globular clusters can also perturb produce such gaps.

The structure of the thesis is as follows. The remainder of the introduction provides background on stellar streams, globular clusters, and their astrophysical context, as well as an overview of the current state of the field. Chapter~2 outlines the physical framework used to model stream formation and interpret their morphology. Chapter~3 describes the numerical methods employed in the simulations, including convergence tests and estimates of computational cost. Chapters~4 and~5 present two published studies, with the final chapter discussing these results in the broader context of the literature and highlighting future directions.
\begin{figure}
    \centering
    \includegraphics[width=\linewidth]{images/S5MilkywayStreams.jpg}
    \caption{An artist's rendition of a galaxy surrounded by stellar streams. Credit: James Josephides and S$^5$ Collaboration \citep{2019MNRAS.490.3508L}.}
    \label{fig:S5MilkywayStreams}
\end{figure}

\section{General context}
    Before explaining why Milky Way globular clusters and stellar streams are scientifically interesting, it is important to first set the scene. This thesis fits neatly within the field of galactic astronomy, which, at its core, studies the current state of our galaxy and the processes that shaped its formation within the broader context of the universe.

    If we start this narrative from the beginning and place stellar streams and globular clusters in the timeline of cosmic history, their importance becomes clearer. The story begins best at the very start. The Lambda Cold Dark Matter ($\Lambda$CDM) cosmological model is currently the leading theory of the universe, successfully unifying a variety of observational evidence—from the Cosmic Microwave Background Radiation and the large-scale distribution of galaxies to the accelerating expansion of the universe \citep{2001LRR.....4....1C,2022NewAR..9501659P}.

    Shortly after the Big Bang, conditions allowed protons, neutrons, and electrons to form and interact. For a few minutes, these particles collided and fused into heavier elements in a process known as Big Bang Nucleosynthesis \citep{2007ARNPS..57..463S}. When this phase ended, the universe's composition was mostly hydrogen, deuterium ($^2$H), helium-4 ($^4$He), and trace amounts of helium-3 ($^3$He) and lithium-7 ($^7$Li). By mass, hydrogen made up roughly 75\%, and helium about 25\% of the primordial universe \citep{1966ApJ...146..542P,2016RvMP...88a5004C}.

    Dark matter, which accounts for about five times more mass than ordinary matter, played a critical role in structure formation \citep{2020A&A...641A...6P}. In the early universe, dark matter was distributed nearly uniformly but developed gravitational instabilities that caused it to collapse into a cosmic web of filaments and nodes. These massive nodes created deep gravitational wells that attracted ordinary matter \citep{1974ApJ...187..425P}. The infalling gas subsequently cooled and formed stars. The resulting complex of stars, gas, and dark matter constitutes a galaxy \citep{2008LNP...740.....P,2010gfe..book.....M}.

    Galaxies contain stars that are born, fuse hydrogen and helium into heavier elements, and eventually die, often as supernovae, enriching the interstellar medium with these heavier elements \citep{2019A&ARv..27....3M}. This process is a strong function of cosmic time as the chemical evolution of the universe becomes more metal rich. For example, the first generation of stars, known as Population III (Pop III) stars\footnote{Stellar populations are named in reverse order of discovery. Population I stars are the youngest, metal-rich stars (including the Sun), Population II stars are older and metal-poor, and Population III stars are the very first, metal-free stars.}, formed in a very different environment, one devoid of metals \citep{2002Sci...295...93A,2005SSRv..117..445G,2013RPPh...76k2901B}. 

    The chemical composition of the gas is crucial since it influences the initial mass function (IMF) of stars. Stars formed from pristine, metal-free gas tend to have a top-heavy IMF, favoring the formation of massive stars \citep{2002ApJ...571...30S,2006MNRAS.369..825S}. In contrast, even small amounts of metals introduced into the gas can dramatically shift the IMF toward the formation of lighter stars \citep{2021MNRAS.508.4175C}.

    Globular clusters are, to first approximation, single-stellar populations: their constituent stars formed from the same molecular cloud over a timescale shorter than the internal dynamical time \citep{1988ApJ...324..288A,2009MNRAS.397..954F,2014PhR...539...49K}. Owing to their uniform chemical composition, stellar evolution within a cluster proceeds along the color-magnitude diagram as a function solely of initial mass \citep{2013sse..book.....K}. This property enables precise age determinations from photometric observations, a method that historically provided the first robust lower limits on the age of the Universe \citep{1959MNRAS.119..124H,1970ApJ...162..841S,1985A&A...147..169G,1992ApJ...400..265M}. With their low metallicities and advanced ages, globular clusters are dominated by Population~II stars, indicating that they formed in environments enriched exclusively by the earliest generations of stars \citep{2022A&A...668A.191C}. They thus constitute valuable fossil records of the initial phases of star formation and chemical evolution in the Universe.
    
    Globular clusters are among the most ubiquitous stellar systems in the Universe \citep{2006ARA&A..44..193B,2019ARA&A..57..227K}. They are found in virtually all galaxy types, from dwarfs to giant ellipticals, and their total number correlates with global properties of their host galaxies such as mass and luminosity \citep[e.g.,][]{2013ApJ...772...82H,2018MNRAS.481.5592F}. In the local Universe, we can directly observe the formation of massive bound stellar systems—so-called young massive star clusters—which may represent present-day analogs of the GC formation process \citep[e.g.,][]{2010ARA&A..48..431P,2020SSRv..216...69A}, though whether all such systems will evolve into globular clusters remains an open question.

    Globular clusters serve as unique astrophysical laboratories, offering insights into stellar dynamics, stellar evolution, and galaxy assembly. Their evolution involves stellar structure and evolution, gravitational dynamics, and relativistic effects in dense environments. For instance, close stellar encounters and binary interactions—such as mass transfer or mergers—are common in such environments and significantly shape cluster evolution \citep{2004MNRAS.349..129D,2016MNRAS.458.1450W,2024MNRAS.528.5119A}. Additionally, globular clusters are suspected to be a pathway for forming intermediate-mass black holes \citep{2013MNRAS.432.2779B,2015MNRAS.454.3150G}. The origin of IMBHs remains uncertain, as their masses are too large to be explained by isolated stellar evolution, yet too small to fall into the supermassive category \citep{2020ARA&A..58..257G}. 
    
    Moreover, while traditionally considered as single stellar populations—homogeneous in age and chemical composition—decades of spectroscopic and photometric evidence have revealed the presence of multiple stellar populations in most GCs \citep{2008MNRAS.391..825D,2012A&ARv..20...50G,2018ARA&A..56...83B}. The origin of these multiple populations remains debated, with proposed explanations ranging from self-enrichment to accretion of external material, but no consensus has yet emerged.

    \textbf{tie in } \citet{2025arXiv250116438K}'s and \citet{2018RSPSA.47470616F}

    A striking property of many GC systems is the bimodality in their metallicity distribution, with one population of metal-poor GCs and another of metal-rich GCs \citep[e.g.,][]{2006ARA&A..44..193B, 2015ApJ...806...36H}. This bimodality is now widely interpreted as evidence for multiple formation channels: metal-rich GCs likely formed in situ within the main progenitor galaxy during intense star-formation episodes, whereas metal-poor GCs were predominantly accreted from lower-mass satellites. The persistence of this bimodality across a wide range of galaxies suggests that hierarchical accretion has been a fundamental process shaping GC systems.

    From the perspective of galaxy formation, the hierarchical model of structure growth posits that galaxies assemble through repeated mergers and accretion of smaller systems \citep{2015ARA&A..53...51S}. In this framework, GCs can form in situ within the main progenitor or be accreted from satellite galaxies \citep[e.g.,][]{2018MNRAS.479.4760F,2020MNRAS.498.2472K,2023A&A...673A..86P,2024MNRAS.528.3198B,2025A&A...693A.155P}. Consequently, the spatial distribution, kinematics, chemical abundances, and metallicity substructure of a galaxy's GC system retain valuable information about its merger history and assembly pathways. This makes GCs not only tracers of early star formation, but also fossil records of the build-up of their host galaxies.
    

    \textit{transition to how stellar streams using the galactic environment }\dots

    \textit{the galactic gravitaitonal field is stronger on the close side cluster than on the far. This differential force is known as tidal froces and rips the body apart. the stars that get stripped from a the cluster trace roughly trace out the orbit of the cluster (cite all the papers that support this and say tricky tricky). To give an idea on why this is incredibly useful, consider the long time scales involved with astronomy. For instance, the Sun takes roughly 220 million years to complete a single orbit around the Galaxy. Stellar streams, by encoding orbital trajectories of their progenitor clusters, provide a unique window into millions of years of dynamical history. By characterizing their shapes and extents, we gain precise constraints on the Galactic gravitational potential.}

    Stellar streams, formed by stars gradually escaping from globular clusters due to tidal forces, are especially valuable. These streams preserve dynamical information about both their progenitor clusters and the gravitational potential of the host galaxy. The timing and properties of escaping stars can reveal the recent internal evolution of the cluster itself \citep{1972ApJ...178..623T,1995AJ....109.2553G}. 

    Moreover, as stellar streams trace arcs across the sky, they serve as natural probes of the Galactic gravitational field. Their coherent kinematics allow constraints on the mass distribution and substructure of the Milky Way \citep{2011MNRAS.417..198V}. This dual insight into cluster dynamics and galactic environment makes streams a uniquely powerful tool for modern galactic archaeology and dynamics.

    \textit{cite some papers about how stellar streams get discovered with Grilmair and some early works that expected this from evaporating globular clusters \dots}

    \textit{now transition into gaia}

\section{The state of the art}

\citet{2021ApJ...909L..26B} showed that streams can also be used to get at the accretion history of the Milky Way. 

\textbf{cite Malhan papers} particularly the one showing how two streams were the same but one got twisted. 

\textbf{cite the bonaca papers}

\textbf{cite the gaia papers}

\textbf{cite ibata papers}

\textbf{other stellar stream papers}
