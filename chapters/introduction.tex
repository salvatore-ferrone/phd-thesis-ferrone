This thesis investigates stellar streams originating from globular clusters in the Milky Way. Stellar streams are long, thin structures composed of stars that have escaped their host stellar system, forming coherent tails that can span large regions of the sky (Fig.~\ref{fig:S5MilkywayStreams}). Globular clusters were first systematically cataloged by Charles Messier in the late 1700s--not for their scientific interest at the time, but to help comet hunters avoid mistaking these nebula-like objects for new comets \citep{1781cote.rept..227M}. Many of the most prominent globular clusters, along with diffuse nebulae and external galaxies, are included in the Messier catalog (Fig.~\ref{fig:All_messier_objects}). Physically, globular clusters are dense, gravitationally bound systems containing hundreds of thousands to millions of stars, and are an important source of stellar streams.
\begin{figure}
    \centering
    \includegraphics[width=\linewidth]{images/S5MilkywayStreams.jpg}
    \caption[Artist Rendition of Stellar Streams]{An artist's rendition of a galaxy surrounded by stellar streams. Credit: James Josephides and S$^5$ Collaboration \citep{2019MNRAS.490.3508L}.}
    \label{fig:S5MilkywayStreams}
\end{figure}
In Chapter 4, I present our study in which we simulated the expected distribution of stellar streams originating from the entire Milky Way globular cluster system. This study was the first of its kind. Stellar streams have high hopes for being inferential tools for fine details of the gravitational field of the Milky Way, notably for constraining the presence of \textit{dark mater subhalos}. These halos may perturb the stellar streams leaving ``gaps'' in their wake. Since structure of stellar streams are sensitive to local and global gravitational field, all phenomena that can influence the streams must be well calibrated to ensure proper inference of the distribution of dark matter within the Milky Way. In Chapter~5, we explored how the collective gravitational influence of all globular clusters can perturb streams to produce such gaps.

\begin{figure}
    \centering
    \includegraphics[width=\linewidth]{images/All_messier_objects.jpg}
    \caption[Messer objects]{The Messier catalog containing ``nebulae-like'' objects: planetary nebulae, diffuse nebulae, supernovae remenants, open clusters, globular clusters, and foreign galaxies. By Michael A. Phillips, an amateur astronomer. - http://astromaphilli14.blogspot.com.br/p/m.html official blog, CC BY 4.0, https://commons.wikimedia.org/w/index.php?curid=38121043}
    \label{fig:All_messier_objects}
\end{figure}

The structure of the thesis is as follows. The remainder of the introduction provides background on stellar streams, globular clusters, and their astrophysical context, as well as an overview of the current state of the field. Chapter~2 outlines the physical framework used to model stream formation and interpret their morphology. Chapter~3 describes the numerical methods employed in the simulations, including convergence tests and estimates of computational cost. Chapters~4 and~5 present two published studies, with the final chapter discussing these results in the broader context of the literature and highlighting future directions.

\section{General context}
    Before explaining why Milky Way globular clusters and stellar streams are scientifically interesting, it is important to first set the scene. This thesis fits neatly within the field of galactic astronomy, which, at its core, studies the current state of our galaxy and the processes that shaped its formation within the broader context of the universe.

    If we start this narrative from the beginning and place stellar streams and globular clusters in the timeline of cosmic history, their importance becomes clear. The story begins best at the very start. The Lambda Cold Dark Matter ($\Lambda$CDM) cosmological model is currently the leading theory of the universe, successfully unifying a variety of observational evidence from the Cosmic Microwave Background Radiation and the large-scale distribution of galaxies to the accelerating expansion of the universe, etc. \citep{2001LRR.....4....1C,2022NewAR..9501659P}.

    Shortly after the Big Bang, conditions allowed protons, neutrons, and electrons to form and interact. For a few minutes, these particles collided and fused into heavier elements in a process known as Big Bang Nucleosynthesis \citep{2007ARNPS..57..463S}. When this phase ended, the universe's composition was mostly hydrogen, deuterium ($^2$H), helium-4 ($^4$He), and trace amounts of helium-3 ($^3$He) and lithium-7 ($^7$Li). By mass, hydrogen made up roughly 75\%, and helium about 25\% of the primordial universe \citep{1966ApJ...146..542P,2016RvMP...88a5004C}.

    $\Lambda$CDM positis the existence of Dark matter and that it has about five times more mass than ordinary matter \citep{2020A&A...641A...6P}. Dark plays a critical role in galaxy formation. In the early universe, dark matter was distributed nearly uniformly. Over dense regions caused it to collapse into a cosmic web of filaments and nodes. These massive nodes created deep gravitational wells that attracted ordinary matter \citep{1974ApJ...187..425P}. The infalling gas subsequently cooled and formed stars. The resulting complex of stars, gas, and dark matter constitutes a galaxy \citep{2008LNP...740.....P,2010gfe..book.....M}.

    Galaxies contain stars that are born, fuse hydrogen and helium into heavier elements, and eventually die, often as supernovae, enriching the interstellar medium with these heavier elements \citep{2019A&ARv..27....3M}. The stellar formation process is a strong function of cosmic time as the chemical evolution of the universe becomes more metal rich. For example, the first generation of stars, known as Population III (Pop III) stars\footnote{Stellar populations are named in reverse order of discovery. Population I stars are the youngest, metal-rich stars (including the Sun), Population II stars are older and metal-poor, and Population III stars are the very first, metal-free stars.}, formed in a very different environment, one devoid of metals \citep{2002Sci...295...93A,2005SSRv..117..445G,2013RPPh...76k2901B}. 

    The chemical composition of the gas is crucial since it influences the initial mass function (IMF) of stars. Stars formed from pristine, metal-free gas tend to have a top-heavy IMF, favoring the formation of massive stars \citep{2002ApJ...571...30S,2006MNRAS.369..825S}. In contrast, even small amounts of metals introduced into the gas can dramatically shift the IMF toward the favoring the formation of lighter stars \citep{2021MNRAS.508.4175C}.

    Globular clusters are, to first approximation, single-stellar populations: their constituent stars formed from the same molecular cloud over a timescale shorter than the internal dynamical time \citep{1988ApJ...324..288A,2009MNRAS.397..954F,2014PhR...539...49K}. Owing to their uniform chemical composition, stellar evolution within a cluster proceeds along the color-magnitude diagram as a function solely of initial mass \citep{2013sse..book.....K}. This property enables precise age determinations from photometric observations, a method that historically provided the first robust lower limits on the age of the Universe \citep{1959MNRAS.119..124H,1970ApJ...162..841S,1985A&A...147..169G,1992ApJ...400..265M}. With their low metallicities and advanced ages, globular clusters are dominated by Population~II stars, indicating that they formed in environments enriched exclusively by the earliest generations of stars \citep{2022A&A...668A.191C}. They thus constitute valuable fossil records of the initial phases of star formation and chemical evolution in the Universe.
    
    Globular clusters are among the most ubiquitous stellar systems in the Universe \citep{2006ARA&A..44..193B,2019ARA&A..57..227K}. They are found in virtually all galaxy types, from dwarfs to giant ellipticals, and their total number correlates with global properties of their host galaxies such as mass and luminosity \citep[e.g.,][]{2013ApJ...772...82H,2018MNRAS.481.5592F}. In the local Universe, we can directly observe the formation of massive bound stellar systems—so-called young massive star clusters—which may represent present-day analogs of the GC formation process \citep[e.g.,][]{2010ARA&A..48..431P,2020SSRv..216...69A}, though whether all such systems will evolve into globular clusters remains an open question.

    Globular clusters serve as unique astrophysical laboratories, offering insights into stellar dynamics, stellar evolution, and galaxy assembly. Their evolution involves stellar structure and evolution, gravitational dynamics, and relativistic effects in dense environments. For instance, close stellar encounters and binary interactions—such as mass transfer or mergers—are common in such environments and significantly shape cluster evolution \citep{2004MNRAS.349..129D,2016MNRAS.458.1450W,2024MNRAS.528.5119A}. Additionally, globular clusters are suspected to be a pathway for forming intermediate-mass black holes (IMBHs) \citep{2013MNRAS.432.2779B,2015MNRAS.454.3150G}. The origin of IMBHs remains uncertain, as their masses are too large to be explained by isolated stellar evolution, yet too small to fall into the supermassive category \citep{2020ARA&A..58..257G}. 
    
    Moreover, while traditionally considered as single stellar population that are homogeneous in age and chemical composition, decades of spectroscopic and photometric evidence have revealed the presence of multiple stellar populations in most GCs \citep{2008MNRAS.391..825D,2012A&ARv..20...50G,2018ARA&A..56...83B}. The origin of these multiple populations remains debated, with proposed explanations ranging from self-enrichment to accretion of external material, but no consensus has yet emerged.

    \citet{2025arXiv250116438K} proposes that the main formation of globular cluster formation is at high redshift galaxies in high pressure environments. However, other mechanisms can exist as well and contribute to this.
    

    \citet{2018RSPSA.47470616F} also discusses two different formation scenarios. That GCs form naturally at high redshift. Or that they form within dark matter sub halos. 
    
    \citep{2016ApJ...823...52K} talks about  the formation of globular clusters within dark matter sub-halos. it's awesome! they discuss prestine gas and also evoke enrichment from pop III and pop II stars. They ran some hydro dynamical simultions. 

    A striking property of many GC systems is the bimodality in their metallicity distribution, with one population of metal-poor GCs and another of metal-rich GCs \citep[e.g.,][]{2006ARA&A..44..193B, 2015ApJ...806...36H}. This bimodality is now widely interpreted as evidence for multiple formation channels: metal-rich GCs likely formed in situ within the main progenitor galaxy during intense star-formation episodes, whereas metal-poor GCs were predominantly accreted from lower-mass satellites. The persistence of this bimodality across a wide range of galaxies suggests that hierarchical accretion has been a fundamental process shaping GC systems.

    From the perspective of galaxy formation, the hierarchical model of structure growth posits that galaxies assemble through repeated mergers and accretion of smaller systems \citep{2015ARA&A..53...51S}. In this framework, GCs can form in situ within the main progenitor or be accreted from satellite galaxies \citep[e.g.,][]{2018MNRAS.479.4760F,2020MNRAS.498.2472K,2021ApJ...920...51M,2022ApJ...930L...9M,2023A&A...673A..86P,2024MNRAS.528.3198B,2025A&A...693A.155P}. Consequently, the spatial distribution, kinematics, chemical abundances, and metallicity substructure of a galaxy's GC system retain valuable information about its merger history and assembly pathways. This makes GCs not only tracers of early star formation, but also fossil records of the build-up of their host galaxies.

    In essence, globular clusters are involved in many astrophysical processes such as the hierarchical formation of galaxies, formation of intermediate black holes, and being products of barely enriched gas in the early states of the universe \citep{2016ApJ...823...52K,2025arXiv250116438K}. Fully understanding their formation and evolution ties neatly into a variety of astrophysical problems.

    Not only are globular clusters fascinating for stellar formation, evolution, rich stellar dynamics, black hole formation, and galactic build up, but also for how they form \textit{stellar streams}. Globular clusters can evaporated and eject low mass stars due to their own internal dynamics \citep{2003gmbp.book.....H}. Additionally, fact the galactic tidal froces strip stars from the globular cluster as well. \citep{2007ApJ...659.1212M}
    
    \textit{some observational history of stellar streams}

    \citep{1992ApJ...386..519O} did the Fokker-planck equations \citet{1999A&A...352..149C} also used N body simulations to predict tidal tails.  \textit{some more papers on the first simulations and predictions of tidal tails}

    Stellar streams, formed by stars gradually escaping from globular clusters due to tidal forces, are especially valuable. These streams preserve dynamical information about both their progenitor clusters and the gravitational potential of the host galaxy. The timing and properties of escaping stars can reveal the recent internal evolution of the cluster itself \citep{1972ApJ...178..623T}.  \citet{1995AJ....109.2553G} did the first discovery.

    \textit{on the use case of stellar streams }

    Moreover, as stellar streams trace arcs across the sky, they serve as natural probes of the Galactic gravitational field. Their coherent kinematics allow constraints on the mass distribution and substructure of the Milky Way \citep{2011MNRAS.417..198V}. This dual insight into cluster dynamics and galactic environment makes streams a uniquely powerful tool for modern galactic archaeology and dynamics.

    \textit{how to tie it all together? and link it to the modern era with Gaia?}

    \citet{2010ApJ...712..260K} used the GD-1 stream to try and constrain the gravitational potential.

\section{The state of the art}
    Gaia $\rightarrow$ other data sets $\rightarrow$ \texttt{streamfinder} and other discoveries $\rightarrow$ advances in simulations in udnerstanding the physics of the environment on the creation of stellar streams $\rightarrow$ The studies of Malhan and Bonaca that are begining to have enough information to start tackeling some of these inference problems 

    The current era of Galactic astronomy is undoubtedly defined by the \emph{Gaia} space mission \citep{2016A&A...595A...1G,2016A&A...595A...2G,2018A&A...616A...1G,2021A&A...650C...3G,2023A&A...674A...1G}. The European Space Agency's space-based observatory has been conducting continuous observations of the sky to perform precise astrometric measurements. Operating since 2014, Gaia improves its data volume and quality each year through repeated parallax measurements.  As of Data Release~3, the mission has provided distances and proper motions for nearly two billion stars in the Galaxy, along with millions of radial velocity measurements \citep{2023A&A...674A...1G}. This yields five-dimensional phase-space information (positions and on-sky velocities) for about 1-2\% of the Galactic stellar population. Although the radial velocities are limited to relatively bright stars, the resulting dataset is still unprecedented in scope.

    When additional measurements are required, \emph{Gaia} data are often complemented with observations from other facilities.  For example, \emph{Gaia} struggles to determine parallaxes in the crowded interiors of globular clusters, where high stellar densities pose challenges for its instruments \citep{2017MNRAS.467..412P}. To address this, \citet{2021MNRAS.505.5957B} combined \emph{Gaia}~EDR3 astrometry with \emph{Hubble Space Telescope} observations, enabling precise distance measurements for 162 globular clusters.

    Several ground-based spectroscopic surveys are specifically designed to extend \emph{Gaia}'s radial velocity measurements to fainter magnitudes or more crowded regions.  Examples include the Apache Point Observatory Galactic Evolution Experiment (APOGEE; \citep{2017AJ....154...94M}), the Gaia-ESO Survey \citep{2023A&A...676A.129H}, and upcoming wide-field facilities such as 4MOST \citep{2019Msngr.175....3D} and WEAVE \citep{2014SPIE.9147E..0LD}.  Other large surveys primarily focus on providing detailed chemical abundances for a wide range of stellar populations, including GALAH \citep{2012ASPC..458..421Z}, the Gaia-ESO Survey \citep{2023A&A...676A.129H}, PAN-STARRS \citep{2016arXiv161205560C}, and APOGEE \citep{2017AJ....154...94M}, often delivering both radial velocities and multi-element abundance measurements in the same observations.

    These large data sets have led to a rapidly increasing number of stellar stream discoveries. Earlier searches relied on detailed analyses of color-magnitude diagrams and isochrone fitting. With the advent of the \texttt{streamfinder} algorithm \citep{2018MNRAS.477.4063M,2018MNRAS.478.3862M}, many streams could be detected simultaneously. This was possible because \texttt{streamfinder} agnostically applies a friend-finding algorithm to identify coherent stellar groups in the data. It was initially applied to a small number of streams to reassess their properties and discuss their characteristics in light of the improved quality and quantity of the data \citep{2019NatAs...3..667I,2020ApJ...891..161I}.

    Encouraged by these initial successes, the method was scaled up to the full \emph{Gaia} dataset, allowing for a systematic search across a much larger region of the sky. \citet{2021ApJ...914..123I} subsequently applied \texttt{streamfinder} to \emph{Gaia}~DR2 and EDR3, resulting in a major increase in the number of known stellar streams. This analysis improved the data quality for many previously known streams and led to several new detections, a selection of which is reproduced in Fig.~\ref{fig:ibata_2021_fig1}. They reported nine new streams, bringing the total number of known Galactic stellar streams to about sixty. 
    \begin{figure}
        \centering
        \includegraphics[width=\linewidth]{images/ibata_2021_fig1.jpg}
        \caption[Milky Way stellar streams discovered with \texttt{streamfinder} in \emph{Gaia} DR2]{Milky Way stellar streams discovered with \texttt{streamfinder} in \emph{Gaia}~DR2. Both plots are shown in Galactic coordinates. The top panel is color-coded by proper motion in longitude, while the bottom is color-coded by proper motion in latitude. This is Fig.~1 from \citet{2021ApJ...914..123I}.}
        \label{fig:ibata_2021_fig1}
    \end{figure}
    \citet{2021ApJ...914..123I} served as a primary motivator for our study. They showed that of the then sixty known streams only twenty are associated with globular cluster. This means that \textit{most} of the globular clusters do not have tidal streams despite the fact that it should be a normal phenomenon. As summarized by Fig.~\ref{2020A&A...637L...2P}, there are no clear correlations between orbital and structural parameters of globular clusters and the presence of tidal tails or not. This fact inspired us to simulate the expect mass loss distribution of the entire Mikly Way globular cluster catalog, which is presented in Chapter~4.
    
    Already with the improved stellar stream data, people have quickly begun trying to see at what point we can: 
    \begin{enumerate}
        \item the accretion history of the galaxy;
        \item the local gravitational field;
        \item the net smooth global gravitational field;
    \end{enumerate}
    
    \citet{2025NewAR.10001713B} published a review discussing the many aspects of Milky Way stellar streams since the publication of Gaia data. Usefully, they divide the description of stream morphology as being dependent into three factors: 
    \begin{enumerate}
        \item the internal dynamics of the progenitor system, which in turn determines the rate at which stars are ejected and with what energies;
        \item the global, smooth, and time-independent gravitational field. This determines the shape of the orbit and the tidal forces that occur along the orbit;
        \item the time-dependent dynamical processes. These include: the bar, dark matter subhalos, giant molecular clouds, spiral arms, etc.
    \end{enumerate}

    We refer the reader to \citet{2025NewAR.10001713B} and the reference therein. We discuss some of these aspects in the the introduction of Chapters~4 and Chapter~5. But I'm also gonna consider some more papers below just to give an idea of all the stuff that's out there. I want to discuss some more considerations too and draw more links between many papers\dots

    Stellar streams are proving useful for theories of globular cluster formation. As summarized in \citet{2025arXiv250116438K}, there are many proposed scenarios for globular cluster formation. One of which is that they are formed within their own dark-matter subhalos. Many people study if a globular cluster were embedded in a dark matter halo what the effects of the kinematics would be \citep{2022A&A...667A.112V}. However, as noted by \citet{2025arXiv250116438K}, this scenario is not favored for a few reasons. First, it would be able to predict the number of globular clusters within the Milky Way today. There's a couple other \citet{2016ApJ...823...52K} perfomed a study on the formation of globular clusters within dark matter sub-halos and found that the deeper potential well could leave to star formation with a larger spread in metallicities between the cluster's members than is observed in real globular clusters. Thanks to Gaia data, \citet{2022ApJ...941L..38M}, was able to study the morphology of a collection of Milky Way streams. \citet{2022ApJ...941L..38M} studied that the velocity dispersion of the Milky Way streams are inconsistent with model predictions for streams coming from clusters embedded in massive and \textit{cuspy} profiled dark-matter subhalos.

    Not only can the globular clusters be used to infer the accretion history of the Galaxy, but also the \textit{streams} \citep{2021ApJ...909L..26B}. For example, \citet{2020ApJ...898L..37Y} discovered a ``Low Mass Stream'' that was later characterized by \citet{2021ApJ...920...51M} to have a wide spread in metallicities and high velocity dispersion, making it more like to be a stream associated to a dwarf galaxy merger and not a globular cluster. 

    \citet{2019ApJ...880...38B} study a gap within the GD-1 stream and performed numerical simulations to describe the properties of a progenitor that could perturb it. This is an example of a work that is aiming to understand the granularity of the Milky Way's gravitational potential. \citet{2020ApJ...892L..37B} did a follow up to say that they could localize the potential dark matter sub halo. It's trajectory is consistent with the dark matter subhalos that could emerge from the sagittarious dwarf galaxy. So this is both formation and granularity.

    % add some paragraphs about the Milky Way's halo and how it's stellar content would come from the globular clusters. Also maybe something about the metal poor clusters. It can go earlier or here. Not sure how to organize the discorse. 

    \citet{2018ApJ...867..101B} did an information theory prospective on how much informaiton can be extract from each stellar streams when inferring the galactic potential of the Milky Way, noting that a single stream will not be able to simultaneously constrain all galactic parameters and some will be strongly detenerage with one another. However, they note that simultaneously fitting many streams will get the job done... This was done with \citet{2024ApJ...967...89I}. \citet{2024ApJ...967...89I} brought the total number of streams to a high number and also used them to come up with a gravitational potential model of the MW \dots 

    stellar stream astronomy is growing, will continue to grow, and has a lot of potential to answer many scientific questions for the formation history of the galaxy and to the behavior of dark matter\dots