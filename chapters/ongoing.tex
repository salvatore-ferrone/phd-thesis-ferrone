\section{Galactic bar}
    \subsection{Elliptical coordinates}
        \textbf{these are results of an unfinished attempt to understand the effect of the bar from a more theoretical point of view}
        \begin{itemize}
            \item Explain the elliptical coordinates 
            \item show the solutions that worked
            \item show the solutions that blew up 
            \item using eigenvalues of the jacobian for studying numerical stability 
            \item "They are a useless complication"
            \item Spherical harmonics 
        \end{itemize}

    \subsection{Numerical results}

\section{Dark matter subahlos}


\section{Mock observations}


\section{Globular cluster formation and multiple stellar populations}

    What if we wanted to go further? Indeed at somepoint, this is just considered other work or future considerations, or other research questions entirely. 

    What if we wanted to further? this is certainly outside the scope of our work, but we could consider the globular cluster formation. 

    The complexity doesn't end here. For instance, we are unsure how globular clusters form. There are some mechanisms proposed (cite the extra galactic GC review). That review summarizes them as: (1), (2), (3). However, to completely understand them would involve understanding the chemical enrichment history of the universe. Starting from big-bang nucleosysntehsis, to first generation stars (POP III), and how these first stars chemically enrich the environment, and how the clusters form from said environment. \citet{2022A&A...668A.191C} studied cluster formation in environments that were enriched by just pop II stars or a mixture of pop III/II stars and showed that pop III stars were necessary to enrich the environment enough to reproduce current cluster qualtities. 

    However, from my literature review there does seem to be a gap linking the chemical environment that could be enriched from a combination of pop II/popIII stars and how cluster formation. At least we know clusters are at least second generation (pop II) stars, since the first stars (pop III) form massive and die young \citep{2002ApJ...571...30S}.

    Ideally, we would like to know about the stars within the cluster to then infer the environment that it must have been born in. Extra-galactic census of globular clusters show that there isa bi-modality in color, making people believe that there are at least two generations or two different mechanisms for forming them. 

    \citet{2024A&A...681A..45L} Elena and Alessandra investigate a scenario that considers multiple populations in globular clusters where the second population is much less massive than the first, but evaporation and tidal stripping processes can mean that there present day contributions are similar to one another. 

    \citet{2025MNRAS.537.2342C} is looking at the dynamics of the different populations within 47 Tuc.

    \citet{2024MNRAS.529.2413U} said that there's evidence of multiple populations within the stellar streams. Something expected and now observationally confirmed 

    \citet{2023A&A...673A.152I} showed that within the last 5~gyr the potential was stable enough where the orbital solutions are pretty stable. 
    
    \citet{2006ARA&A..44..193B} made a review on extra galactic globular clusters. They discuss things like the observation evidence, the globular cluster mass function (all clusters), and find it to have an equal power spectrum, but the issue is that you don't know where to truncate the power law. maybe at lower end you can argue $10^4 M_\odot$ from two body relaxation and a tidal field. However, the upper limit is hard since you already don't expect that many, and the inclusion of a couple more high mass members can significantly alter the ``mean'' mass GC. They also note how a lot of cluster properties match the galactic properties. They talk about a bi-modal distribution. It really isn't that bimodal, there's a ton of overlap. There's a bluer one and a redder one. One of them might form during the collisions between galaxies while the other could be formed deeped in the potential wells. They also said that as of 2014 there is evidence that GCs are still forming. They also discuss the chemical abundance patterns within the clusters. It's complicated stuff, I wish I was better at stellar physics and knowing all the different sequences\dots
    
    \citet{2002ApJ...571...30S} proposed the of pair-unstable supernovae $\gamma + \gamma \rightarrow e^{+} + e^{-}$ to create the positrons and electrons, that reduces the radiation pressure, to then make the star implode then explode. This shows the expected yeild of some selected heavier elements as a function of the ZAMs. Pretty sick. 
    
    \citet{2006MNRAS.369..825S} proposed the initial mass function on pop III stars. I think this explains why the were expected to be soooo large and how the metals allow them to grow so much. I don't know what the  WMAP eletron scatter probe is. I think they're already talking about detectability. and saying that if a scenario produces many eave pop III stars, then it won't be detectable in the WMAP data. Perhaps pop III stars should do something to the electron scattering optical depth? What is a Larson initial mass function?  

    \textit{Try and make some notes on multiple stellar populations. Show some of Milone's \& Alessandra's work. Talk about some things in general how we don't even know the formation of globular clusters. Say that in general they are pop II stars that are the stars that formed right after first stars exploded, which is a very interesting research question. Do they exist? or are they all dead? Try and cite some of Rafaella's papers, tying together stellar evolution, globular cluster formation, multiple stellar populations, and primordial stars.}

    \textit{Anything else? Try to make a note on particle-spray methods. How those can be just send out a particles, while with accurate energies, with an escape rate that might not be realisitic due to the perscriptive nature.}

    \textit{perhaps it can be interesting to try and detect multiple population stuff within the stelalr streams\dots that could constrain the globular cluster evolution}





