\section{General intro}

How should I introduce this paper??

\section{\bf{Numerical method}}\label{methods}

    To model the formation and evolution of extra-tidal features around Galactic globular clusters, we use a set of codes, called Globular Clusters' Tidal Tails (GCsTT)  developed by our group. It comprises two python codes, for the backward and forward integration of a stellar system, made of N test-particles (see Sect.~\ref{numerical}). These codes are separated for data organization and management, while the (computationally) most expensive part, namely, the calculation of the accelerations acting on the N particles and the orbits integration, is realized by means of a Fortran module written by our group. This module is interfaced to python by means of f2py directives from NumPy. The use of test-particle methods for modeling the tidal stripping process is widespread in the literature, where these methods are  usually applied to one or few clusters at a time \citep[see, e.g., ][]{lane12, mastrobuono12, palau19, piatti21a, grillmair22}. In this work, we apply a test-particle methodology to the whole set (159) of Galactic globular clusters for which this is currently possible, also taking into account, for each cluster, the errors on astrometry, line-of-sight velocities\footnote{Note: the term ``line-of-sight velocities''  adopted in this paper corresponds to the term ``radial velocities'' often used in the literature, as well as in the Gaia catalogues. We prefer the use of the first term, since the second is usually used also to indicate the (Galactocentric) radial velocities and can introduce some ambiguity, especially when different coordinate systems are used. We emphasize that the choice to use the term ``line-of-sight velocity'' is not new \citep[see, e.g., ][]{vasiliev21}.} and distances. In the following, we describe the two main steps of the procedure used by GCsTT  to simulate the tidal stripping process (Sect.~\ref{numerical}), the initial conditions adopted for the clusters' parameters and their mass distribution (Sect.~\ref{initialconds}), as well as the Galactic potentials (Sect.~\ref{galmod}).
    \subsection{Simulations of the tidal stripping process:  Two-step procedure}\label{numerical}
        To model the formation and evolution of extra-tidal features around Galactic globular clusters, and predict their current properties, we proceed as follows: 

        \textit{Step i: Backward integration. Reconstructing the globular cluster orbit over the last 5~Gyr}: First, for each Galactic globular cluster for which the distances from the Sun, proper motions, line-of-sight velocities, and structural parameters are  available (see Sect.~\ref{initialconds}), we determine their current positions and velocities in a Galactocentric reference frame, in which the Sun is at $(x_\odot, y_\odot, z_\odot) = (-8.34, 0., 0.027)$~kpc \citep{chen01, reid14} and at a given velocity for the local standard of rest, $v_{LSR}= 240$~km/s \citep{reid14}, and a peculiar velocity of the Sun with respect to the LSR, $(U_{\odot}, V_{\odot}, W_{\odot})  = (11.1, 12.24, 7.25)$~km/s \citep{schonrich10}. We then integrate the orbit of a single point mass, representing the cluster barycenter, backwards in time for 5~Gyr, and in this way, we retrieve its position and velocity at that time in the chosen Galactic potential (see Sect.~\ref{galmod}). We notice that other choices for the Sun's position or velocity with respect to the Galactocentric frame would have been possible. For example, \citet{piatti21a} adopted the same values as ours for the $v_{LSR}$ and for the peculiar velocity of the Sun but with a different distance to the Galactic center \citep[8.1~kpc in their work, see][]{gravity18}. The difference in the adopted position of the Sun is, however, generally smaller than the uncertainties affecting our knowledge of the distance of Galactic globular clusters to the Sun. For this reason, we do not to explore the dependency of the results presented in this paper with regard to these choices. 

        \textit{Step ii: Forward integration. Test-particle streams from the past to the present day}: Once the positions and velocities of the barycenter of each cluster, 5~Gyr ago, have been determined, we build  the corresponding $N$-body system, with N = 100 000 particles.  The phase-space coordinates of these particles are generated following a Plummer distribution, with the total mass and half-mass radius as described in Sect.~\ref{initialconds}. The barycenter of this $N$-body cluster is then assigned initial positions and velocities in the Galactic model, as those retrieved at step $(i),$ and the cluster is then integrated forward in time until the present day. Particles in this $N$-body system are modeled as test-particles, that is, they experience the gravitational field exerted by the globular cluster itself (see Sect.~\ref{initialconds}) and by the Galaxy (see Sect.~\ref{galmod}), but do not generate any gravitational field themselves. This allows us to maintain a computational time which scales as $O(N)$ and not as $O(N^2)$, as would be the case for a direct $N$-body self-consistent computation.

        In the following, we refer to these simulations, made by using the most probable values on distances, proper motions, and line-of-sight velocities, as the ``reference simulations.'' In addition, for each globular cluster, we also take into account the errors on its distance, proper motions, and line-of-sight velocity,  assuming Gaussian distributions of the errors, treated independently, and by generating 50 random realizations of these parameters.  For each of these realizations, we repeat the steps  described above, that is: (\textit{step i}) we determine the associated current positions and velocities in the chosen Galactocentric reference frame, we integrate the orbit of the single-point mass (representing the cluster barycenter) backwards in time, retrieving the corresponding values 5~Gyr ago, (\textit{step ii}) we build an $N$-body cluster containing $N$= 100~000 particles, with total mass and half-mass radius as those used for the reference simulation, and then we integrate the $N$-body cluster forwards in time until the present-day position. 

        To summarize, for a given Galactic potential, we run $159\times (50+1)=8109$ simulations, where 159 is the total number of clusters for which we currently have both 6D phase-space information and structural parameters. As we discuss in the following section, the whole set of globular clusters has been evolved in three different Galactic potentials, which implies that a total of 24 327 simulations have been run.

        For the orbit integration, a leap-frog algorithm is used, with a fixed time-step, $\Delta t$, and a total number of steps, $N_{steps}$, such that the total simulated time is  $\Delta t \times N_{steps}=5$~Gyr. The choice of the value of $\Delta t$ adopted to simulate each cluster in the Galactic potential has been based on the energy conservation of the corresponding cluster evolved in isolation (i.e., without the effect of the Galactic gravitational field for 5~Gyr). For the majority of the clusters (109/159), this value was set to $\Delta t = 10^5$~yr (for a corresponding value of $N_{steps}=50\,000$), while for the remaining clusters (50/159) a  $\Delta t = 10^4$~yr (for a corresponding value of $N_{steps}=500\,000$) was used. We refer to Appendix~\ref{deltat} (and in particular to Table~\ref{tcross-energy}) for additional details on the choice of $\Delta t$ for the whole set of clusters. As for the total simulated time, while globular clusters are much older than 5~Gyr, we chose this time limit because the longer back in time we could go, the less certain we would be of the Galactic environment. In addition, the last significant mergers in the Galaxy happened between 9 and 11~Gyr ago  \citep[see][]{belokurov18, helmi18, dimatteo19, gallart19, kruijssen20} -- well before the time interval simulated in this study. Other more recent interactions, such as the accretion of Sagittarius and of the Magellanic Clouds, may perturb the Galactic potential as well \citep[see, e.g.,][]{vasiliev21b} and we plan to investigate their impact on the properties of globular cluster streams in the future.

        For each realization, we generate an output file in an hdf5 format\footnote{\url{https://www.hdfgroup.org/solutions/hdf5/}} containing the values for the right ascension ($\alpha$), declination ($\delta$), distance from the Sun ($D$), along with the components for proper motion in the equatorial coordinate system ($\rm \mu_{\alpha}\cos(\delta)$ and $\rm \mu_\delta$), the line-of-sight velocity ($\rm v_{\ell os}$), longitude ($\ell$), latitude ($ b$), as well as the components for proper motion in the Galactic coordinate system ($\rm \mu_{\ell}\cos(\mathit{b})$ and $\rm \mu_b$) and the Galactocentric positions ($x, y, z$), velocities ($v_x, v_y, v_z$) and energy, $E$, of each particle in the simulated system. We used Astropy \citep{astropy13, astropy18} to convert the Galactocentric positions and velocities in the equatorial and Galactic quantities $\alpha, \delta, D, \rm \mu_{\alpha}\cos(\delta), \rm \mu_\delta, \rm v_{\ell os}, \ell, b, \rm \mu_{\ell}\cos(\mathit{b})$, and $\rm \mu_b$.

        For each particle, we also save its escape time $t_{\rm esc}$,  defined as the time at which the particle escapes from the cluster, that is, the time, $t,$ at which the particle satisfies the relation\footnote{If the particle is gravitationally bound to the cluster until the end of the simulation, $t_{\rm esc}$ is set equal to $-9999$.}:
        \begin{equation}
            E_{GC}= 0.5 \times \left( (v_x-v_{x, GC})^2+(v_y-v_{y,GC})^2+(v_z-v_{z,GC})^2\right)+\Phi_{GC} > 0,
        \end{equation}
        with $E_{GC}$ being the total specific energy of the particle relative to the cluster, that is, the sum of the potential energy, $\Phi_{GC}$, due to the gravitational field of the cluster (see Eq.~\ref{gcpot}), and of the kinetic energy, relative to the cluster barycenter, $T_{GC}=0.5 \times \left( \left(v_x-v_{x, GC}\right)^2+\left(vy-v_{y,GC}\right)^2+\left(vz-v_{z,GC}\right)^2\right)$, where $v_x, v_y$, and $v_z$ are its velocity components at time, $t$, and $v_{x,GC}, v_{y,GC}$, and $v_{z,GC}$  of the cluster barycenter at the same time. A positive value of $E_{GC}$ implies that the particle is no longer gravitationally bound to the cluster and, hence, it is lost in the field. Overall, the total volume of the whole set of  24 327 simulations, saved in hdf5 format, amounts to about 370 Gb.
