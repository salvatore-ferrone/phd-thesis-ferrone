\section{General intro}

How should I introduce this paper??

\section{\bf{Numerical method}}\label{methods}

    To model the formation and evolution of extra-tidal features around Galactic globular clusters, we use a set of codes, called Globular Clusters' Tidal Tails (GCsTT)  developed by our group. It comprises two python codes, for the backward and forward integration of a stellar system, made of N test-particles (see Sect.~\ref{numerical}). These codes are separated for data organization and management, while the (computationally) most expensive part, namely, the calculation of the accelerations acting on the N particles and the orbits integration, is realized by means of a Fortran module written by our group. This module is interfaced to python by means of f2py directives from NumPy. The use of test-particle methods for modeling the tidal stripping process is widespread in the literature, where these methods are  usually applied to one or few clusters at a time \citep[see, e.g., ][]{lane12, mastrobuono12, palau19, piatti21a, grillmair22}. In this work, we apply a test-particle methodology to the whole set (159) of Galactic globular clusters for which this is currently possible, also taking into account, for each cluster, the errors on astrometry, line-of-sight velocities\footnote{Note: the term ``line-of-sight velocities''  adopted in this paper corresponds to the term ``radial velocities'' often used in the literature, as well as in the Gaia catalogues. We prefer the use of the first term, since the second is usually used also to indicate the (Galactocentric) radial velocities and can introduce some ambiguity, especially when different coordinate systems are used. We emphasize that the choice to use the term ``line-of-sight velocity'' is not new \citep[see, e.g., ][]{vasiliev21}.} and distances. In the following, we describe the two main steps of the procedure used by GCsTT  to simulate the tidal stripping process (Sect.~\ref{numerical}), the initial conditions adopted for the clusters' parameters and their mass distribution (Sect.~\ref{initialconds}), as well as the Galactic potentials (Sect.~\ref{galmod}).
    \subsection{Simulations of the tidal stripping process:  Two-step procedure}\label{numerical}
        To model the formation and evolution of extra-tidal features around Galactic globular clusters, and predict their current properties, we proceed as follows: 

        \textit{Step i: Backward integration. Reconstructing the globular cluster orbit over the last 5~Gyr}: First, for each Galactic globular cluster for which the distances from the Sun, proper motions, line-of-sight velocities, and structural parameters are  available (see Sect.~\ref{initialconds}), we determine their current positions and velocities in a Galactocentric reference frame, in which the Sun is at $(x_\odot, y_\odot, z_\odot) = (-8.34, 0., 0.027)$~kpc \citep{chen01, reid14} and at a given velocity for the local standard of rest, $v_{LSR}= 240$~km/s \citep{reid14}, and a peculiar velocity of the Sun with respect to the LSR, $(U_{\odot}, V_{\odot}, W_{\odot})  = (11.1, 12.24, 7.25)$~km/s \citep{schonrich10}. We then integrate the orbit of a single point mass, representing the cluster barycenter, backwards in time for 5~Gyr, and in this way, we retrieve its position and velocity at that time in the chosen Galactic potential (see Sect.~\ref{galmod}). We notice that other choices for the Sun's position or velocity with respect to the Galactocentric frame would have been possible. For example, \citet{piatti21a} adopted the same values as ours for the $v_{LSR}$ and for the peculiar velocity of the Sun but with a different distance to the Galactic center \citep[8.1~kpc in their work, see][]{gravity18}. The difference in the adopted position of the Sun is, however, generally smaller than the uncertainties affecting our knowledge of the distance of Galactic globular clusters to the Sun. For this reason, we do not to explore the dependency of the results presented in this paper with regard to these choices. 

        \textit{Step ii: Forward integration. Test-particle streams from the past to the present day}: Once the positions and velocities of the barycenter of each cluster, 5~Gyr ago, have been determined, we build  the corresponding $N$-body system, with N = 100 000 particles.  The phase-space coordinates of these particles are generated following a Plummer distribution, with the total mass and half-mass radius as described in Sect.~\ref{initialconds}. The barycenter of this $N$-body cluster is then assigned initial positions and velocities in the Galactic model, as those retrieved at step $(i),$ and the cluster is then integrated forward in time until the present day. Particles in this $N$-body system are modeled as test-particles, that is, they experience the gravitational field exerted by the globular cluster itself (see Sect.~\ref{initialconds}) and by the Galaxy (see Sect.~\ref{galmod}), but do not generate any gravitational field themselves. This allows us to maintain a computational time which scales as $O(N)$ and not as $O(N^2)$, as would be the case for a direct $N$-body self-consistent computation.

        In the following, we refer to these simulations, made by using the most probable values on distances, proper motions, and line-of-sight velocities, as the ``reference simulations.'' In addition, for each globular cluster, we also take into account the errors on its distance, proper motions, and line-of-sight velocity,  assuming Gaussian distributions of the errors, treated independently, and by generating 50 random realizations of these parameters.  For each of these realizations, we repeat the steps  described above, that is: (\textit{step i}) we determine the associated current positions and velocities in the chosen Galactocentric reference frame, we integrate the orbit of the single-point mass (representing the cluster barycenter) backwards in time, retrieving the corresponding values 5~Gyr ago, (\textit{step ii}) we build an $N$-body cluster containing $N$= 100~000 particles, with total mass and half-mass radius as those used for the reference simulation, and then we integrate the $N$-body cluster forwards in time until the present-day position. 

        To summarize, for a given Galactic potential, we run $159\times (50+1)=8109$ simulations, where 159 is the total number of clusters for which we currently have both 6D phase-space information and structural parameters. As we discuss in the following section, the whole set of globular clusters has been evolved in three different Galactic potentials, which implies that a total of 24 327 simulations have been run.

        For the orbit integration, a leap-frog algorithm is used, with a fixed time-step, $\Delta t$, and a total number of steps, $N_{steps}$, such that the total simulated time is  $\Delta t \times N_{steps}=5$~Gyr. The choice of the value of $\Delta t$ adopted to simulate each cluster in the Galactic potential has been based on the energy conservation of the corresponding cluster evolved in isolation (i.e., without the effect of the Galactic gravitational field for 5~Gyr). For the majority of the clusters (109/159), this value was set to $\Delta t = 10^5$~yr (for a corresponding value of $N_{steps}=50\,000$), while for the remaining clusters (50/159) a  $\Delta t = 10^4$~yr (for a corresponding value of $N_{steps}=500\,000$) was used. We refer to Appendix~\ref{deltat} (and in particular to Table~\ref{tcross-energy}) for additional details on the choice of $\Delta t$ for the whole set of clusters. As for the total simulated time, while globular clusters are much older than 5~Gyr, we chose this time limit because the longer back in time we could go, the less certain we would be of the Galactic environment. In addition, the last significant mergers in the Galaxy happened between 9 and 11~Gyr ago  \citep[see][]{belokurov18, helmi18, dimatteo19, gallart19, kruijssen20} -- well before the time interval simulated in this study. Other more recent interactions, such as the accretion of Sagittarius and of the Magellanic Clouds, may perturb the Galactic potential as well \citep[see, e.g.,][]{vasiliev21b} and we plan to investigate their impact on the properties of globular cluster streams in the future.

        For each realization, we generate an output file in an hdf5 format\footnote{\url{https://www.hdfgroup.org/solutions/hdf5/}} containing the values for the right ascension ($\alpha$), declination ($\delta$), distance from the Sun ($D$), along with the components for proper motion in the equatorial coordinate system ($\rm \mu_{\alpha}\cos(\delta)$ and $\rm \mu_\delta$), the line-of-sight velocity ($\rm v_{\ell os}$), longitude ($\ell$), latitude ($ b$), as well as the components for proper motion in the Galactic coordinate system ($\rm \mu_{\ell}\cos(\mathit{b})$ and $\rm \mu_b$) and the Galactocentric positions ($x, y, z$), velocities ($v_x, v_y, v_z$) and energy, $E$, of each particle in the simulated system. We used Astropy \citep{astropy13, astropy18} to convert the Galactocentric positions and velocities in the equatorial and Galactic quantities $\alpha, \delta, D, \rm \mu_{\alpha}\cos(\delta), \rm \mu_\delta, \rm v_{\ell os}, \ell, b, \rm \mu_{\ell}\cos(\mathit{b})$, and $\rm \mu_b$.

        For each particle, we also save its escape time $t_{\rm esc}$,  defined as the time at which the particle escapes from the cluster, that is, the time, $t,$ at which the particle satisfies the relation\footnote{If the particle is gravitationally bound to the cluster until the end of the simulation, $t_{\rm esc}$ is set equal to $-9999$.}:
        \begin{equation}
            E_{GC}= 0.5 \times \left( (v_x-v_{x, GC})^2+(v_y-v_{y,GC})^2+(v_z-v_{z,GC})^2\right)+\Phi_{GC} > 0,
        \end{equation}
        with $E_{GC}$ being the total specific energy of the particle relative to the cluster, that is, the sum of the potential energy, $\Phi_{GC}$, due to the gravitational field of the cluster (see Eq.~\ref{gcpot}), and of the kinetic energy, relative to the cluster barycenter, $T_{GC}=0.5 \times \left( \left(v_x-v_{x, GC}\right)^2+\left(vy-v_{y,GC}\right)^2+\left(vz-v_{z,GC}\right)^2\right)$, where $v_x, v_y$, and $v_z$ are its velocity components at time, $t$, and $v_{x,GC}, v_{y,GC}$, and $v_{z,GC}$  of the cluster barycenter at the same time. A positive value of $E_{GC}$ implies that the particle is no longer gravitationally bound to the cluster and, hence, it is lost in the field. Overall, the total volume of the whole set of  24 327 simulations, saved in hdf5 format, amounts to about 370 Gb.

        \subsection{Simulations of the tidal stripping process: Globular clusters' current and initial conditions and their gravitational potential}\label{initialconds}

        Steps $(i)$ and $(ii)$ described in the previous section require some input conditions to be adequately executed. The current distances from the Sun, proper motions, and line-of-sight velocities, as well as the related uncertainties, of all 159 globular clusters considered in this study are taken respectively from \citet{baumgardt21} and \citet{vasiliev21}. These values are then converted into Galactocentric positions and velocities by making use of Astropy and used as initial conditions to execute step $(i)$. 

        Step $(ii)$ requires generating an $N$-body system, representing the globular cluster, whose initial total mass and half-mass radius are assigned on the basis of their current values, as given by \citet{baumgardt18}\footnote{In particular, the adopted  values have been taken from the edition available at \url{https://people.smp.uq.edu.au/HolgerBaumgardt/globular/parameter.html}, up to January 14, 2022.} and reported in Table~\ref{TableIC}. As anticipated at step $(ii)$ in Sect.~\ref{numerical}, the phase-space coordinates of each $N$-body cluster are generated by assuming  a Plummer distribution of total mass, $M_{GC}$, and half-mass radius, $r_{h}$, for which the corresponding potential is:
        \begin{equation}\label{gcpot}
            \Phi_{GC}(r) = -\frac{GM_{GC}}{\sqrt{r^2+{r_c}^2}},
            \end{equation}
        where $r_c$ is the cluster scale radius and it is related to the half-mass radius, $r_{h}$, through $r_{h} \simeq 1.305 r_c$ \citep{heggie03}. The variable $r$ here indicates the distance of the test particle from the center of the cluster. For each cluster, the same Plummer distribution used to generate the $N$-body system is also used to calculate the accelerations exerted on each particle as the system moves through time. The Plummer sphere, representing the cluster potential, indeed moves through the Galaxy along the orbit retrieved at step (i), traveling this time in the opposite direction, from 5 Gyr ago to the present day.

        It might be noted that this implies that the globular cluster density profile and its internal parameters (total mass and characteristic radius) are constant over time in these models. This is, of course, a crude approximation, because in reality both the internal parameters and the density profile itself can change over time. We consider these assumptions to be acceptable within the scope of our work given that we are primarily interested in the distribution of extra-tidal stars, which had once escaped from the cluster have dynamics primarily dictated by the Galactic potential rather than the globular cluster itself. Of course,  the density of stars along the extra-tidal structures, as well as the total mass lost, depend on these assumptions. That is to say that if the mass of the cluster was not assumed constant over time, but could possibly decrease, the gravitational attraction exerted by the cluster itself on its stars would be weaker and this would lead to an increasing mass loss and density along the tails. We could have proceeded with diminishing the mass over time, based on some assumptions on the temporal behavior of this relation, however, we did not find this approach satisfying. In this way. we would have taken into account a temporal evolution of the mass, but not of the size of the cluster, adding a supplementary hypothesis to the problem. For these reasons, we decided to maintain the simplest approach. We emphasize that other groups have followed the same methodology, maintaining masses and sizes that remain constant over time \citep[see, e.g.,][]{palau19}.

        The summary tables giving both the current internal parameters of the clusters (total mass and half-mass radius), their astrometric quantities of relevance for this study and the line-of-sight velocities are publicly available\footnote{All data can be found here \url{https://people.smp.uq.edu.au/HolgerBaumgardt/globular}.}. We have made use of these tables for our work and we report them in a unique table in our paper for the sake of the completeness and self-consistency of the data used (see Table~\ref{TableIC}). 

    \subsection{Simulations of the tidal stripping process: Galactic potentials}\label{galmod}
    
    \begin{table*}
        \centering
        \caption{Parameters of the Galactic mass models adopted in this work. Masses are in units of $2.32\times10^7M_{\odot}$, distances given in units of kpc.
        \label{PII}}
        \tiny
        \begin{tabular}{  l c  c  c  c  c  c  c  c  c  c  c  c  c   c } \hline
        Parameters & $M_{bulge}$ &  $M_{bar}$ & $M_{thin}$ &  $M_{thick}$  & $M_{halo}$ & \ $b_{bulge}$  & $a_{bar}$ & $b_{bar}$ & $c_{bar}$ & $a_{thin}$ & $b_{thin}$  & $a_{thick}$ &  $b_{thick}$ & $a_{halo}$ \\  \hline \hline \\
        PI & 460.0 & 0.0 & 1700.0 & 1700.0  & 6000.0  & 0.3 & -- & -- & -- & 5.3000 & 0.25 & 2.6 & 0.8 & 14.0 \\  \hline    
            PII & 0.0 & 0.0 & 1600.0 & 1700.0  & 9000.0  & -- & -- & -- & -- & 4.8000 & 0.25 & 2.0 &  0.8 & 14.0 \\    \hline    
            PII-0.3-SLOW & 0 & 990.0 & 1120.0 & 1190.0  & 9000.0  & -- & 4.0 & 1. & 0.5 & 4.8000 &0.25 & 2.0 &  0.8 & 14.0 \\ \hline 
        \end{tabular} 
        \normalsize
    \end{table*}    

    As for the Galactic mass distribution, we make use of the two axisymmetric Galactic mass models presented in \citet{pouliasis17} and of an asymmetric mass model, containing a central stellar bar, and we present it here for the first time. We recall  the main properties of the two models of  \citet{pouliasis17} below and we describe the asymmetric Galactic mass model, presented here for the first time, in more detail.

    \subsubsection{Model I by \citet{pouliasis17}: An axisymmetric mass model for the Galaxy including a spherical bulge}
        Model I by \citet{pouliasis17} (abbreviated name: PI) consists of four components: two disks (thin and thick), both described by Miyamoto \& Nagai potentials, a dark matter halo, and a central bulge. Its total potential is:
        \begin{equation}
            \Phi_{tot}(R, z) = \Phi_{thin}(R, z) + \Phi_{thick}(R, z) + \Phi_{halo}(r)+  \Phi_{bulge}(r),
        \end{equation}
        with $r=\sqrt{R^2 + z^2}$,
        \begin{eqnarray}
            \Phi_{thin}(R,z)&=&\frac{-GM_{thin}}{\left(R^2+\left[a_{thin}+\sqrt{z^2+b_{thin}^2}\right]^2\right)^{1/2}},\\
            \Phi_{thick}(R,z)&=&\frac{-GM_{thick}}{\left(R^2+\left[a_{thick}+\sqrt{z^2+b_{thick}^2}\right]^2\right)^{1/2}},
        \end{eqnarray}
        \begin{equation}
            \begin{split}
                \Phi_{halo}(r)=&\frac{-GM_{halo}}{r}-\frac{M_{halo}}{1.02a_{halo}}\times\\
                & \Bigg[\frac{-1.02}{1+\left(\frac{r}{a_{halo}}\right)^{1.02}}+ln{(1+\left(\frac{r}{a_{halo}}\right)^{1.02})}\Bigg]_R^{100},
            \end{split}
        \end{equation}
        and 
        \begin{equation}
            \Phi_{bulge}(r) = -\frac{GM_{bulge}}{\sqrt{r^2+{b_{bulge}^2}}},
        \end{equation}
        where $M_{thin}, M_{thick}$, $M_{halo}$, and $M_{bulge}$ are the masses of the disks, halo, and bulge. Also, $a_{thin}, b_{thin},  a_{thick}, b_{thick}, a_{halo} ,b_{bulge}$ are the characteristic scale lengths of the thin and thick disks, the halo, and the central bulge, respectively (see Table~\ref{PII}).

        This model is a modification of the classical \citet{allen91} model, made to include also the presence of a thick disk. As it has been discussed in detail by \citet{pouliasis17}, the choice to include  a massive spheroid in this model, as well as in the original \citet{allen91} model, is dictated by the need to reproduce CO/HI-based velocity curves, as those provided by \citet{sofue12}, which show a rise and then a sudden decrease of the velocity curve in the inner Galactic regions ($R \le 2-3$~kpc). In an axisymmetric model, such a rise can be reproduced only if a central spheroidal component, with a typical mass greater than 10\% of that of the disk(s), is added. However, as shown by \citet{chemin15}, the central rise observed in the rotation of the molecular gas in the inner Galaxy may be an effect of non-circular motions generated by large-scale asymmetries such as the bar. Moreover, this feature is not reported in all the observational studies \citep[see, e.g., ][on which model PII is based]{reid14}. In other words, if we do not assume that the mass distribution of the inner Galaxy is axisymmetric, the need for a massive spheroidal component to reproduce velocity curves, such as those from \citet{sofue12}, no longer persists. In addition to that, in the last decade, a number of works  have shown that if a spheroidal bulge exists in the central regions of our Galaxy, it has to be small \citep[few percents of the mass of the disk at the most, see among others][]{shen10, kunder12, dimatteo15, gomez18}. All these arguments suggest to employ this model \citep[as well as all models including a massive central spheroid; see, e.g.,][]{irrgang13} with care when dealing with the central parts of the Galaxy. Since models with a massive central spheroid, however, are still used in the literature, we have included model PI here, as a term of comparison.    



    \subsubsection{Model II by \citet{pouliasis17}: An axisymmetric, bulge-less mass model for the Galaxy}

        Model II by \citet{pouliasis17} (abbreviated name: PII) consists of a spherical dark matter halo, with the same functional form adopted in the \citet{allen91} model, and of two disk components (a thin and a thick disk), with same functional form as PI. This model does not include any central spheroid (i.e., it is a bulge-less model) and thus its total potential is the sum of three components only:
        \begin{equation}
            \Phi_{tot}(R, z) = \Phi_{thin}(R, z) + \Phi_{thick}(R, z) + \Phi_{halo}(r),
        \end{equation}
        with the thin, thick disks, and dark matter halo having the same functional forms adopted in PI.

        As it has been shown in \citet{pouliasis17}, this model satisfies a number of observational constraints, such as the stellar density at the solar vicinity, thin- and thick-disk scale lengths and heights, the rotation curve as provided by \citet{reid14}  and the absolute value of the perpendicular force, $K_z$, as a function of distance to the Galactic centre \citep[see Sect.~2.5 in][]{pouliasis17}. As it is, however, an axisymmetric model, it fails to  accurately describe the inner few kpc of the Galaxy, where the stellar mass distribution has been shown to be asymmetric.     

    \subsubsection{Model~II with a massive, slowly rotating stellar bar}
        The third mass model (abbreviated name: PII-0.3-SLOW)  that we use in this paper is a version of PII by \citet{pouliasis17} modified to include a rotating stellar bar, whose mass has been assigned to be 30\% of the (thin+thick) disk mass of PII. We assume that the bar rotates with a constant pattern speed of $\Omega_{bar}=38\rm km~s^{-1}kpc^{-1}$ and that it is currently inclined of $25^\circ$ with respect to the Sun-Galactic center direction \citep[see][]{blandhawthorn16}. We model it as a triaxial distribution, whose gravitational potential is given by  \citet{long92}:
        \begin{equation}
            \Phi_{bar}(x,y,z)=\frac{GM_{bar}}{2a_{bar}}ln\left( \frac{x-a_{bar}+T_{-}}{x+a_{bar}+T_{+}}   \right),
        \end{equation}
        with $T_{\pm}=\left[ (a_{bar}\pm x)^2+ y^2 +  (b_{bar}+ \sqrt{c_{bar}^2+z^2})^2 \right]^{1/2}$
        and $a_{bar}, b_{bar}, c_{bar}$ the characteristic bar parameters. The total gravitational potential generated by this model thus takes the form:
        \begin{equation}
            \Phi_{tot}(x, y, z) = \Phi_{thin}(R, z) + \Phi_{thick}(R, z) + \Phi_{halo}(r)+  \Phi_{bar}(x, y, z).
        \end{equation}
        with all characteristic values given in Table~\ref{PII}. Practically, to include the bar, we reduced the mass of the disks in such a way to maintain the total stellar mass of this model as that of PII.  \citet{long92} provide the formulas of the accelerations generated by this triaxial distribution in the reference frame of the bar. To calculate and add them to the accelerations generated by the disks and dark matter halo, at each time step, we converted the positions of all particles  in the rotating, non-inertial reference frame of the bar, computed the corresponding accelerations on each particle, and then transformed these accelerations back in the inertial reference frame described in Sect.~\ref{numerical}. In this way, the accelerations due to the bar are added to those generated by the other terms of the Galactic mass distribution. 

        We emphasize that we do not consider this model as the best possible representation of the Galactic mass distribution, especially in the central region. It can, however, provide a first indication on how the inclusion of a rotating asymmetric component in the inner Galaxy can affect the globular cluster streams, near and far from the Galactic center. 

        Moreover, since the exact characteristics of the Milky Way bar are still subject to debate \citep[see, e.g.,][]{blandhawthorn16}, it is important to explore how varying the parameters adopted in this paper, such as the pattern speed, the mass, or the length of the bar, can affect the characteristics of the whole set of streams.  More complex shapes for the bar can also be explored, for example, by substituting the inner parts of the triaxial bar with a boxy-peanut-shaped morphology, which has been shown to characterize the inner Milky Way  \citep[see, e.g., ][]{wegg13, wegg15}. These topics are, however, beyond the scope of the present paper. In sum, given the uncertainties on the bar's physical extent and how it can change over the time span investigated here, its affect on the streams presented here are purely indicative. 


\section{Results}\label{results}
    \twocolumn
    \begin{figure*}[h!]
        \begin{center}
            \includegraphics[clip=true, trim = 0mm 20mm 0mm 20mm, width=0.9\columnwidth]{images/PI_ensemble_LB_count_density.pdf}
            \includegraphics[clip=true, trim = 0mm 20mm 0mm 20mm, width=0.9\columnwidth]{images/PI_ensemble_LB_density_mass.pdf}
            
            \includegraphics[clip=true, trim = 0mm 20mm 0mm 20mm, width=0.9\columnwidth]{images/PII_ensemble_LB_count_density.pdf}
            \includegraphics[clip=true, trim = 0mm 20mm 0mm 20mm, width=0.9\columnwidth]{images/PII_ensemble_LB_density_mass.pdf}

            \includegraphics[clip=true, trim = 0mm 20mm 0mm 20mm, width=0.9\columnwidth]{images/PII_0.3_SLOW_ensemble_LB_count_density.pdf}
            \includegraphics[clip=true, trim = 0mm 20mm 0mm 20mm, width=0.9\columnwidth]{images/PII_0.3_SLOW_ensemble_LB_density_mass.pdf}

            \caption{The stellar surface density of the \textit{ensemble} of extra-tidal features around the entire population of Galactic globular clusters at the current time, as predicted by our models. \textit{Left:} Surface number density. \textit{Right:} Surface mass density. Top row corresponds to model PI, the middle to model PII, and the bottom column to model PII-0.3-SLOW, as indicated. All densities are expressed in a logarithmic scale. The red point-like density maxima correspond to the current positions of the globular clusters. Values of higher density are overplotted. Thus in the case of mass density, diffuse tidal debris of more massive globular clusters covers the entire $(\ell,b)$ space and occults delicate tidal features, which are more visible when considering number density counts. In all panels, only the reference simulations are shown for clarity.} \label{galleryLB}
        \end{center}
    \end{figure*}
    \onecolumn

    \paragraph{A few basic premises:}
    Given the large number of simulations carried out and the wealth of information contained in them, it is not possible to exhaust all possible applications of this simulation database in this paper. We have therefore chosen to proceed as follows. 

    In Section~\ref{results1}, we present an overview of the distribution of all streams in  Galactic coordinates. This coordinate space will be the one used in the remainder of the entire article. This first section allows us to show qualitatively how the global distribution of streams varies, depending on the Galactic potential used.

    We  then move on (Section~\ref{streamvsD}) to present the global system of streams as a function of their distance from the Sun. In this section, we also show the kinematic properties of the streams, such as proper motions and line-of-sight velocities, that can be directly compared to Gaia data or other astrometric and spectroscopic surveys. This section also allows us to show the variety of morphologies that the stars which escaped from globular clusters can take. In Section~\ref{sec:morphologies}, we explore this issue in more detail, showing how these morphologies depend primarily on the orbital characteristics of the clusters and their distance from the Galactic center. For the most interesting cases, we compare the tidal structures predicted by our simulations with streams found in observational data. For this purpose, we make use of the \emph{galstreams} library of stellar streams in the Milky Way \citep{mateu22}, which constitutes a unique and public database summarizing angular positions, distances, proper motions, and line-of-sight velocity tracks for nearly a hundred Galactic stellar streams. Any stream that is not included in this library is not be compared to our simulations in the context of this paper.

    We note that the tidal features associated with each of the 159 simulated clusters are presented in Appendix~\ref{allstreams}. To avoid making this appendix too long, the  tidal features above are presented only in the case of the potential PII. However, all the data will be made available to the community at a dedicated site\footnote{\url{http://etidal-project.obspm.fr/}} where it is possible to  the way the characteristics of these streams change, for any cluster, in the three chosen potentials. 

    \subsection{A sky full of streams}\label{results1}

    Figure~\ref{galleryLB} shows the number and mass density distributions of the whole set of simulated globular clusters and their extra-tidal features in Galactic longitude and latitude for the three Galactic potentials.

    For all Galactic mass models adopted, a striking characteristic among the plots in Figs.~\ref{galleryLB} is the variety of features that our models predict, which are reminiscent of the tidal tails, stellar streams, and shells that are produced by interacting and merging galaxies in the process of mass assembly \citep[see, e.g., ][]{mancillas19}. Some clusters have very thin and elongated streams, which describe arcs that can extend up to $180^\circ$ in longitude, or tens of kpc in physical space. In some other cases, extra-tidal features appear shorter (a few up to ten degrees) and sometimes also thicker (about $10^\circ$ in the sky) than others. Finally, in some cases, clusters are surrounded by extended structures, such as halos, rather than coherent and thin streams. 

    This variety of properties depends on several factors: the distance of the stream to the Sun (due to projection; i.e., for a given physical thickness, the closer the stream is to the Sun, the more extended it appears in the $(\ell, b)$ plane), its orbital phase (towards the peri-center or the apo-center of the orbit), and the orbital properties of the parent globular cluster. We also see from these figures that stellar particles stripped from their parent clusters do not only redistribute in coherent structures, but in some cases, they can also contribute to a more diffuse density distribution. 

    Because of the large number of simulated clusters, we have chosen not to present the corresponding extra-tidal features one-by-one in the main part of this paper, but we have rather decided to describe these extra-tidal features with a global approach, by first adopting a criterion based on the distance of these features to the Sun (see Sect.~\ref{streamvsD}) and then discussing the types of distributions tidal debris can have and how these depend on the cluster orbital parameters (see Sect.~\ref{sec:morphologies}). All the extra-tidal structures generated by the 159 globular clusters simulated in this paper, and their corresponding uncertainties, are reported in Appendix~\ref{allstreams}. Among them, we include clusters with thin and elongated tails, such as IC~4499, NGC~3201, NGC~4590, NGC~5024, NGC~5053, Pal~5, to cite a few, as well as clusters such as AM~1, Pal~14, Pal~4, and Pal~15, whose extra-tidal material shows a halo-like configuration, and clusters such as NGC~1261, NGC~4147, NGC~6356, and UKS~1, whose stripped stars show a remarkable diffuse distribution in the field. 

    Finally, even when our models are not tailored to accurately reproduce the mass loss from globular clusters, since (as described in Sect.~\ref{initialconds}) we have adopted a test-particle approach with a time-constant globular cluster potential, it is however tempting to estimate, to a first order, the total mass associated with the tidally stripped population and compare it to the current mass. By calculating the mass lost in the field in the past 5~Gyr as the sum of the mass of all particles\footnote{\label{footnote:mass}To estimate the mass of particles in each cluster we have quantified the number of particles, $N_{bound}$, bound to the cluster at the end of the simulation and calculated the corresponding particle mass as $m_p=M_{GC}/N_{bound}$, where $M_{GC}$ is the current mass of a cluster given in Table~\ref{TableIC}.} that have escaped the cluster ($t_{\rm esc}> 0$), we find that the PI model sheds $2.1\times 10^{7} M_{\odot}$, which is 55\% of the GC population's current mass. The PII model shed is $2.7\times 10^{6} M_{\odot}$, which is 7\% of the current mass. Similarly, the PII-0.3-SLOW model lost $3.7\times 10^{6} M_{\odot}$, which is 10\% of the current mass and gives a half mass radius of 6.3~kpc.

    This mass roughly constitutes one-hundredth to one-tenth of the total stellar halo mass \citep{blandhawthorn16} and it is probably only a lower limit on the mass of escaped stars in the field, since a number of clusters initially in the Galaxy must have been destroyed over time (see introduction) and are thus no longer identifiabl as globular clusters today. It is also interesting to note that escaped stars are mostly redistributed in the inner Galaxy, with the half-mass radius of the PI model being 4.0~kpc and that of the PII and PII-0.3-SLOW models being 6.3~kpc. 

    The total mass lost from the clusters, as well as its spatial distribution, thus depends on the Galactic potential adopted: the variations between the PI model and both the PII and PII-0.3-SLOW models are of course caused by the PI's inclusion of the bulge, which leads to larger tidal forces in the center of the galaxy and subsequently drives larger mass loss. 

    Despite the differences in the modeling approach, it is interesting to compare our results to those of \citet{baumgardt2017global}. Briefly, our experiments differ in the following ways: they employ N-body simulations while we use our test-particle approach; they have an integration time of 12 Gyrs compared to our 5 Gyrs; lastly, their clusters have circular orbits in a Galactic potential modeled as an isothermal sphere as compared to the more realistic orbits and Galactic potentials considered in this work. Interestingly, the authors find that over 12 Gyrs their population of globular clusters lose two-thirds of their initial mass. This is roughly consistent with our application of the PI model, whose globular clusters shed 35\% of their initial mass in 5 Gyrs, which is roughly half the mass found by \citet{baumgardt2017global} in a period that is also about half as long. 

    \subsection{From the nearest to the furthest extra-tidal structures}\label{streamvsD}

    The analysis presented in this section along with the corresponding information are given in Figures~\ref{D0-10},~\ref{D10-20},~\ref{D20-30},~and~\ref{D30-300} are restricted to the PII model.

    \twocolumn
    \begin{figure*}[h!]
        \begin{center}
            \includegraphics[clip=true, trim = 0mm 15mm 0mm 20mm, width=0.9\columnwidth]{images/PII_ensemble_LB_D0-5_scatter.pdf}
            \includegraphics[clip=true, trim = 0mm 15mm 0mm 20mm, width=0.9\columnwidth]{images/PII_ensemble_LB_D5-10_scatter.pdf}

            \includegraphics[clip=true, trim = 0mm 20mm 0mm 20mm, width=0.9\columnwidth]{images/PII_ensemble_LB_D0-5_mass_est_new.pdf}
            \includegraphics[clip=true, trim = 0mm 20mm 0mm 20mm, width=0.9\columnwidth]{images/PII_ensemble_LB_D5-10_mass_est_new.pdf}

            \includegraphics[clip=true, trim = 0mm 20mm 0mm 20mm, width=0.9\columnwidth]{images/PII_ensemble_LB_D0-5_PML_new.pdf}
            \includegraphics[clip=true, trim = 0mm 20mm 0mm 20mm, width=0.9\columnwidth]{images/PII_ensemble_LB_D5-10_PML_new.pdf}

            \includegraphics[clip=true, trim = 0mm 20mm 0mm 20mm, width=0.9\columnwidth]{images/PII_ensemble_LB_D0-5_PMB_new.pdf}
            \includegraphics[clip=true, trim = 0mm 20mm 0mm 20mm, width=0.9\columnwidth]{images/PII_ensemble_LB_D5-10_PMB_new.pdf}

            \includegraphics[clip=true, trim = 0mm 20mm 0mm 20mm, width=0.9\columnwidth]{images/PII_ensemble_LB_D0-5_RV_new.pdf}
            \includegraphics[clip=true, trim = 0mm 20mm 0mm 20mm, width=0.9\columnwidth]{images/PII_ensemble_LB_D5-10_RV_new.pdf}
        \end{center}
    \caption{The extra-tidal features colored and weighted by various quantities within limited distance bins from the Sun projected in the $(\ell, b)$ plane. All outputs arrise from using the PII galactic potential model and only the reference simulation is shown in order to preserve clairty. \emph{Left column}: Extra-tidal features found at a distance of [0,5]~kpc from the Sun. \emph{Top row:} Scatter plot, with different colors indicating different progenitor clusters. \emph{Second row:} Mass density map in logarithmic scale. \emph{Third row:} Map color-coated by proper motions in longitudinal direction.    \emph{Fourth row:} Map color-coated by proper motions in latitudinal direction.  \emph{Bottom row:} Map color-coded by line-of-sight velocities. \emph{Right column}: Same as left column, but for the tidal features found at a distance of [5, 10]~kpc from the Sun. Note: the 10 colors used in the top panels are recycled between the 159 clusters, thus, all particles from the same cluster share one color, but a color is not unique to a cluster. }\label{D0-10}
    \end{figure*}
    \onecolumn

    \twocolumn
    \begin{figure*}[h!]
        \begin{center}
            \includegraphics[clip=true, trim = 0mm 15mm 0mm 20mm, width=0.9\columnwidth]{images/PII_ensemble_LB_D10-15_scatter.pdf}
            \includegraphics[clip=true, trim = 0mm 15mm 0mm 20mm, width=0.9\columnwidth]{images/PII_ensemble_LB_D15-20_scatter.pdf}

            \includegraphics[clip=true, trim = 0mm 20mm 0mm 20mm, width=0.9\columnwidth]{images/PII_ensemble_LB_D10-15_mass_est_new.pdf}
            \includegraphics[clip=true, trim = 0mm 20mm 0mm 20mm, width=0.9\columnwidth]{images/PII_ensemble_LB_D15-20_mass_est_new.pdf}

            \includegraphics[clip=true, trim = 0mm 20mm 0mm 20mm, width=0.9\columnwidth]{images/PII_ensemble_LB_D10-15_PML_new.pdf}
            \includegraphics[clip=true, trim = 0mm 20mm 0mm 20mm, width=0.9\columnwidth]{images/PII_ensemble_LB_D15-20_PML_new.pdf}

            \includegraphics[clip=true, trim = 0mm 20mm 0mm 20mm, width=0.9\columnwidth]{images/PII_ensemble_LB_D10-15_PMB_new.pdf}
            \includegraphics[clip=true, trim = 0mm 20mm 0mm 20mm, width=0.9\columnwidth]{images/PII_ensemble_LB_D15-20_PMB_new.pdf}

            \includegraphics[clip=true, trim = 0mm 20mm 0mm 20mm, width=0.9\columnwidth]{images/PII_ensemble_LB_D10-15_RV_new.pdf}
            \includegraphics[clip=true, trim = 0mm 20mm 0mm 20mm, width=0.9\columnwidth]{images/PII_ensemble_LB_D15-20_RV_new.pdf}
        \end{center}
        \caption{As for  Fig.~\ref{D0-10}, but for two father distance bins. Here, we are beyond the galactic nuclues and tidal debris are less diffuse in nature and more often stream-like. The \emph{left column} shows debris at distances between [10, 15]~kpc from the Sun while the \emph{right column} shows debris at distances between [15, 20]~kpc from the Sun.}\label{D10-20}
    \end{figure*}  
    \onecolumn  

    \twocolumn
    \begin{figure*}[h!]
        \begin{center}
            \includegraphics[clip=true, trim = 0mm 15mm 0mm 20mm, width=0.9\columnwidth]{images/PII_ensemble_LB_D20-25_scatter.pdf}
            \includegraphics[clip=true, trim = 0mm 15mm 0mm 20mm, width=0.9\columnwidth]{images/PII_ensemble_LB_D25-30_scatter.pdf}

            \includegraphics[clip=true, trim = 0mm 20mm 0mm 20mm, width=0.9\columnwidth]{images/PII_ensemble_LB_D20-25_mass_est_new.pdf}
            \includegraphics[clip=true, trim = 0mm 20mm 0mm 20mm, width=0.9\columnwidth]{images/PII_ensemble_LB_D25-30_mass_est_new.pdf}


            \includegraphics[clip=true, trim = 0mm 20mm 0mm 20mm, width=0.9\columnwidth]{images/PII_ensemble_LB_D20-25_PML_new.pdf}
            \includegraphics[clip=true, trim = 0mm 20mm 0mm 20mm, width=0.9\columnwidth]{images/PII_ensemble_LB_D25-30_PML_new.pdf}

            \includegraphics[clip=true, trim = 0mm 20mm 0mm 20mm, width=0.9\columnwidth]{images/PII_ensemble_LB_D20-25_PMB_new.pdf}
            \includegraphics[clip=true, trim = 0mm 20mm 0mm 20mm, width=0.9\columnwidth]{images/PII_ensemble_LB_D25-30_PMB_new.pdf}

            \includegraphics[clip=true, trim = 0mm 20mm 0mm 20mm, width=0.9\columnwidth]{images/PII_ensemble_LB_D20-25_RV_new.pdf}
            \includegraphics[clip=true, trim = 0mm 20mm 0mm 20mm, width=0.9\columnwidth]{images/PII_ensemble_LB_D25-30_RV_new.pdf}
        \end{center}
        \caption{As for  Fig.~\ref{D0-10}~\&~\ref{D10-20}, but for increased distances from the Sun. Here we notice that the tidal debris becomes more sparse as we move even father away from the galactic center. The \emph{left column} shows debris at distances between [20, 25]~kpc from the Sun while the \emph{right column} shows debris at distances between [25, 30]~kpc from the Sun.}\label{D20-30}
    \end{figure*}  
    \onecolumn  

    \twocolumn
    \begin{figure*}[h!]
        \begin{center}
            \includegraphics[clip=true, trim = 0mm 15mm 0mm 20mm, width=0.9\columnwidth]{images/PII_ensemble_LB_D30-35_scatter.pdf}
            \includegraphics[clip=true, trim = 0mm 15mm 0mm 20mm, width=0.9\columnwidth]{images/PII_ensemble_LB_D35-300_scatter.pdf}

            \includegraphics[clip=true, trim = 0mm 15mm 0mm 20mm, width=0.9\columnwidth]{images/PII_ensemble_LB_D30-35_mass_est_new.pdf}
            \includegraphics[clip=true, trim = 0mm 15mm 0mm 20mm, width=0.9\columnwidth]{images/PII_ensemble_LB_D35-300_mass_est_new.pdf}


            \includegraphics[clip=true, trim = 0mm 20mm 0mm 20mm, width=0.9\columnwidth]{images/PII_ensemble_LB_D30-35_PML_new.pdf}
            \includegraphics[clip=true, trim = 0mm 20mm 0mm 20mm, width=0.9\columnwidth]{images/PII_ensemble_LB_D35-300_PML_new.pdf}

            \includegraphics[clip=true, trim = 0mm 20mm 0mm 20mm, width=0.9\columnwidth]{images/PII_ensemble_LB_D30-35_PMB_new.pdf}
            \includegraphics[clip=true, trim = 0mm 20mm 0mm 20mm, width=0.9\columnwidth]{images/PII_ensemble_LB_D35-300_PMB_new.pdf}

            \includegraphics[clip=true, trim = 0mm 20mm 0mm 20mm, width=0.9\columnwidth]{images/PII_ensemble_LB_D30-35_RV_new.pdf}
            \includegraphics[clip=true, trim = 0mm 20mm 0mm 20mm, width=0.9\columnwidth]{images/PII_ensemble_LB_D35-300_RV_new.pdf}
        \end{center}
        \caption{As for  Fig.~\ref{D0-10}-\ref{D20-30}, but now we include debris from the most outer extent of  the galaxy. The \emph{left column} shows debris at distances between [30, 35]~kpc from the Sun while the \emph{right column} shows debris at distances between [35, 300]~kpc from the Sun.}\label{D30-300}
    \end{figure*}    
    \onecolumn
    In this section, we analyze structures located at different distances from the Sun. We identify structures as main or secondary on the basis of the fraction of stars they represent. If, for example, a cluster contributes more than 10\% of its stripped stars in a given distance range, we consider the associated structure to be significant and we define it as a main structure in this distance range. If, on the other hand, the fraction of stripped stars from a cluster is between 1\% and 10\%, we define the associated structure  as being "secondary." Structures that constitute less than 1\% of the cluster mass are considered insignificant in that range. The main and secondary structures for each distance bin are reported in Table~\ref{summarySTREAMS}, where they are named by their progenitor cluster. Below, we describe the structures encountered at different distances from the Sun, from the closest to the most distant. For the following discussions,  we refer to Figs.~\ref{D0-10} to \ref{D30-300}, where we report the different streams in a given distance range (top row in each figure), the corresponding mass density generated by all the structures found in that bin (second row), the longitudinal and latitudinal proper-motions map (third and fourth rows), and the line-of-sight velocities (bottom row).

    \paragraph{The [0--2]~kpc distance range: }
    Among the 159 simulated clusters, only  NGC~6121 is currently at a distance of less than 2~kpc from the Sun. In addition to stars stripped from this cluster within this distance-to-the-Sun range, we have identified very diffuse and low density extra-tidal stars associated with five other clusters, which are currently several kpc away from the Sun, as reported in Table~\ref{summarySTREAMS}. Of these, only UKS~1 and NGC~6121 have a significant fraction of their stripped mass in this distance range. All the others contribute with only a few percent. None of these clusters seem to show well-defined, stream-like features in the ($\ell,b$) plane. This extra-tidal material appears indeed quite uniformly redistributed, in a latitude range mostly inside $-30^\circ$ to $30^\circ$, and is mostly found at negative longitudes. Because in this interval range there is no remarkable extra-tidal structure found on the sky, we decided to plot this distance bin together with the [2--5]~kpc bin (see Fig.~\ref{D0-10}, left column).


    \paragraph{The [2--5]~kpc distance range: }
    As the distance from the Sun increases, many more tidal structures are intercepted. Eleven globular clusters are found at a distance between 2 and 5~kpc from the Sun; and together with structures emanating from these clusters, we also find extra-tidal material associated with other 48 clusters, 19 of which are significant (as listed in Table~\ref{summarySTREAMS}). Some extra-tidal structures are clearly identifiable, even when over-plotted with all the others: this is the case, for example, for the tidal material associated with the E~3 cluster, which appears as an extended thick stream (as shown in Fig.~\ref{D0-10}) at approximately $-100^{\circ}$ longitude. This is also the case for the complex tidal structure associated with BH~140, which resembles a ribbon in the sky, with a bifurcation at negative longitudes whose edges extend to about $-30^\circ$ and $30^\circ$ latitude. In addition to these, there are a series of circular halos concentric about the Galactic center, with abrupt drop-offs in density tracing the furthest extent of diffuse debris emanating from a variety of clusters. Overall, few streams are immediately recognizable in this range, although plenty of debris is present. This is expected given that this range samples a bite of the Sun-side of the Galactic disk and that this range has a relatively small volume. In this, as in the following distance ranges, the $v_{\ell os}$ maps show that the tidal material at negative Galactic longitudes has, on average, positive $v_{\ell os}$ while material at positive Galactic longitudes has, on average, negative $v_{\ell os}$, which is due to the solar reflex velocity. This is the same trend observed for the whole set of radial (i.e., line-of-sight) velocities in Gaia DR3; for instance, we refer to the bottom panel of Fig.~5 from \citet{katz22}, although less extreme velocities are reported in their plot since their dataset is dominated by disk stars, whereas our maps have a high proportional contribution of halo stars (additionally, they use median values in their bins, while we use an average). In order to quantify the net rotation of the system of streams, we calculated the mean angular momentum about the Galactic pole as $\braket{L_z} =\sum_i L_{z,i} m_{p,i} / \sum_i m_{p,i} $, where $m_{p,i}$ is the mass of each star particle indexed by $i$ (as discussed in footnote~\ref{footnote:mass}) and $L_{z,i}$ is the corresponding particle's angular momentum. The mean angular momentum is found to be $\braket{L_z}=-300$~kpc~km~s$^{-1}$, which shows a slight co-rotation of the system of streams with the disk though there is much dispersion about this value -- as shown in bottom right panel of Fig.~\ref{orbparam} in the Appendix.


    \paragraph{The [5--10]~kpc distance range: }
    The [5--10]~kpc distance range, which includes the Galactic center, contains much more material. A total of 79 clusters are found in this range, together with tidal structures associated with 131 different progenitors redistributed among main and secondary structures. Some tiny streams are visible in the density maps as well as in proper motions and line-of-sight velocity spaces (see Fig.~\ref{D0-10}, right column): the trailing portion of the tail of the globular cluster Pal~1, at $(\ell, b)\sim(140^\circ, 25^\circ)$; the most extreme portion of the trailing tail of NGC~3201 at $(\ell, b)\sim(150^\circ, -37^\circ)$; the waterfall-like shape of NGC~288, particularly evident at $b \lesssim -60^{\circ}$; the thin inverted U-shape of NGC~4590 at positive latitudes spanning a large longitude extent from $\ell \simeq -60^\circ$ to $100^\circ$; the portion of the E~3 tails the closest to the cluster at $(\ell, b)\sim(-75^{\circ},-15^{\circ})$, which continues from the more easily recognizable portion in the [0--5]~kpc bin.

    \paragraph{The [10--15]~kpc distance range: }
    In the [10--15]~kpc range, we find tidal structures associated with 134 progenitors, as listed in Table~\ref{summarySTREAMS} and reported in Fig.~\ref{D10-20} (left column), 27 of which are related to globular clusters that are also found in this distance bin. We note that at these distances from the Sun, the distribution of tidal features in directions towards the Galactic center, from $-30^{\circ}$ to $30^{\circ}$ longitude, appears less fuzzy than the one characterizing the [0--5] and [5--10]~kpc distance bins. Tidal features here are beyond the Galactic center, and are mostly associated with disk or halo clusters. 

    Among the thinnest structures, we find the stream associated with Pal~1, at $(\ell, b)\sim(120^\circ, 15^\circ),$ which is also visible in the distance bin [5--10]~kpc (see previous discussion), but which is even more elongated here. NGC~6101 shows the nearest portion of its long thin diagonal tidal tail that spans negative longitudes and ranges from $-15^{\circ}$ to $45^\circ$ latitude. Additionally, this stream is also unique against its counter parts in proper motion space. NGC~5053's nearest portion appears as a vertical tidal tail at $-80^{\circ}$ longitude. Similarly, NGC~5466 is shown vertically at $25^{\circ}$ longitude.

    Among the thickest structures, we can recognize general diffuse and bowtie-like shapes. There are also spoke-like structures departing radially from the Galactic center. For instance, we can associated the extra-tidal material with NGC~7078 at $(\ell, b)\sim(60^\circ,-28^{\circ}),$ as well as NGC~7089, which is nearly parallel to the previous structure, but at lower latitudes at $(\ell, b)\sim(50^\circ,-40^{\circ})$.

    \paragraph{The [15--20]~kpc distance range: }
    At larger distances ($[15--20]$~kpc range, see Fig.~\ref{D10-20}), some of the most striking features are associated with the clusters NGC~5024 and NGC~5053, whose long thin tails essentially overlap in this distance, with the latter covering the former, and appearing at high latitudes spanning a longitudinal range from $\ell \sim -90^{\circ}$ to $\ell \sim 45^{\circ}$. Again, the long thin stream of NGC~5466 appears in this range (as will be the case for the next) and is at high latitudes at roughly $85^{\circ}$ and positive longitudes. There is also the thicker extended structure of NGC~4147 whose diffuse structure emanates from about $(\ell, b)\sim(-100^{\circ},80^{\circ})$.

    In this distance bin, we find long tidal tails emanating from globular clusters associated  with the Sagittarius dwarf galaxy, which are particularly visible at negative longitudes: a long thin stream is associated  with Pal~12 at positive longitudes and latitudes $b \le-15^{\circ}$, as well as two overlapping structures at $0^\circ \lesssim \ell \lesssim 30^\circ$ longitude, namely, Ter~7 (and Ter~8 in the next distance bin). A word of caution is needed here: the mass loss from these clusters may be incorrect, since we do not include the presence of the Sagittarius dwarf galaxy itself. The potential well associated with this latter could change the tidal effects experienced by clusters associated with Sagittarius, especially in the case of NGC~6715, which sits at the center of this dwarf galaxy. The inclusion of the Sagittarius dwarf will be the subject of future investigations. Overall, in this distance bins, we find 11 clusters  and 61 streams, all listed in Table~\ref{summarySTREAMS}.

    \paragraph{The [20--25] and [25--30]~kpc distance ranges: }
    In the following distance bins (at [20--25]~kpc and [25--30]; see Fig.~\ref{D20-30}), globular clusters and extra-tidal structures become less numerous, although some are still visible, such as Pal~5 at $(\ell, b)\sim(0^{\circ},45^{\circ})$. In more detail, in the [20--25]~kpc bin we find tidal features associated  with 37 different progenitors, 8 of which are associated with globular clusters whose current positions are in the same distance bin; in the [25--30]~kpc bin, 7 clusters are found, together with tidal features associated with 30 other progenitor clusters which do not lie in this same distance range. In both bins, the streams emanating from globular clusters associated with the Sagittarius dwarf galaxy are still visible, as well as the most extreme portion of the tail associated with NGC~5466. 

    \paragraph{The [30--35] and [35--300]~kpc distance ranges: }
    Finally, in the last distance bins (see Fig.~\ref{D30-300}), thin streams become rare. Some small streams are visible: Pyxis at $(\ell, b)\sim(-100^{\circ},0^{\circ})$; NGC~2419 at $(\ell, b)\sim(-180^{\circ},30^{\circ})$; Pal~4 at $(\ell, b)\sim(-160^{\circ},75^{\circ})$; Pal~3 at $(\ell, b)\sim(-120^{\circ},45^{\circ})$. Many more have a diffuse and halo-like structure. For instance, the blob associated with Pal~15, centered at $(\ell, b)\sim (15^\circ, 20^\circ$); AM~1 at $(\ell, b)\sim(-100^{\circ},-55^{\circ})$; Eridanus at $(\ell, b)\sim(-140^{\circ},-45^{\circ})$; Pal~14 at $(\ell, b)\sim(30^{\circ},45^{\circ})$; Laevens~3 at $(\ell, b)\sim(65^\circ,-20^\circ)$, which is completely enveloped by NGC~7006. In total, in these two distance bins, we find 4 and 12 clusters, respectively, along with their associated streams, together with extra-tidal material associated with 14 and 19 progenitors in total. 


    \begin{table*}
        \tiny
        \centering                                      % used for centering table
        \caption{List of tidal structures found in different intervals of distance to the Sun. The tidal structures are named by their progenitor clusters. If the parent cluster is also in the distance range under consideration, the name of the tidal structure is shown in bold. Second column reports the main structures found in a given distance bin. The third column list secondary structures. The numbers in parenthesis in the second column  and third column (numbers with normal font) correspond to the total number of main and secondary structures found in a given distance range. The number of clusters in each distance bin is also reported in the second column (bold numbers in parenthesis). }\label{summarySTREAMS}
        \begin{tabularx}{\textwidth}{l X X }          % centered columns (4 columns)
        \hline
        Distance (kpc) &   Main tidal structures & Secondary tidal structures\\ 
        %(kpc) &  \\ 
        \hline   \\
        \tiny
        \vspace{0.1cm}
        [0-2] & (\textbf{1},2) \textbf{NGC6121}, UKS1 & (4) BH140, Djor1, NGC6333, NGC6356\\ 
        \vspace{0.1cm}

        [2-5] & (\textbf{11},30) \textbf{NGC6397}, \textbf{NGC6544}, \textbf{NGC3201}, \textbf{BH140}, \textbf{NGC104}, \textbf{NGC6838}, \textbf{NGC6366}, \textbf{NGC6752}, \textbf{IC1276}, \textbf{NGC6656}, \textbf{2MASS-GC01}, NGC6284, NGC6356, NGC6287, VVV-CL001, NGC6254, NGC5927, E3, NGC6121, VVV-CL001, Djor1, UKS1, Ter10, 2MASS-GC02, Pal10, NGC5139, NGC6333, NGC6441, NGC6541, NGC288 & (29) FSR1716, FSR1758, NGC1851, NGC1904, NGC2298, NGC2808, NGC362, NGC4372, NGC4833, NGC5897, NGC5986, NGC6205, NGC6218, NGC6235, NGC6273, NGC6316, NGC6352, NGC6388, NGC6496, NGC6681, NGC6749, NGC6760, NGC6809, NGC6864, NGC7078, Pal2, Pal8, Ter12, Ton2\\ 
        \vspace{0.1cm}

        [5-10] & (\textbf{79},124) \textbf{VVV-CL001}, \textbf{NGC7099}, \textbf{NGC6362}, \textbf{Ton2}, \textbf{Djor1}, \textbf{VVV-CL001}, \textbf{NGC6496}, \textbf{Djor2}, \textbf{NGC6535}, \textbf{NGC6528}, \textbf{NGC6539}, \textbf{NGC6540}, \textbf{NGC6553}, \textbf{2MASS-GC02}, \textbf{Ter12}, \textbf{BH261}, \textbf{Ter9}, \textbf{NGC6712}, \textbf{NGC6717}, \textbf{NGC6723}, \textbf{NGC6749}, \textbf{NGC6760}, \textbf{Pal10}, \textbf{HP1}, \textbf{Ter4}, \textbf{Ter2}, \textbf{Ter3}, \textbf{NGC2298}, \textbf{E3}, \textbf{NGC4372}, \textbf{NGC4833}, \textbf{NGC5904}, \textbf{NGC5927}, \textbf{FSR1716}, \textbf{Lynga7}, \textbf{NGC6144}, \textbf{NGC6171}, \textbf{NGC6352}, \textbf{ESO452-SC11}, \textbf{NGC6218}, \textbf{FSR1735}, \textbf{NGC6254}, \textbf{NGC6256}, \textbf{NGC6287}, \textbf{NGC6293}, \textbf{NGC6304}, \textbf{NGC6355}, \textbf{NGC6809}, \textbf{NGC6637}, \textbf{NGC6402}, \textbf{NGC6325}, \textbf{NGC6341}, \textbf{NGC6342}, \textbf{NGC6380}, \textbf{NGC6401}, \textbf{NGC6440}, \textbf{NGC6517}, \textbf{NGC6522}, \textbf{NGC6541}, \textbf{NGC6558}, \textbf{NGC6624}, \textbf{NGC6626}, \textbf{NGC6638}, \textbf{NGC6642}, \textbf{NGC6652}, \textbf{NGC6681}, \textbf{Pal6}, \textbf{Ter1}, \textbf{Ter5}, \textbf{Ter6}, \textbf{NGC6333}, \textbf{NGC288}, \textbf{NGC362}, \textbf{NGC6273}, \textbf{NGC6266}, \textbf{NGC6205}, \textbf{NGC5139}, \textbf{Liller1}, \textbf{NGC5946}, NGC5897, NGC1904, NGC6752, NGC6656, NGC6121, NGC1851, NGC6864, NGC7078, NGC7089, NGC6316, NGC5272, NGC6779, NGC2808, NGC4590, Rup106, NGC104, NGC4147, NGC3201, Pal11, Pal1, NGC1261, BH140, NGC6235, Ter10, NGC6569, UKS1, NGC6453, NGC6139, NGC6426, NGC6397, NGC6093, NGC6388, FSR1758, IC1276, NGC6838, NGC6366, NGC5986, NGC6441, NGC6356, NGC6584, NGC5286, NGC6544, NGC6284, Pal8, NGC6981 & (7) 2MASS-GC01, IC1257, NGC5634, NGC5694, NGC6229, NGC7006, Pal2\\ 
        \vspace{0.1cm}

        [10-15] & (\textbf{27},115) \textbf{NGC6453}, \textbf{NGC5272}, \textbf{NGC6584}, \textbf{Pal8}, \textbf{NGC6316}, \textbf{NGC5897}, \textbf{NGC6284}, \textbf{NGC6139}, \textbf{NGC6093}, \textbf{NGC5986}, \textbf{NGC5286}, \textbf{NGC6235}, \textbf{NGC2808}, \textbf{NGC1904}, \textbf{NGC1851}, \textbf{NGC6101}, \textbf{NGC6779}, \textbf{NGC6388}, \textbf{Pal11}, \textbf{Pal1}, \textbf{Ter10}, \textbf{NGC4590}, \textbf{NGC7089}, \textbf{NGC6569}, \textbf{NGC7078}, \textbf{FSR1758}, \textbf{NGC6441}, NGC6304, NGC6254, FSR1735, NGC6426, NGC6397, NGC6362, Ton2, NGC6256, NGC6366, NGC6287, NGC6352, NGC6355, NGC6356, NGC6218, NGC6293, VVV-CL001, NGC6171, NGC5024, NGC1261, NGC2298, E3, NGC3201, NGC4147, NGC4372, Rup106, BH140, NGC4833, NGC5053, Ter3, NGC5466, NGC5634, IC4499, NGC5904, NGC5927, FSR1716, UKS1, NGC6121, NGC6144, Lynga7, NGC6553, VVV-CL001, NGC6496, NGC5946, NGC6205, NGC6266, NGC6273, NGC6333, NGC6341, NGC6342, NGC6401, NGC6402, NGC6517, NGC6541, NGC6544, NGC6558, NGC6626, NGC6652, NGC6656, NGC6681, NGC6864, Pal6, Ter1, Ter5, NGC5139, NGC362, NGC104, Ter12, Djor2, NGC6535, NGC6528, NGC6539, NGC6540, 2MASS-GC01, Ter9, 2MASS-GC02, IC1276, BH261, NGC7099, NGC6712, NGC6723, NGC6749, NGC6752, NGC6760, NGC6809, NGC6838, NGC6934, NGC6981, NGC288 & (18) Djor1, ESO280-SC06, ESO452-SC11, HP1, IC1257, NGC5694, NGC5824, NGC6229, NGC6325, NGC6380, NGC6638, NGC6642, NGC6717, NGC7006, NGC7492, Pal10, Pal2, Ter6\\ 
        \vspace{0.1cm}

        [15-20] & (\textbf{11},46) \textbf{NGC6356}, \textbf{NGC5466}, \textbf{NGC1261}, \textbf{UKS1}, \textbf{NGC4147}, \textbf{IC4499}, \textbf{NGC5024}, \textbf{NGC5053}, \textbf{Pal12}, \textbf{NGC6981}, \textbf{NGC6934}, NGC6101, NGC5904, Pal5, NGC5824, NGC5634, NGC7089, FSR1758, NGC4833, BH140, NGC4590, Rup106, NGC3201, Pal2, NGC5272, Djor1, NGC6426, NGC7078, NGC6864, NGC6715, NGC6656, NGC6341, NGC6333, NGC5286, NGC362, NGC2808, NGC1851, NGC104, NGC7492, Pal10, Ter7, NGC6779, NGC6584, IC1276, ESO280-SC06, NGC288 & (15) 2MASS-GC02, IC1257, NGC1904, NGC2298, NGC4372, NGC5139, NGC5694, NGC6121, NGC6205, NGC6229, NGC6838, NGC7006, NGC7099, Pal13, Ter8\\ 
        \vspace{0.1cm}

        [20-25] & (\textbf{8},29) \textbf{Pal5}, \textbf{NGC7492}, \textbf{Pal13}, \textbf{Rup106}, \textbf{NGC6864}, \textbf{NGC6426}, \textbf{ESO280-SC06}, \textbf{Ter7}, NGC5466, IC4499, NGC5634, NGC7089, NGC5824, NGC5024, NGC4590, NGC4147, NGC3201, NGC5272, IC1257, NGC5904, NGC6101, NGC6584, Arp2, Ter8, NGC6934, NGC6981, Pal12, NGC6229, Pal2 & (9) Djor1, FSR1758, NGC1261, NGC1851, NGC1904, NGC2298, NGC2808, NGC5694, NGC7006\\ 
        \vspace{0.1cm}

        [25-30] & (\textbf{7},20) \textbf{NGC6715}, \textbf{Pal2}, \textbf{AM4}, \textbf{NGC5634}, \textbf{Ter8}, \textbf{Arp2}, \textbf{IC1257}, NGC5824, Rup106, NGC5694, IC4499, NGC6101, NGC5904, NGC6229, Ter7, NGC6934, NGC6981, Pal13, NGC7492, Whiting1 & (10) NGC1851, NGC1904, NGC3201, NGC4147, NGC4590, NGC5466, NGC7006, NGC7089, Pal15, Pyxis\\ 
        \vspace{0.1cm}

        [30-35] & (\textbf{4},11) \textbf{NGC6229}, \textbf{NGC5824}, \textbf{NGC5694}, \textbf{Whiting1}, NGC7006, Ter8, Arp2, Ter7, Rup106, Pyxis, Pal2 & (3) NGC6101, NGC6934, Pal15\\ 
        \vspace{0.1cm}

        [35-300] & (\textbf{12},15) \textbf{Laevens3}, \textbf{NGC7006}, \textbf{SagittariusII}, \textbf{Pal15}, \textbf{Pal14}, \textbf{Crater}, \textbf{Pal4}, \textbf{Pal3}, \textbf{Pyxis}, \textbf{NGC2419}, \textbf{Eridanus}, \textbf{AM1}, NGC6715, NGC5824, NGC5694 & (4) Arp2, NGC6934, Pal2, Ter8
        \end{tabularx}
        \normalsize
    \end{table*}


    \subsection{Disks of inner and outer halo clusters: A variety of morphologies and shapes for extra-tidal structures}\label{sec:morphologies}
        The analysis presented in the previous section allows us to appreciate the variety of morphologies found for extra tidal structures, from padlocks to ``Easter eggs,'' disks, ribbons, and canonical streams. Moreover, some structures are limited in latitude and longitude, while some others fill nearly the entire sky. 

        To more easily capture the similarity and differences in the morphology of the extra-tidal features surrounding Galactic globular clusters, we can group the latter on the basis of their orbital parameters\footnote{We caution the reader that the classification of disk, inner and outer clusters made in this Section is based on the orbital parameters of the clusters, as found when their orbits are integrated in model PII. This classification may slightly change if model PI or model PII-0.3-SLOW were adopted.}(see Appendix~\ref{class} for more details), as follows: \emph{Disk clusters}: A cluster is classified as a disk cluster if arctan($z_{max}/R_{max}$) $\le 10^\circ$, where $z_{max}$ and $R_{max}$ are, respectively, the maximum height above or below the Galactic plane reached by its orbit in the last 5~Gyr and its maximum in-plane distance from the Galactic center. \emph{Inner clusters}: All clusters with $r_{max} \le R_{\odot}$ that are not classified as disk clusters enter this group. Contrary to $R_{max}$, which is the maximum in-plane distance that a cluster reaches from the Galactic center, $r_{max}$ is the maximum 3D distance, that is, $r_{max} = \textrm{max}(\sqrt{R^2+z^2}),$ with the maximum calculated over the whole cluster orbit. \emph{Outer clusters}: All clusters with $r_{max} > R_{\odot}$ that are not classified as disk clusters are included in this group. 

        By using the orbital radius of the Sun as the criterion for inner and outer clusters, debris from outer clusters can span the whole sky while inner clusters must be restricted in longitude and latitude. With these definitions, 21 clusters are disk clusters, 71 are inner clusters, and 67 are outer clusters (see  Table~\ref{classification} in Appendix~\ref{class}). We emphasize that this classification does not aim to suggest any specific origin for these systems \citep[e.g., whether they are in-situ or accreted, see][]{massari19}, but it is uniquely based on their current orbital characteristics and helps in capturing some of the properties in the extension (projected in to the sky) and shape of their extra-tidal material, as we discuss in the following.

        \subsubsection{Extra-tidal features originating from disk clusters: ribbons in the Galactic plane}

            Disk clusters are defined on the basis of the flatness of their orbits (i.e., on their $z_{max}/R_{max}$ ratio). As a result, they typically are restricted to low latitudes, though the exact distribution depends on the relationship of their orbit to the solar radius. To specify, clusters whose $R_{max}$ are interior to the Solar radius generate tidal debris in a limited range in longitude and latitude. For instance, the material associated with clusters as Ter~1, Ter~5, Ter~6, and Ter~9 has a disk-like shape and is completely confined to $|\ell| <30^{\circ}$ and $|b|<10^{\circ}$. If $R_{max}$ is greater than the solar radius, material can cover the full longitude space and most of the material will still appear at low latitudes. This is the case, for example, of BH~140, whose escaped stars diffusely occupy all longitudes and most of them are found at $|b| \le 30^\circ$, while for Pal~2 and Pal~10, their extra-tidal stars have a very limited latitudinal extension and appear as ribbons in the sky. In the following, we discuss some of the structures associated with NGC~6121, Pal~2, and Pal~10. We refer to Appendix~\ref{allstreams} for the tidal structures generated by the whole set of disk clusters.  

            \twocolumn
            \begin{figure}
                \begin{center}
                    \includegraphics[clip=true, trim = 0mm 2mm 0mm 0mm, width=0.9\columnwidth]{images/PII_individual_NGC6121_NGC6121orbitRZXY.pdf}

                    \includegraphics[clip=true, trim = 0mm 20mm 0mm 10mm, width=0.9\columnwidth]{images/PII_individual_NGC6121_NGC6121orbit.pdf}

                    \includegraphics[clip=true, trim = 0mm 20mm 0mm 10mm, width=0.9\columnwidth]{images/PII_individual_NGC6121_NGC6121_LB_D.pdf}

                    \includegraphics[clip=true, trim = 0mm 20mm 0mm 10mm, width=0.9\columnwidth]{images/PII_individual_NGC6121_NGC6121_LB_tesc.pdf}
                \end{center}
                \caption{ \emph{Top-left panel:} Projection of the orbit of NGC~6121 in the meridional $\rm R-z$ plane. Colors trace time, from 5~Gyr ago (negative values) to 5~Gyr forward in time (positive values). \emph{Top-right panel:} Projection of the orbit of NGC~6121 in the Galactic $x-y$ plane.  \emph{Second row:} Projection of the NGC~6121 orbit for the past and future 5~Gyr, in the longitude-latitude plane.   \emph{Third row:} Projection in the longitude-latitude plane of the extra-tidal material lost by NGC~6121. Colors indicate the average distance of the stripped material from the Sun.  \emph{Bottom panel:} Projection in the longitude-latitude plane of the extra-tidal material lost by NGC~6121. Colors indicate the average time at which stellar particles become gravitationally unbound to the cluster, from 5~Gyr ago (negative time) to the current time ($\textrm{escape time} = 0$). In the bottom and middle panels, only the reference simulation without errors is shown for clarity. In all plots, the current position of the cluster is given by the white circle with a black outline. The yellow star, when present, indicates the position of the Sun. }\label{ngc6121_stream}
            \end{figure}
            \onecolumn


            \paragraph{NGC~6121: }

            With a current position at $x=-6.58$, $y=-0.28$ and $z=0.53$~kpc, NGC~6121 is the closest globular cluster to the Sun in our list. This cluster has a  remarkably planar orbit, with a maximal vertical excursion from the Galactic plane of only 0.5~kpc (see  Fig.~\ref{ngc6121_stream}), and an eccentricity $e=0.80 \pm 0.01$, which makes it oscillate between an apo-center at $R_{\rm max}=6.81 \pm 0.02$~kpc and a peri-center at $R_{min}=0.76 \pm 0.04$~kpc. Because this cluster lies inside the solar circle, its orbit is limited to a longitude interval from $-60^{\circ}$ to $60^{\circ}$; because the cluster currently lies very close to the Sun, and is at its highest height above the Galactic plane, the orbit forms a hook-like pattern in longitude-latitude space. This hook-like portion of the orbit, nearest to the cluster, is traced by the recently stripped tidal material (see Fig.~\ref{ngc6121_stream}), with a leading tail oriented mostly in a vertical direction in the $(\ell,b)$ plane, from the current cluster location, up to approximately $0^\circ$ latitude. This portion of the stripped material lies at less than 2~kpc from the Sun, and it constitutes the nearest stream found in our simulations.

            To our knowledge, no extra-tidal structure has been discovered  around NGC~6121 thus far. Recently, \citet{kundu19} used RR-Lyrae stars to trace the extra-tidal material around NGC~6121, without finding any clear evidence of structures. The current position of the cluster in the sky, at a latitude of roughly $20^\circ$ and at a longitude close to $0^\circ$, makes this search  difficult due to the strong contamination of field disk stars, despite the fact that this portion of the stream is expected to be very close to the Sun.

            \paragraph{Pal~10 and Pal~2: }

            Pal~10 and Pal~2 are two disk clusters whose orbit crosses the solar radius. While for Pal~10, the maximal in-plane distance, $R_{\rm max}$, is approximately $12$~kpc, in the case of Pal~2, the orbit can reach about 40~kpc from the Galactic center. The fact that both these clusters have a radial excursion of the orbit which is beyond the solar radius implies that their stripped stars can redistribute over the whole longitudinal range and, thus, is also in the anti-center direction. The fact that both clusters have orbits confined close to the disk plane implies that the escaped material redistributes in very thin structures (i.e., confined in a limited latitude interval), which resemble typical ``ribbons" in the sky.  Our models predict that both clusters are surrounded by a long stream of tidal material, which is however probably very difficult to identify because in both cases, these extra-tidal stars are confined close to the Galactic plane. No tidal streams emanating from these two clusters, to our knowledge, have been identified in the observational data so far.  Because the tidal structures associated with these two clusters have similar properties, we report only the case of Pal~10 in Fig.~\ref{pal10_stream}. 

            \begin{figure}
                \includegraphics[clip=true, trim = 0mm 2mm 0mm 0mm, width=0.9\columnwidth]{images/PII_individual_Pal10_Pal10orbitRZXY.pdf}

                \includegraphics[clip=true, trim = 0mm 20mm 0mm 10mm, width=0.9\columnwidth]{images/PII_individual_Pal10_Pal10orbit.pdf}

                \includegraphics[clip=true, trim = 0mm 20mm 0mm 10mm, width=0.9\columnwidth]{images/PII_individual_Pal10_Pal10_LB_D.pdf}

                \includegraphics[clip=true, trim = 0mm 20mm 0mm 10mm, width=0.9\columnwidth]{images/PII_individual_Pal10_Pal10_LB_tesc.pdf}
                \caption{Various projections of the host globular cluster orbit with its accompaning tidal debris as Fig.~\ref{ngc6121_stream}, but for the cluster Pal~10. \label{pal10_stream}}
            \end{figure}    
            
        \subsubsection{Extra-tidal features originating from inner clusters: Bow-ties and more complex shapes}

            We have defined inner globular clusters as systems that are not disk clusters (their orbit is not confined close to the Galactic plane) but that are confined inside the solar radius. Seventy-one clusters are found in this category (see  Table~\ref{classification}). We discuss some of them in the following.

            \paragraph{NGC~5946 \& NGC~5986:}

            These are inner-non disk clusters whose escaped stars redistribute in a characteristic ``bow-tie" shape. These stars are all confined in a relatively narrow longitudinal range (typically within $-30^\circ$ to $30^\circ$). Towards the edges of the longitude interval, the distribution of extra-tidal stars tends to flare, whereas it instead shrinks at zero longitude. These trends can be explained as an effect of the projection of the orbits of these clusters in the $(\ell, b)$ plane.  Moreover, because these clusters always stay in the inner region of the Galaxy, where the dynamical timescales are short, their orbit - and consequently their stripped stars - can experience many disk crossings over the whole duration of the simulation, filling the whole $(\ell, b)$ space allowed by their orbital parameters. An example of such a distribution is given in Fig.~\ref{ngc5986_stream} for the extra-tidal material associated with the cluster NGC~5986.

            \begin{figure}
                \begin{center}
                    \includegraphics[clip=true, trim = 0mm 2mm 0mm 0mm, width=0.9\columnwidth]{images/PII_individual_NGC5986_NGC5986orbitRZXY.pdf}
                    \includegraphics[clip=true, trim = 0mm 20mm 0mm 10mm, width=0.9\columnwidth]{images/PII_individual_NGC5986_NGC5986orbit.pdf}
                    \includegraphics[clip=true, trim = 0mm 20mm 0mm 10mm, width=0.9\columnwidth]{images/PII_individual_NGC5986_NGC5986_LB_D.pdf}
                    \includegraphics[clip=true, trim = 0mm 20mm 0mm 10mm, width=0.9\columnwidth]{images/PII_individual_NGC5986_NGC5986_LB_tesc.pdf}
                \end{center}
                \caption{Various projections of the host globular cluster orbit with its accompaning tidal debris as in Fig.~\ref{ngc6121_stream}~\&~\ref{pal10_stream}, but for the cluster NGC~5986.\label{ngc5986_stream}}
            \end{figure}            
