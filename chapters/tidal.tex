\section{General intro}

How should I introduce this paper??

\section{\bf{Numerical method}}\label{methods}

    To model the formation and evolution of extra-tidal features around Galactic globular clusters, we use a set of codes, called Globular Clusters' Tidal Tails (GCsTT)  developed by our group. It comprises two python codes, for the backward and forward integration of a stellar system, made of N test-particles (see Sect.~\ref{numerical}). These codes are separated for data organization and management, while the (computationally) most expensive part, namely, the calculation of the accelerations acting on the N particles and the orbits integration, is realized by means of a Fortran module written by our group. This module is interfaced to python by means of f2py directives from NumPy. The use of test-particle methods for modeling the tidal stripping process is widespread in the literature, where these methods are  usually applied to one or few clusters at a time \citep[see, e.g., ][]{lane12, mastrobuono12, palau19, piatti21a, grillmair22}. In this work, we apply a test-particle methodology to the whole set (159) of Galactic globular clusters for which this is currently possible, also taking into account, for each cluster, the errors on astrometry, line-of-sight velocities\footnote{Note: the term ``line-of-sight velocities''  adopted in this paper corresponds to the term ``radial velocities'' often used in the literature, as well as in the Gaia catalogues. We prefer the use of the first term, since the second is usually used also to indicate the (Galactocentric) radial velocities and can introduce some ambiguity, especially when different coordinate systems are used. We emphasize that the choice to use the term ``line-of-sight velocity'' is not new \citep[see, e.g., ][]{vasiliev21}.} and distances. In the following, we describe the two main steps of the procedure used by GCsTT  to simulate the tidal stripping process (Sect.~\ref{numerical}), the initial conditions adopted for the clusters' parameters and their mass distribution (Sect.~\ref{initialconds}), as well as the Galactic potentials (Sect.~\ref{galmod}).
    \subsection{Simulations of the tidal stripping process:  Two-step procedure}\label{numerical}
        To model the formation and evolution of extra-tidal features around Galactic globular clusters, and predict their current properties, we proceed as follows: 

        \textit{Step i: Backward integration. Reconstructing the globular cluster orbit over the last 5~Gyr}: First, for each Galactic globular cluster for which the distances from the Sun, proper motions, line-of-sight velocities, and structural parameters are  available (see Sect.~\ref{initialconds}), we determine their current positions and velocities in a Galactocentric reference frame, in which the Sun is at $(x_\odot, y_\odot, z_\odot) = (-8.34, 0., 0.027)$~kpc \citep{chen01, reid14} and at a given velocity for the local standard of rest, $v_{LSR}= 240$~km/s \citep{reid14}, and a peculiar velocity of the Sun with respect to the LSR, $(U_{\odot}, V_{\odot}, W_{\odot})  = (11.1, 12.24, 7.25)$~km/s \citep{schonrich10}. We then integrate the orbit of a single point mass, representing the cluster barycenter, backwards in time for 5~Gyr, and in this way, we retrieve its position and velocity at that time in the chosen Galactic potential (see Sect.~\ref{galmod}). We notice that other choices for the Sun's position or velocity with respect to the Galactocentric frame would have been possible. For example, \citet{piatti21a} adopted the same values as ours for the $v_{LSR}$ and for the peculiar velocity of the Sun but with a different distance to the Galactic center \citep[8.1~kpc in their work, see][]{gravity18}. The difference in the adopted position of the Sun is, however, generally smaller than the uncertainties affecting our knowledge of the distance of Galactic globular clusters to the Sun. For this reason, we do not to explore the dependency of the results presented in this paper with regard to these choices. 

        \textit{Step ii: Forward integration. Test-particle streams from the past to the present day}: Once the positions and velocities of the barycenter of each cluster, 5~Gyr ago, have been determined, we build  the corresponding $N$-body system, with N = 100 000 particles.  The phase-space coordinates of these particles are generated following a Plummer distribution, with the total mass and half-mass radius as described in Sect.~\ref{initialconds}. The barycenter of this $N$-body cluster is then assigned initial positions and velocities in the Galactic model, as those retrieved at step $(i),$ and the cluster is then integrated forward in time until the present day. Particles in this $N$-body system are modeled as test-particles, that is, they experience the gravitational field exerted by the globular cluster itself (see Sect.~\ref{initialconds}) and by the Galaxy (see Sect.~\ref{galmod}), but do not generate any gravitational field themselves. This allows us to maintain a computational time which scales as $O(N)$ and not as $O(N^2)$, as would be the case for a direct $N$-body self-consistent computation.

        In the following, we refer to these simulations, made by using the most probable values on distances, proper motions, and line-of-sight velocities, as the ``reference simulations.'' In addition, for each globular cluster, we also take into account the errors on its distance, proper motions, and line-of-sight velocity,  assuming Gaussian distributions of the errors, treated independently, and by generating 50 random realizations of these parameters.  For each of these realizations, we repeat the steps  described above, that is: (\textit{step i}) we determine the associated current positions and velocities in the chosen Galactocentric reference frame, we integrate the orbit of the single-point mass (representing the cluster barycenter) backwards in time, retrieving the corresponding values 5~Gyr ago, (\textit{step ii}) we build an $N$-body cluster containing $N$= 100~000 particles, with total mass and half-mass radius as those used for the reference simulation, and then we integrate the $N$-body cluster forwards in time until the present-day position. 

        To summarize, for a given Galactic potential, we run $159\times (50+1)=8109$ simulations, where 159 is the total number of clusters for which we currently have both 6D phase-space information and structural parameters. As we discuss in the following section, the whole set of globular clusters has been evolved in three different Galactic potentials, which implies that a total of 24 327 simulations have been run.

        For the orbit integration, a leap-frog algorithm is used, with a fixed time-step, $\Delta t$, and a total number of steps, $N_{steps}$, such that the total simulated time is  $\Delta t \times N_{steps}=5$~Gyr. The choice of the value of $\Delta t$ adopted to simulate each cluster in the Galactic potential has been based on the energy conservation of the corresponding cluster evolved in isolation (i.e., without the effect of the Galactic gravitational field for 5~Gyr). For the majority of the clusters (109/159), this value was set to $\Delta t = 10^5$~yr (for a corresponding value of $N_{steps}=50\,000$), while for the remaining clusters (50/159) a  $\Delta t = 10^4$~yr (for a corresponding value of $N_{steps}=500\,000$) was used. We refer to Appendix~\ref{deltat} (and in particular to Table~\ref{tcross-energy}) for additional details on the choice of $\Delta t$ for the whole set of clusters. As for the total simulated time, while globular clusters are much older than 5~Gyr, we chose this time limit because the longer back in time we could go, the less certain we would be of the Galactic environment. In addition, the last significant mergers in the Galaxy happened between 9 and 11~Gyr ago  \citep[see][]{belokurov18, helmi18, dimatteo19, gallart19, kruijssen20} -- well before the time interval simulated in this study. Other more recent interactions, such as the accretion of Sagittarius and of the Magellanic Clouds, may perturb the Galactic potential as well \citep[see, e.g.,][]{vasiliev21b} and we plan to investigate their impact on the properties of globular cluster streams in the future.

        For each realization, we generate an output file in an hdf5 format\footnote{\url{https://www.hdfgroup.org/solutions/hdf5/}} containing the values for the right ascension ($\alpha$), declination ($\delta$), distance from the Sun ($D$), along with the components for proper motion in the equatorial coordinate system ($\rm \mu_{\alpha}\cos(\delta)$ and $\rm \mu_\delta$), the line-of-sight velocity ($\rm v_{\ell os}$), longitude ($\ell$), latitude ($ b$), as well as the components for proper motion in the Galactic coordinate system ($\rm \mu_{\ell}\cos(\mathit{b})$ and $\rm \mu_b$) and the Galactocentric positions ($x, y, z$), velocities ($v_x, v_y, v_z$) and energy, $E$, of each particle in the simulated system. We used Astropy \citep{astropy13, astropy18} to convert the Galactocentric positions and velocities in the equatorial and Galactic quantities $\alpha, \delta, D, \rm \mu_{\alpha}\cos(\delta), \rm \mu_\delta, \rm v_{\ell os}, \ell, b, \rm \mu_{\ell}\cos(\mathit{b})$, and $\rm \mu_b$.

        For each particle, we also save its escape time $t_{\rm esc}$,  defined as the time at which the particle escapes from the cluster, that is, the time, $t,$ at which the particle satisfies the relation\footnote{If the particle is gravitationally bound to the cluster until the end of the simulation, $t_{\rm esc}$ is set equal to $-9999$.}:
        \begin{equation}
            E_{GC}= 0.5 \times \left( (v_x-v_{x, GC})^2+(v_y-v_{y,GC})^2+(v_z-v_{z,GC})^2\right)+\Phi_{GC} > 0,
        \end{equation}
        with $E_{GC}$ being the total specific energy of the particle relative to the cluster, that is, the sum of the potential energy, $\Phi_{GC}$, due to the gravitational field of the cluster (see Eq.~\ref{gcpot}), and of the kinetic energy, relative to the cluster barycenter, $T_{GC}=0.5 \times \left( \left(v_x-v_{x, GC}\right)^2+\left(vy-v_{y,GC}\right)^2+\left(vz-v_{z,GC}\right)^2\right)$, where $v_x, v_y$, and $v_z$ are its velocity components at time, $t$, and $v_{x,GC}, v_{y,GC}$, and $v_{z,GC}$  of the cluster barycenter at the same time. A positive value of $E_{GC}$ implies that the particle is no longer gravitationally bound to the cluster and, hence, it is lost in the field. Overall, the total volume of the whole set of  24 327 simulations, saved in hdf5 format, amounts to about 370 Gb.

        \subsection{Simulations of the tidal stripping process: Globular clusters' current and initial conditions and their gravitational potential}\label{initialconds}

        Steps $(i)$ and $(ii)$ described in the previous section require some input conditions to be adequately executed. The current distances from the Sun, proper motions, and line-of-sight velocities, as well as the related uncertainties, of all 159 globular clusters considered in this study are taken respectively from \citet{baumgardt21} and \citet{vasiliev21}. These values are then converted into Galactocentric positions and velocities by making use of Astropy and used as initial conditions to execute step $(i)$. 

        Step $(ii)$ requires generating an $N$-body system, representing the globular cluster, whose initial total mass and half-mass radius are assigned on the basis of their current values, as given by \citet{baumgardt18}\footnote{In particular, the adopted  values have been taken from the edition available at \url{https://people.smp.uq.edu.au/HolgerBaumgardt/globular/parameter.html}, up to January 14, 2022.} and reported in Table~\ref{TableIC}. As anticipated at step $(ii)$ in Sect.~\ref{numerical}, the phase-space coordinates of each $N$-body cluster are generated by assuming  a Plummer distribution of total mass, $M_{GC}$, and half-mass radius, $r_{h}$, for which the corresponding potential is:
        \begin{equation}\label{gcpot}
            \Phi_{GC}(r) = -\frac{GM_{GC}}{\sqrt{r^2+{r_c}^2}},
            \end{equation}
        where $r_c$ is the cluster scale radius and it is related to the half-mass radius, $r_{h}$, through $r_{h} \simeq 1.305 r_c$ \citep{heggie03}. The variable $r$ here indicates the distance of the test particle from the center of the cluster. For each cluster, the same Plummer distribution used to generate the $N$-body system is also used to calculate the accelerations exerted on each particle as the system moves through time. The Plummer sphere, representing the cluster potential, indeed moves through the Galaxy along the orbit retrieved at step (i), traveling this time in the opposite direction, from 5 Gyr ago to the present day.

        It might be noted that this implies that the globular cluster density profile and its internal parameters (total mass and characteristic radius) are constant over time in these models. This is, of course, a crude approximation, because in reality both the internal parameters and the density profile itself can change over time. We consider these assumptions to be acceptable within the scope of our work given that we are primarily interested in the distribution of extra-tidal stars, which had once escaped from the cluster have dynamics primarily dictated by the Galactic potential rather than the globular cluster itself. Of course,  the density of stars along the extra-tidal structures, as well as the total mass lost, depend on these assumptions. That is to say that if the mass of the cluster was not assumed constant over time, but could possibly decrease, the gravitational attraction exerted by the cluster itself on its stars would be weaker and this would lead to an increasing mass loss and density along the tails. We could have proceeded with diminishing the mass over time, based on some assumptions on the temporal behavior of this relation, however, we did not find this approach satisfying. In this way. we would have taken into account a temporal evolution of the mass, but not of the size of the cluster, adding a supplementary hypothesis to the problem. For these reasons, we decided to maintain the simplest approach. We emphasize that other groups have followed the same methodology, maintaining masses and sizes that remain constant over time \citep[see, e.g.,][]{palau19}.

        The summary tables giving both the current internal parameters of the clusters (total mass and half-mass radius), their astrometric quantities of relevance for this study and the line-of-sight velocities are publicly available\footnote{All data can be found here \url{https://people.smp.uq.edu.au/HolgerBaumgardt/globular}.}. We have made use of these tables for our work and we report them in a unique table in our paper for the sake of the completeness and self-consistency of the data used (see Table~\ref{TableIC}). 

    \subsection{Simulations of the tidal stripping process: Galactic potentials}\label{galmod}
    
    \begin{table*}
        \centering
        \caption{Parameters of the Galactic mass models adopted in this work. Masses are in units of $2.32\times10^7M_{\odot}$, distances given in units of kpc.
        \label{PII}}
        \tiny
        \begin{tabular}{  l c  c  c  c  c  c  c  c  c  c  c  c  c   c } \hline
        Parameters & $M_{bulge}$ &  $M_{bar}$ & $M_{thin}$ &  $M_{thick}$  & $M_{halo}$ & \ $b_{bulge}$  & $a_{bar}$ & $b_{bar}$ & $c_{bar}$ & $a_{thin}$ & $b_{thin}$  & $a_{thick}$ &  $b_{thick}$ & $a_{halo}$ \\  \hline \hline \\
        PI & 460.0 & 0.0 & 1700.0 & 1700.0  & 6000.0  & 0.3 & -- & -- & -- & 5.3000 & 0.25 & 2.6 & 0.8 & 14.0 \\  \hline    
            PII & 0.0 & 0.0 & 1600.0 & 1700.0  & 9000.0  & -- & -- & -- & -- & 4.8000 & 0.25 & 2.0 &  0.8 & 14.0 \\    \hline    
            PII-0.3-SLOW & 0 & 990.0 & 1120.0 & 1190.0  & 9000.0  & -- & 4.0 & 1. & 0.5 & 4.8000 &0.25 & 2.0 &  0.8 & 14.0 \\ \hline 
        \end{tabular} 
        \normalsize
    \end{table*}    

    As for the Galactic mass distribution, we make use of the two axisymmetric Galactic mass models presented in \citet{pouliasis17} and of an asymmetric mass model, containing a central stellar bar, and we present it here for the first time. We recall  the main properties of the two models of  \citet{pouliasis17} below and we describe the asymmetric Galactic mass model, presented here for the first time, in more detail.

    \subsubsection{Model I by \citet{pouliasis17}: An axisymmetric mass model for the Galaxy including a spherical bulge}
        Model I by \citet{pouliasis17} (abbreviated name: PI) consists of four components: two disks (thin and thick), both described by Miyamoto \& Nagai potentials, a dark matter halo, and a central bulge. Its total potential is:
        \begin{equation}
            \Phi_{tot}(R, z) = \Phi_{thin}(R, z) + \Phi_{thick}(R, z) + \Phi_{halo}(r)+  \Phi_{bulge}(r),
        \end{equation}
        with $r=\sqrt{R^2 + z^2}$,
        \begin{eqnarray}
            \Phi_{thin}(R,z)&=&\frac{-GM_{thin}}{\left(R^2+\left[a_{thin}+\sqrt{z^2+b_{thin}^2}\right]^2\right)^{1/2}},\\
            \Phi_{thick}(R,z)&=&\frac{-GM_{thick}}{\left(R^2+\left[a_{thick}+\sqrt{z^2+b_{thick}^2}\right]^2\right)^{1/2}},
        \end{eqnarray}
        \begin{equation}
            \begin{split}
                \Phi_{halo}(r)=&\frac{-GM_{halo}}{r}-\frac{M_{halo}}{1.02a_{halo}}\times\\
                & \Bigg[\frac{-1.02}{1+\left(\frac{r}{a_{halo}}\right)^{1.02}}+ln{(1+\left(\frac{r}{a_{halo}}\right)^{1.02})}\Bigg]_R^{100},
            \end{split}
        \end{equation}
        and 
        \begin{equation}
            \Phi_{bulge}(r) = -\frac{GM_{bulge}}{\sqrt{r^2+{b_{bulge}^2}}},
        \end{equation}
        where $M_{thin}, M_{thick}$, $M_{halo}$, and $M_{bulge}$ are the masses of the disks, halo, and bulge. Also, $a_{thin}, b_{thin},  a_{thick}, b_{thick}, a_{halo} ,b_{bulge}$ are the characteristic scale lengths of the thin and thick disks, the halo, and the central bulge, respectively (see Table~\ref{PII}).

        This model is a modification of the classical \citet{allen91} model, made to include also the presence of a thick disk. As it has been discussed in detail by \citet{pouliasis17}, the choice to include  a massive spheroid in this model, as well as in the original \citet{allen91} model, is dictated by the need to reproduce CO/HI-based velocity curves, as those provided by \citet{sofue12}, which show a rise and then a sudden decrease of the velocity curve in the inner Galactic regions ($R \le 2-3$~kpc). In an axisymmetric model, such a rise can be reproduced only if a central spheroidal component, with a typical mass greater than 10\% of that of the disk(s), is added. However, as shown by \citet{chemin15}, the central rise observed in the rotation of the molecular gas in the inner Galaxy may be an effect of non-circular motions generated by large-scale asymmetries such as the bar. Moreover, this feature is not reported in all the observational studies \citep[see, e.g., ][on which model PII is based]{reid14}. In other words, if we do not assume that the mass distribution of the inner Galaxy is axisymmetric, the need for a massive spheroidal component to reproduce velocity curves, such as those from \citet{sofue12}, no longer persists. In addition to that, in the last decade, a number of works  have shown that if a spheroidal bulge exists in the central regions of our Galaxy, it has to be small \citep[few percents of the mass of the disk at the most, see among others][]{shen10, kunder12, dimatteo15, gomez18}. All these arguments suggest to employ this model \citep[as well as all models including a massive central spheroid; see, e.g.,][]{irrgang13} with care when dealing with the central parts of the Galaxy. Since models with a massive central spheroid, however, are still used in the literature, we have included model PI here, as a term of comparison.    



    \subsubsection{Model II by \citet{pouliasis17}: An axisymmetric, bulge-less mass model for the Galaxy}

        Model II by \citet{pouliasis17} (abbreviated name: PII) consists of a spherical dark matter halo, with the same functional form adopted in the \citet{allen91} model, and of two disk components (a thin and a thick disk), with same functional form as PI. This model does not include any central spheroid (i.e., it is a bulge-less model) and thus its total potential is the sum of three components only:
        \begin{equation}
            \Phi_{tot}(R, z) = \Phi_{thin}(R, z) + \Phi_{thick}(R, z) + \Phi_{halo}(r),
        \end{equation}
        with the thin, thick disks, and dark matter halo having the same functional forms adopted in PI.

        As it has been shown in \citet{pouliasis17}, this model satisfies a number of observational constraints, such as the stellar density at the solar vicinity, thin- and thick-disk scale lengths and heights, the rotation curve as provided by \citet{reid14}  and the absolute value of the perpendicular force, $K_z$, as a function of distance to the Galactic centre \citep[see Sect.~2.5 in][]{pouliasis17}. As it is, however, an axisymmetric model, it fails to  accurately describe the inner few kpc of the Galaxy, where the stellar mass distribution has been shown to be asymmetric.     

    \subsubsection{Model~II with a massive, slowly rotating stellar bar}
        The third mass model (abbreviated name: PII-0.3-SLOW)  that we use in this paper is a version of PII by \citet{pouliasis17} modified to include a rotating stellar bar, whose mass has been assigned to be 30\% of the (thin+thick) disk mass of PII. We assume that the bar rotates with a constant pattern speed of $\Omega_{bar}=38\rm km~s^{-1}kpc^{-1}$ and that it is currently inclined of $25^\circ$ with respect to the Sun-Galactic center direction \citep[see][]{blandhawthorn16}. We model it as a triaxial distribution, whose gravitational potential is given by  \citet{long92}:
        \begin{equation}
            \Phi_{bar}(x,y,z)=\frac{GM_{bar}}{2a_{bar}}ln\left( \frac{x-a_{bar}+T_{-}}{x+a_{bar}+T_{+}}   \right),
        \end{equation}
        with $T_{\pm}=\left[ (a_{bar}\pm x)^2+ y^2 +  (b_{bar}+ \sqrt{c_{bar}^2+z^2})^2 \right]^{1/2}$
        and $a_{bar}, b_{bar}, c_{bar}$ the characteristic bar parameters. The total gravitational potential generated by this model thus takes the form:
        \begin{equation}
            \Phi_{tot}(x, y, z) = \Phi_{thin}(R, z) + \Phi_{thick}(R, z) + \Phi_{halo}(r)+  \Phi_{bar}(x, y, z).
        \end{equation}
        with all characteristic values given in Table~\ref{PII}. Practically, to include the bar, we reduced the mass of the disks in such a way to maintain the total stellar mass of this model as that of PII.  \citet{long92} provide the formulas of the accelerations generated by this triaxial distribution in the reference frame of the bar. To calculate and add them to the accelerations generated by the disks and dark matter halo, at each time step, we converted the positions of all particles  in the rotating, non-inertial reference frame of the bar, computed the corresponding accelerations on each particle, and then transformed these accelerations back in the inertial reference frame described in Sect.~\ref{numerical}. In this way, the accelerations due to the bar are added to those generated by the other terms of the Galactic mass distribution. 

        We emphasize that we do not consider this model as the best possible representation of the Galactic mass distribution, especially in the central region. It can, however, provide a first indication on how the inclusion of a rotating asymmetric component in the inner Galaxy can affect the globular cluster streams, near and far from the Galactic center. 

        Moreover, since the exact characteristics of the Milky Way bar are still subject to debate \citep[see, e.g.,][]{blandhawthorn16}, it is important to explore how varying the parameters adopted in this paper, such as the pattern speed, the mass, or the length of the bar, can affect the characteristics of the whole set of streams.  More complex shapes for the bar can also be explored, for example, by substituting the inner parts of the triaxial bar with a boxy-peanut-shaped morphology, which has been shown to characterize the inner Milky Way  \citep[see, e.g., ][]{wegg13, wegg15}. These topics are, however, beyond the scope of the present paper. In sum, given the uncertainties on the bar's physical extent and how it can change over the time span investigated here, its affect on the streams presented here are purely indicative. 


\section{Results}\label{results}
