\section{Astronomical units and scaling}
    \begin{itemize}
    \item units in astronomy are weird
    \item the bucking ham pi theorem 
    \end{itemize}



\section{Tstrippy}
    \begin{itemize}
        \item f2py, and why did we choose to use Fortran? 
        \item Bovy's guide for making a public python package
        \item migrating going from setuptools to meson
        \item a brief overview of how it works. 
        \item how I can either save orbits or snapshots
    \end{itemize}



\section{Numerical Errors and Schema} 

    \begin{verbatim}
    VIDEO: cluster_showing_scale_and_dynamical_time.mp4
    \end{verbatim}
    Display here some results where I tried to use the higher order leap frog... i.e. the Ruth method. 

    Make a qualitative argument against the 

    \begin{itemize}
        \item leapfrog is sympletic, preserve hamiltonian in the transform
        \item integrating at high resolution, downsampling for storage, and then interpolating if higher resolution is needed. 
        \item I tried to implement the Ruth-Ford
        \item I tried to implement the king model, which was slower and worse and not worth it 
    \end{itemize}


    \section{Computation time and Data Volume}
    \begin{itemize}
        \item there are practical restraints
        \item we choose to solve the restricted three body problem b/c
    \end{itemize}

\section{My work flow}
    \begin{itemize}
        \item Tstrippy is the production code
        \item then I have gcs which does data i/o
        \item then I have an analysis package that makes plots based on the final data products
    \end{itemize}

