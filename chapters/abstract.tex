The advent of \textit{Gaia} astrometry has enabled large-scale statistical and dynamical studies of stars and Galactic substructures. In particular, the Milky Way's system of $\sim 160$ globular clusters (GCs), each containing hundreds of thousands to millions of stars, can now be characterized in detail thanks to accurate measurements of distances, proper motions, radial velocities, masses, and radii. At the same time, the number of known stellar streams has grown rapidly, from about 60 at the start of this thesis to more than 120 today. Because stellar streams trace the orbits of their progenitors, they provide excellent probes of the Milky Way's gravitational potential and of perturbations from dark matter substructure. I developed the open-source and publicly available code \texttt{tstrippy}, which models the tidal stripping of globular cluster stars by solving the restricted three-body problem. Using this framework, we generated predictions for the distribution of tidal debris from the full Galactic GC population. We then carried out targeted simulations of Palomar~5, introducing perturbations from GC flybys. For each case, we quantified the survival and detectability of stream features such as gaps and epicyclic overdensities, and assessed the role of internal velocity dispersion in erasing or sustaining such signatures. We present the first global predictions of tidal debris from all Milky Way GCs \citep{2023A&A...673A..44F}. These simulations are publicly available and already being employed by the community. In a follow-up study \citep{2025A&A...699A.289F}, we quantified the frequency and range of GC encounters capable of perturbing Palomar~5, demonstrating that such interactions must be carefully accounted for to avoid false positives in searches for dark matter subhaloes. We also showed that increased internal velocity dispersion and the presence of epicyclic overdensities can significantly reduce the persistence of gap-like features, thereby lowering the sensitivity of streams to external perturbations. This work establishes the expected distribution of tidal streams from the Galactic GC population and quantifies the rate at which globular clusters can generate perturbations in the Palomar~5 stream. These results pave the way for future studies aimed at disentangling the different origins of stream perturbations, as well as for investigations of the internal dynamics, formation, and evolution of globular clusters, the assembly history of the Milky Way, and constraints on its gravitational potential in both the visible and dark matter components.



L'avènement de l'astrométrie de \textit{Gaia} a rendu possibles des études statistiques et dynamiques à grande échelle des étoiles et des sous-structures de la Galaxie. En particulier, le système de la Voie lactée, composé d'environ 160 amas globulaires (GCs - Globular Clusters), chacun contenant de plusieurs centaines de milliers à plusieurs millions d'étoiles, peut désormais être caractérisé en détail grâce à des mesures précises des distances, des mouvements propres, des vitesses radiales, des masses et des tailles. Parallèlement, le nombre de courants stellaires connus a augmenté rapidement, passant d'environ 60 au début de cette thèse à plus de 120 aujourd'hui. Comme les courants stellaires tracent les orbites de leurs progéniteurs, ils constituent d'excellentes sondes du potentiel gravitationnel de la Voie lactée et des perturbations dues à la sous-structure de matière noire. J'ai développé le code open source et librement accessible \texttt{tstrippy}, qui modélise le dépouillement tidal des étoiles d'amas globulaires en résolvant le problème restreint à trois corps. En utilisant ce cadre, nous avons produit des prédictions pour la distribution des débris tidaux de l'ensemble de la population galactique d'amas globulaires. Nous avons ensuite effectué des simulations ciblées de Palomar~5, en introduisant des perturbations liées aux passages rapprochés d'autres amas globulaires. Pour chaque cas, nous avons quantifié la survie et la détectabilité des structures du courant, telles que les lacunes, et évalué le rôle de la dispersion de vitesses interne ainsi que la présence de surdensités épicycliques, qui rendent plus difficile la survie des lacunes dans ces régions. Nous présentons les premières prédictions globales des débris tidaux de tous les amas globulaires de la Voie lactée \citep{2023A&A...673A..44F}. Ces simulations sont publiquement disponibles et déjà utilisées par la communauté. Dans une étude de suivi \citep{2025A&A...699A.289F}, nous avons quantifié la fréquence et l'amplitude des rencontres entre amas globulaires susceptibles de perturber Palomar~5, en démontrant que de telles interactions doivent être soigneusement prises en compte afin d'éviter de faux positifs dans les recherches de sous-halos de matière noire. Nous avons également montré qu'une dispersion de vitesses interne plus élevée et la présence de surdensités épicycliques peuvent réduire significativement la persistance de structures de type lacune, diminuant ainsi la sensibilité des courants stellaires aux perturbations externes. Ce travail établit la distribution attendue des courants stellaires issus de la population galactique d'amas globulaires et quantifie le taux auquel les amas globulaires peuvent générer des perturbations dans le courant de Palomar~5. Ces résultats ouvrent la voie à de futures études visant à démêler les différentes origines des perturbations des courants, ainsi qu'à l'exploration de la dynamique interne, de la formation et de l'évolution des amas globulaires, de l'histoire d'assemblage de la Voie lactée, et à la mise en contrainte de son potentiel gravitationnel dans ses composantes visibles et de matière noire.
