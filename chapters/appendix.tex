\addchapnonumber{Appendices}
\section{Additional Publications}
    \subsection*{Galactic Astronomy}
        \reference{\citep{2022A&A...664A..31C} Casamiquela, L., J. Olivares, Y. Tarricq, S. Ferrone, C. Soubiran, P. Jofre, et al. (2022). ``Unravelling UBC 274: A morphological, kinematical, and chemical analysis of a disrupting open cluster.'' A\&A 664, A31, A31. doi: 10.1051/0004-6361/202243658. \\~\\ \citep{2024ApJ...967...89I} Ibata, R., K. Malhan, W. Tenachi, A. Ardern-Arentsen, M. Bellazzini, P. Bianchini, et al. (2024). ``Charting the Galactic Acceleration Field. II. A Global Mass Model of the Milky Way from the STREAMFINDER Atlas of Stellar Streams Detected in Gaia DR3.'' ApJ 967.2, 89, p. 89. doi: 10.3847/1538-4357/ad382d}

        Regarding \citet{2022A&A...664A..31C}, Casamiquela reported the discovery of an open cluster in the Galactic disk and analyzed its spatial and chemical properties. They noted that the cluster was not spherically symmetric but instead showed a distorted shape. Using \texttt{tstrippy}'s predecessor, GCsTTs, we modeled this cluster and demonstrated that its observed deviation from spherical symmetry can be explained by tidal features. See their Figure 1.

        We have already discussed \citet{2024ApJ...967...89I} several times throughout this thesis, as this work reported the detection of many new stellar streams using Gaia~DR3, increasing the known total from roughly 60 to about 100. In addition, the paper developed a framework to correct for the known misalignment between a stream and its progenitor orbit \citep{2007ApJ...659.1212M,2013MNRAS.433.1813S}. The recovered progenitor orbits were then used in a Markov Chain Monte Carlo (MCMC) analysis to obtain best-fit parameters for the Milky Way's gravitational potential. This work is particularly exciting and opens many avenues for future investigation. For instance, the stream catalog itself is immensely valuable and can serve as a foundation for follow-up observations and analyses. Furthermore, as argued in \citet{2023A&A...678A.180T}, these streams could also be used to constrain the Galactic bar through MCMC modeling.

    \subsection*{Asteroid science}
        \reference{\citep{2023A&A...676A...5F} Ferrone, S., M. Delbo, C. Avdellidou, R. Melikyan, A. Morbidelli, K. Walsh, et al. (2023). ``Identification of a 4.3 billion year old asteroid family and planetesimal population in the Inner Main Belt.'' A\&A 676, A5, A5. doi: 10.1051/0004-6361/202245594.\\~\\ \citep{2024A&A...682A..64B} Bourdelle de Micas, J., S. Fornasier, M. Delbo, S. Ferrone, G. van Belle, P. Ochner, et al. (2024). ``Compositional characterization of a primordial S-type asteroid family of the inner main belt.'' A\&A 682, A64, A64. doi: 10.1051/0004-6361/202347391 }

        In \citet{2023A&A...676A...5F}, we reported the discovery of an ancient asteroid family in the main belt using the ``V-shape'' technique. Asteroid families are groups of bodies that share similar Keplerian orbital parameters—semi-major axis, eccentricity, and inclination. These bodies once belonged to the same parent body, but a catastrophic collision can produce a violent breakup, ejecting smaller fragments. The ejection velocities cause the fragments to acquire slightly different orbital elements. Over time, however, the Yarkovsky effect, a non-gravitational force, can cause an asteroid's semi-major axis to drift. The drift rate is inversely proportional to the asteroid's diameter, meaning smaller bodies drift more efficiently. Moreover, prograde rotators drift outward, while retrograde rotators drift inward. As a result, when plotting inverse diameter against semi-major axis, the distribution of family members takes the shape of a ``V''. The opening angle of the V correlates with the family's age, with wider shapes corresponding to older families. This technique has been successfully applied to identify other families \citep{2019A&A...624A..69D,2017Sci...357.1026D}.  

        An additional outcome of our study was that, once known asteroid families are removed from the main belt, the remaining bodies are likely primordial—objects that formed in the proto-planetary disk and were not significantly altered by subsequent collisional evolution. In \citet{2023A&A...676A...5F}, we provided a list of such candidate primordial asteroids. In \citet{2024A&A...682A..64B}, we carried out spectroscopic follow-up observations to characterize the taxonomic classes of the objects within the newly identified family.
