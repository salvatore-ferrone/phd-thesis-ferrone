\section{Conclusion}

\citet{2022A&A...664A..31C} Laia's paper that used the first version of the code.

\begin{itemize}
    \item I need to restate the importance of stellar streams 
    \item I need to recap the conclusions of this work 
    \item I need to restate the state of the art 
    \item I need to introduce the rest of the chapter, which are discussing prospects of future work, both in scientific questions and modeling techinques 
\end{itemize}

\section{Future scientific questions}
    \subsection{dark matter subhalos}
    \begin{itemize}
        \item Press-schecter
        \item Pierre's dark matter subhaloes
        \item Duncan's paper 
        \item Bayesian hierarchal models 
        \item Infer the parameters of the distribution function from dark matter subhaloes
    \end{itemize}

    \subsection{Time-varying bulk components of the Milky Way }

    \subsection{internal dynamics}
    \citet{2004AJ....127.2753D} presented a great work on Palomar~5 and found the parameters of a king model that best reproduce Palomar~5. Perhaps I can take form this study to have the best description of the tidal tails that corresponds to Palomar~5, and iterate over this to have the best description of of the tails as possible. 

    \subsection{mock observations}
        Knowing where the stars should 




    \section{Globular cluster formation and multiple stellar populations}
        What if we wanted to go further? Indeed at somepoint, this is just considered other work or future considerations, or other research questions entirely. 

        What if we wanted to further? this is certainly outside the scope of our work, but we could consider the globular cluster formation. 

        The complexity doesn't end here. For instance, we are unsure how globular clusters form. There are some mechanisms proposed (cite the extra galactic GC review). That review summarizes them as: (1), (2), (3). However, to completely understand them would involve understanding the chemical enrichment history of the universe. Starting from big-bang nucleosysntehsis, to first generation stars (POP III), and how these first stars chemically enrich the environment, and how the clusters form from said environment. \citet{2022A&A...668A.191C} studied cluster formation in environments that were enriched by just pop II stars or a mixture of pop III/II stars and showed that pop III stars were necessary to enrich the environment enough to reproduce current cluster qualtities. 

        However, from my literature review there does seem to be a gap linking the chemical environment that could be enriched from a combination of pop II/popIII stars and how cluster formation. At least we know clusters are at least second generation (pop II) stars, since the first stars (pop III) form massive and die young \citep{2002ApJ...571...30S}.

        Ideally, we would like to know about the stars within the cluster to then infer the environment that it must have been born in. Extra-galactic census of globular clusters show that there isa bi-modality in color, making people believe that there are at least two generations or two different mechanisms for forming them. 

        Yaaquib (2025) is showing that we can use streams to get at how the gravitational field changes due the LMC infall\dots maybe in the conclusion 

        \citet{2025arXiv250507491V} is looking at black hole formation inside of globular clusters that are being born. 

        \citet{2024A&A...681A..45L} Elena and Alessandra investigate a scenario that considers multiple populations in globular clusters where the second population is much less massive than the first, but evaporation and tidal stripping processes can mean that there present day contributions are similar to one another. 

        \citet{2025MNRAS.537.2342C} is looking at the dynamics of the different populations within 47 Tuc.

        \citet{2024MNRAS.529.2413U} said that there's evidence of multiple populations within the stellar streams. Something expected and now observationally confirmed 

        \citet{2023A&A...673A.152I} showed that within the last 5~gyr the potential was stable enough where the orbital solutions are pretty stable. 
    
        \citet{2006ARA&A..44..193B} made a review on extra galactic globular clusters. They discuss things like the observation evidence, the globular cluster mass function (all clusters), and find it to have an equal power spectrum, but the issue is that you don't know where to truncate the power law. maybe at lower end you can argue $10^4 M_\odot$ from two body relaxation and a tidal field. However, the upper limit is hard since you already don't expect that many, and the inclusion of a couple more high mass members can significantly alter the ``mean'' mass GC. They also note how a lot of cluster properties match the galactic properties. They talk about a bi-modal distribution. It really isn't that bimodal, there's a ton of overlap. There's a bluer one and a redder one. One of them might form during the collisions between galaxies while the other could be formed deeped in the potential wells. They also said that as of 2014 there is evidence that GCs are still forming. They also discuss the chemical abundance patterns within the clusters. It's complicated stuff, I wish I was better at stellar physics and knowing all the different sequences\dots
    
        \citet{2002ApJ...571...30S} proposed the of pair-unstable supernovae $\gamma + \gamma \rightarrow e^{+} + e^{-}$ to create the positrons and electrons, that reduces the radiation pressure, to then make the star implode then explode. This shows the expected yeild of some selected heavier elements as a function of the ZAMs. Pretty sick. 
    
        \citet{2006MNRAS.369..825S} proposed the initial mass function on pop III stars. I think this explains why the were expected to be soooo large and how the metals allow them to grow so much. I don't know what the  WMAP eletron scatter probe is. I think they're already talking about detectability. and saying that if a scenario produces many eave pop III stars, then it won't be detectable in the WMAP data. Perhaps pop III stars should do something to the electron scattering optical depth? What is a Larson initial mass function?  

        \citet{2025MNRAS.540.1235C} presents a cool scenario for investigating globular cluster formation during galaxy formation and mergers. They discuss the chemical environment the clusters form in as an enrichment process from POP III and POP II stars. I must read this more in detail. It is fascinating work. 

        Something about the chemical abundance patterns of the halo stars that can be consistent with the globular clusters \dots

        Schiavon (2017) showing that there exist many N-rich stars which have chemical abundance patterns similar to stars within the clusters all over the galactic field. 

        Fernández-Trincado, José G (2021) says that there are some metal rich cluster debris, which is interesting 

        Fernández-Trincado, J. G (2017) said they found 11 stars with the chemical abundance patterms of second general stars within globular cluster \dots 

\section{Future modeling techinques}

    \subsection{N-body codes}
    \citep{2012MNRAS.424..545N} showed the GPU's can replace GRAPE

    \citet{2018ComAC...5....2V} talks about a series of papers between a larger collaboration of people who specialize in collisional dynamics and who have performed a series of workshops together. The introduction stated that the collaboration wants to tackle many open questions regarding stellar clusters and build the necessary codes to interprete the future large quantity of data that was destined to come. It has now come since the review was 2018. An interesting point was that in general globular clusters are approximated as being orderless, i.e. isotropic but order does present itself within these stelalr systems. Another large problem is no one knows what a good set of initial conditiosn is. Unresolved binaries pose a problem because you can overestiamte the total mass of the system. If I talk about this review, I should probably discuss some of the results from the papers that is builds on or at least their techinques.

    The MODEST review led me to discover AMUSE, which is an framework for integrating various astrophysical codes for solving 4 types of problems: gravitational dynamics, radiative transfer, hydrodynamics, and stellar evolution. The codes are written by the community and are interfaced together with Amuse. The user end is python. I have spent some time reading the book, which is instructive and well written. Steve McMillian is one of the authors. The code has a large support on GitHub and is still being developped. I have had trouble trying to install the code. It seems as though their documentation is incoherrent. At one place, it said `pip' is the easiest way to install. It didn't work. In another place, I was instructed to install a zipped up tarball. The setup failed becuase it expected there to be a .git file in the directory. I successfully downloaded the code by cloneing the repository, despite the fact that this was not recommended. I can use some aspects of the code but not all of them. For instance, my memory tells me that about 80\% of the test suite passed, thus many scripts failed. This was when I only installed the frame-work, which was advised since installing the whole package is huge and unnecessary since I am not solving all astrophysical problems. However, I wasn't able to use one of the gravity solvers that was presented in the textbook `AstrophysicalRecipes The art of AMUSE'. The install still has some codes that failed for instance: amuse-adaptb, amuse-hermite-grx, amuse-mi6. However, I'm hoping that this isn't necessary. I want to educate myself and make some examples. 

    Installing other codes and figuring out their functionalities to me has never been trivial. This is similar to galpy when I tried to figure out particle spray method and got less than good results. Agama also confused me a bit. The main point is that for each package, at the end of the day I decided that it was easier and better if I solved the problem myself with my own code. Because, even with the other packages, I know that they can be used to solve other astrophysical problems and it wasn't clear to me how to make the codes solve my specific set of of the restricted three body problems in a potential with other perturbers flying around. 

    In this search, I also discovered another review called \textit{Computational methods for collisional stellar systems} by Spurzem and Kamlah 2023. It is also interesting and instructive. I found it insightful when they called NBody an industry. I think the story of GRAPE and Makino is really interesting, how he build dedicated hardware for the nbody problem which were great for 10 years but were quickly replaced by GPU technology. 

    \section{particle-spray methods}

    this could be really important if I want to come up with something \textit{fast} particularly if using 

    \citet{2012MNRAS.420.2700K} created the ``streak-line'' method, which in essence approximates streams as having the same orbital parameters as the progenitor. \citet{2014ApJ...795...94B} continued this and used it to quickly generate streams in a monte-carlo code to find the best halo parameters to reproduced observed streams to recover a galaxy's mass. They generate orbits on the fly giving the star particles the same parameters as the progenitor but offset. 

    The ``streak-line'' method is limited and ignores the dispersion in orbital parameters of the escaped stars. \citet{2011MNRAS.413.1852E} presented streams in action-angle variables. As stated by \citet{2015MNRAS.452..301F}, this is the most elegant description of stellar streams. Each stream star has an offset from the progenitor's hamiltonian which can be expressed with a second order taylor expansion. 

    \citet{2015MNRAS.452..301F} prescriptive stream generation model could be much faster than my simulations. Yes, this model is prescriptive but it the escape rate is tied to orbital properties, as is mine. It can operate at the on the dynamical time of the galaxy and not the dynamical time of the internal dynamics, which could greatly speed up simulations. \citet{2014ApJ...795...95B} did a good job of developing this in terms of the actions and angles. However, \citet{2015MNRAS.452..301F} provides an escape rate that is more motivated by the orbit of the progenitor, which is good since we know more particles escape at peri-center due to the scaling of tidal forces with $r^{-3}$. So the Fardal is pretty great. Perhaps adopting one of these methods for certain problems could be great. 

    The advantage of my method is that it is not prescriptive, so if we explore a region of the parameter space where the assumptions behind the prescriptions break down, we may be led astray. \dots but it's so fast \dots 







