In this chapter, I summarize the works undertaken in this thesis and reconnect it to the broader literature with proposing more avenues for investigation. 

\section{Summary}
    In Chapter 1, I outlined the importance of stellar streams from globular clusters, highlighting both their intrinsic scientific interest and their role as probes for a wide range of astrophysical questions. These include the chemical enrichment of the Universe, the assembly history of the Milky Way, pathways of black hole formation, and constraints on the global and local properties of the Galactic gravitational field—particularly in relation to the detection of dark matter.

    In Chapter 2, I presented the exact equations of motion used to model the formation of stellar streams from globular clusters, formulated as a restricted three-body problem. I then explained how these equations can be interpreted to describe the escape of stars through the Lagrange points under tidal forces, followed by their evolution via phase mixing to form the streams. I summarized the literature on the mechanisms by which perturbations can create gaps. I also discussed the limitations of our models: specifically, that without $N$-body simulations or analytical prescriptions for internal cluster dynamics, certain questions cannot be addressed.

    In Chapter 3, I introduced my simulation code, \texttt{tstrippy}, describing its numerical solution of the equations of motion, its performance in terms of accuracy and computation time, and a comparison of two numerical schemes. I explained how the code is written in Fortran and interfaced with Python via \texttt{f2py}, and provided a minimal, {\tiny non}-working example illustrating the workflow from the user's perspective.

    In Chapter 4, I presented \citet{2023A&A...673A..44F}, in which we simulated the expected tidal debris from the entire globular cluster catalogue. These simulations have been made publicly available and are now used by the community, including in searches for additional tidal debris beyond some of the clusters in our sample \citep{2025arXiv250705590K,2025ApJ...988...39W}.
    
    In Chapter 5, I present \citet{2025A&A...699A.289F}, where we investigated how the Palomar~5 stellar stream responds to the granularity of the Milky Way's gravitational field—specifically, the presence of other globular clusters. We showed that the internal dynamics of globular clusters can strongly influence the morphology of their stellar streams. In particular, gap formation from the fly-by of a massive perturber requires very specific stream conditions: regions close to the progenitor, where the morphology is still dominated by epicyclic overdensities, are unfavorable for gap formation, as different stellar packets respond differently to the perturbation and subsequently drift out of phase. This is a new contribution, as previous studies primarily considered the effects of random motions, such as velocity dispersion.

\section{Prospectives}
    One of the main strengths of this work is its focus on Milky Way streams and globular clusters. While idealised toy models are invaluable for testing new ideas and exploring a wide range of theoretical phenomena, our emphasis on the Milky Way allows us to ask more grounded questions: how often should these phenomena occur in our Galaxy, with what significance, and how do they connect directly to what we observe? By anchoring our models to the Milky Way, we can close the loop between theory and data.

    The ubiquity and sensitivity of stellar streams make them exceptional probes for inferring the properties of various mass distributions. This opens multiple avenues for future investigation. Some phenomena still require detailed numerical forward modelling—either to understand the underlying physical mechanisms or to obtain robust predictions for their expected occurrence rates. In other cases, where the theoretical picture is mature, we can turn directly to observations and begin the task of inference.
    
    \subsection{Dark Matter Subhalos}
        The prospect of detecting dark matter (DM) on scales smaller than those accessible in the extragalactic context is particularly compelling. Our goal is to perform a study similar to that in Chapter~5, but rather than limiting ourselves to a population of Galactic globular clusters, we will also include a $\Lambda$CDM-motivated population of dark matter subhalos (Boldrini et al., in prep). Figure~\ref{fig:mollweide-density-with-haloes.png} shows an example of such a population from a cosmological simulation of a Milky Way analogue. \citet{2024arXiv241213144A} recently carried out a similar analysis for the GD-1 stream and found that, over the last few billion years, GD-1 has on average one detectable gap from a DM subhalo encounter. \citet{2025arXiv250207781L} adopted this rate of $\sim1$ gap per stream to discuss the detectability of gaps in $\sim 50$ known Milky Way streams.

        \begin{figure}
            \includegraphics[width=\linewidth]{images/mollweide-density-with-haloes.png}
            \caption[Plausible $\Lambda$CDM dark matter subhalo population as seen from the Sun]{Plausible $\Lambda$CDM dark matter subhalo population as seen from the Sun. The subhalos are modelled as Plummer spheres, with their light integrated along the line of sight from the Sun and scaled according to the total halo mass. Overlaid is the integrated surface light density profile of the Milky Way's stellar disk, modelled with a Miyamoto--Nagai potential.}
            \label{fig:mollweide-density-with-haloes.png}
        \end{figure}

        There is considerable modelling work on gap formation in stellar streams. While we aim to make theoretical predictions for the expected number of gaps, these must be set in the context of a realistic Milky Way potential that includes known time-dependent and non-axisymmetric features. To date, most studies have examined either subhalo encounters in isolation \citep{2013ApJ...775...90C,2015MNRAS.450.1136E,2016MNRAS.463..102E,2016MNRAS.457.3817S,2024arXiv241213144A,2025arXiv250207781L} or the effect of the Galactic bar alone \citep{2016MNRAS.460..497H,2016ApJ...824..104P,2017NatAs...1..633P,2023A&A...678A.180T}. However, it is known that perturbations from baryonic structures can also generate gap-like signals, potentially mimicking the effects of DM subhalos \citep{2020ApJ...891..161I}. To correctly frame the inference problem, the false positive rate must therefore be quantified and calibrated.

        Ultimately, the inversion of this problem is a case study in Bayesian hierarchical modelling \citep{2020sdmm.book.....I}. Each gap detection is underdetermined \citep{2015MNRAS.450.1136E}: even in the absence of measurement uncertainties, a single gap cannot uniquely constrain the perturber's mass, size, time of impact, and relative velocity simultaneously. Nevertheless, with a sufficiently large sample of gaps and posterior distributions for each encounter, we can begin to infer statistical properties of the DM subhalo population.

        This is a challenging problem. We must account for other astrophysical processes that can erase gap signatures or create false positives, quantify their rates, and embed them in a hierarchical inference framework. The application of such an analysis to the Milky Way will require extremely high-quality data, which may be achievable with future LSST observations and the final Gaia data releases.

        As highlighted in Chapter~5, the coherence of a stream is crucial for gap survival, and this must be modelled accurately to avoid overestimating gap counts. In order to obtain proper predictions, we must accurate model stream generation. While full $N$-body simulations would be the most physically accurate way to model internal cluster dynamics and their mapping into streams, they are computationally prohibitive for the parameter space we must explore. Capturing variations in the Galaxy's potential, cluster internal dynamics, orbital initial conditions, DM subhalo populations, and the properties of multiple streams could require computing tens of thousands of stream realisations. $N$-body simulations have long been at the avant-garde of computational astrophysics, often driving innovations in both hardware and software. A notable example is the GRAPE (GRAvity PipE) series of special-purpose computers developed for gravitational $N$-body problems \citep{1991PASJ...43..841F,1997ApJ...480..432M}. For over a decade, GRAPE systems enabled simulations that would have been prohibitively slow on general-purpose hardware, pushing the limits of the field. Eventually, advances in general-purpose graphics processing units (GPUs) offered comparable performance with greater flexibility, leading to the widespread adoption of GPU-accelerated $N$-body codes \citep{2012MNRAS.424..545N,2015MNRAS.450.4070W}. Despite these hardware revolutions, the computational cost of the large ensembles of simulations required for our purposes remains high.

        Several methods have been developed to avoid the need for full $N$-body modelling. The ``streak-line'' method \citep{2012MNRAS.420.2700K} approximates streams as having the same orbital parameters as their progenitor. The action-angle formalism of \citet{2011MNRAS.413.1852E} describes each stream star as having a small offset from the progenitor's Hamiltonian, expressible via a second-order Taylor expansion. \citet{2014ApJ...795...95B} used this to develop the ``particle-spray'' method, which improves upon the streak-line model by introducing velocity dispersion into the streams. As noted by \citet{2015MNRAS.452..301F}, this is one of the most elegant stream modelling approaches to date.

        \citet{2015MNRAS.452..301F} also developed a prescriptive stream-generative model, fitting analytic functions to escape rates measured from $N$-body simulations. This method operates on the dynamical timescale of the Galaxy rather than the cluster's internal timescale, greatly speeding up computations.  

        The internal dynamics of globular clusters involve a rich range of processes \citep{1997A&ARv...8....1M}. While particle-spray and semi-analytic methods can be extremely efficient, extrapolating beyond the regime covered by their $N$-body calibrations can be risky. However, machine learning techniques \citep{2023ApJ...959...99T} offer a promising way to emulate $N$-body simulations without assuming a specific parametric form, providing both flexibility and speed.

        In summary, with upcoming improvements in both data quality and modelling techniques, there is strong potential to make significant progress in this field. Depending on computational constraints, I may either adopt an existing stream simulation method in place of \texttt{tstrippy} or implement a particle-spray approach within it.

    \subsection{The bulk gravitational field of the MW}
        At the onset of this thesis, our aim was to investigate large-scale features of the Milky Way such as spiral arms, the Galactic bar, giant molecular clouds, and satellite galaxies. Each of these components represents a potential avenue for further study.  

        In particular, we are interested in the role of the Galactic bar. Previous works have shown that the bar can induce chaotic orbits, causing stellar streams to ``fan out'' more than they otherwise would \citep{2016ApJ...824..104P,2020ApJ...889...70B}. Our preliminary investigation, based on the simulations of \citet{2023A&A...673A..44F}, revealed several intriguing effects. As noted by other authors \citep[see Fig.~9 of][]{2025NewAR.10001713B}, the bar can generate large underdensities along streams. Our own simulations suggest additional signatures beyond this.  

        The total mass, orientation, and length of the bar have been constrained in previous studies, though significant uncertainties remain. Most notably, the bar pattern speed (its angular velocity) still has two plausible regimes: slow or fast rotation \citep{2015MNRAS.450.4050W,2016ARA&A..54..529B,2023MNRAS.520.4779L,2024MNRAS.528.3576V}.  

        We are conducting new experiments to explore whether stellar streams can help break some of these degeneracies. As an example, Fig.~\ref{fig:pal5_with_bar} shows simulations of the Palomar~5 stream for a range of bar pattern speeds. In some cases, the bar splits the stream and creates prominent gaps; in others, it shifts part of the stream off its typical track; in yet others, it suppresses the stream's growth entirely, keeping debris bound near the progenitor cluster.  

        \begin{verbatim}
            VIDEO: Pal5_longmurali_5000_monte_carlo_002_white_bg.mp4
        \end{verbatim}

        \begin{figure}
            \centering
            \begin{tabular}{ccc}
                \includegraphics[width=.32\linewidth]{images/frame_0002.png}&
                \includegraphics[width=.32\linewidth]{images/frame_0004.png}&
                \includegraphics[width=.32\linewidth]{images/frame_0008.png}\\
                
                \includegraphics[width=.32\linewidth]{images/frame_0016.png}&
                \includegraphics[width=.32\linewidth]{images/frame_0019.png}&
                \includegraphics[width=.32\linewidth]{images/frame_0023.png}\\
                
                \includegraphics[width=.32\linewidth]{images/frame_0038.png}&
                \includegraphics[width=.32\linewidth]{images/frame_0048.png}&
                \includegraphics[width=.32\linewidth]{images/frame_0062.png}\\
                
                \includegraphics[width=.32\linewidth]{images/frame_0065.png}&
                \includegraphics[width=.32\linewidth]{images/frame_0066.png}&
                \includegraphics[width=.32\linewidth]{images/frame_0104.png}\\
            \end{tabular}
            \caption[The presence of a stellar bar with different rotational speeds affecting the Palomar~5 stream]{Simulations of the Palomar~5 stream using \texttt{tstrippy} with 5000 particles. The Galactic potential follows \citet{2017A&A...598A..66P} with a bar model from \citet{1997MNRAS.291..717M}, as described in Chapter~4. The initial conditions are identical in all runs, but the bar pattern speed is varied between 25 and 61~km\,s$^{-1}$\,kpc$^{-1}$ over 150 samples. The animated version of this figure is available in the online thesis.}
            \label{fig:pal5_with_bar}
        \end{figure}

        The results in Fig.~\ref{fig:pal5_with_bar} raise several questions. There is room to understand the underlying mechanisms theoretically, and to identify whether any Milky Way globular clusters or streams occupy one of these ``special'' resonances. Such signatures could occur only for a restricted range of bar pattern speeds (or other bar parameters), making them valuable dynamical probes. This is especially important because most existing models of the Milky Way potential, such as that of \citet{2024ApJ...967...89I}, are static and axisymmetric, omitting the bar due to its complexity. If these resonances can be robustly identified in observed streams, they may offer a new route to constraining the Galactic bar.

    \subsection{Multiple-stellar populations, Stellar evolution, and globular cluster formation and internal dynamics, stellar populations}
        Globular clusters are peculiar objects. In addition to being massive star clusters, they exhibit distinctive chemical abundance patterns compared to typical Galactic field stars \citep{2012A&ARv..20...50G,2018ARA&A..56...83B,2019A&ARv..27....8G}. Not only this, but some globular clusters have Multiple Stellar Populations. This are often understood as being generational. For some clusters, most stars within a cluster are thought to be second-generation (Gen II), implying that the original system was much more massive at formation. However, no single formation and evolution scenario fully reproduces the observed abundance patterns.
        
        Some studies have explored dynamical pathways for producing a second generation. For example, \citet{2024A&A...681A..45L} investigated a scenario in which a gaseous disk within a globular cluster forms a second generation of stars. This disk later relaxes into a spherical configuration, and the final cluster mass ends up about 20 times lower than the initial mass—resulting in roughly equal numbers of Gen I and Gen II stars. In this model, material lost to a stellar stream is predominantly first-generation.
        
        While theoretical models link stellar evolution to cluster dynamics, observational studies are now providing kinematic evidence for multiple populations. For instance, \citet{2025MNRAS.537.2342C} measured kinematic differences between two stellar generations in 47 Tuc.        
        
        The chemical composition of globular cluster stars is also key to tracing the Milky Way's assembly history. These patterns are distinct from those of Galactic disk stars, yet halo stars often share similar abundances with globular clusters \citep{2016ApJ...825..146M,2017MNRAS.465..501S}. Ex-cluster stars are not confined to the halo: \citet{2021ApJ...918L..37F} identified relatively metal-rich cluster debris (still below solar metallicity) in the Galactic disk, while \citet{2017ApJ...846L...2F} found 11 disk stars with the chemical signatures of Gen II cluster stars.

        As data quality improves, it is becoming possible to study multiple stellar populations not only in clusters but also in their tidal debris. \citet{2022MNRAS.510.3727P} reported nitrogen-rich stars in the tidal tails of Palomar 5—such stars are characteristic of halo populations. Likewise, \citet{2024MNRAS.529.2413U} presented evidence for multiple populations within a stellar stream unassociated with any known cluster. Using a sufficiently large sample and detailed chemical abundances, they identified the stream as globular cluster debris and even found a star with the chemical pattern of a Gen II member.
        
        In the coming years, as chemical abundance measurements are obtained for more stars in tidal tails, we may be able to map the spatial distribution of Gen I and Gen II stars in streams. Their relative kinematics could offer a direct probe of the progenitor cluster's internal dynamical evolution.

\section{final word}
    With all this said and done, there is more to do. 
