\noindent In this chapter, I summarize the theoretical background necessary to understand how we model the disruption of globular clusters to form stellar streams. My goals are threefold:
\begin{enumerate}
    \item To describe the modeling process in detail;
    \item To describe the physical mechanisms that govern this process;
    \item To acknowledge other physical ingredients that are relevant to globular cluster evolution but were deliberately omitted in our modeling, and to discuss the consequences of these omissions. 
\end{enumerate}
This chapter serves multiple roles: half learning resource for a Master's-level student interested in modeling stellar streams; half documentation and reflection for myself, consolidating what I've learned; half annotated bibliography pointing to more complete references; and half justification for the methodological choices made in this project. That's four halves—because none of them is fully complete on its own. Still, together they form something useful. I tried to write the chapter I wish I could hand to myself in 2022 to jump-start this research.

Conceptually, the chapter is divided into three sections that mirror these motivations:

\begin{itemize}
    \item \textbf{Explicit physics}: These are the fundamental equations that underpin our models;
    \item \textbf{Implicit physics}: Often, while the governing equations can be written exactly, they are too terse to provide intuition. This section interprets simplified versions of the equations to explain qualitatively what's happening in the simulations;
    \item \textbf{Ignored physics}: Here I highlight physical effects excluded from our models, many of which appear in the broader literature. I discuss why we excluded them and how that limits our results.
\end{itemize}


\section{The Explicit Physics}

    Globular clusters are dense stellar systems containing hundreds of thousands to millions of stars. Each star orbits within the gravitational potential of the cluster, while the cluster itself orbits the center of mass of the host galaxy. The galaxy, in turn, is composed of billions of stars, along with gas and dark matter, all contributing to its gravitational potential. A natural modeling approach is to treat the stars as point masses and to represent the gas and dark matter as continuous density distributions. This strategy would allow the construction of the Galactic gravitational field, potentially through a combination of hydrodynamical simulations for the gas and $N$-body simulations for the stars and dark matter. However, the feasibility of such an approach must be carefully considered.

    The scaling of computation time is often analyzed using Big-O notation. For direct $N$-body simulations, the computational cost scales as $\mathcal{O}(N^2)$, since the gravitational force on each particle must be computed from every other particle. For the Milky Way, with approximately $10^{11}$ stars, we need to compute $5 \times 10^{21}$ pairwise distances per time step.

    Let us consider the practicality of performing such a computation. \textit{Frontier} is one of the most powerful modern supercomputers and operates at approximately one exaflop, or $10^{18}$ FLOPS (Floating Point Operations Per Second) \citep{atchley2023frontier}. Assuming—conservatively—that each pairwise computation requires a single FLOP, a full $N$-body computation of the Galaxy at one time step would still require several hours. While this may seem manageable, a single time step is insufficient for modeling the long-term evolution of the system, which could require millions of steps or more. Dedicating an entire exascale machine to such a task is therefore a considerable demand.

    Moreover, the cost of power is non-negligible. At an estimated rate of $\sim 0.10~\mathrm{USD}/\mathrm{kWh}$ \citep{table20245}, and with Frontier consuming roughly 21 MW, a single 6-hour computation would cost approximately 12,600~USD. Running the system for a full day would amount to about 50,000~USD, or ~42,000 EUR \citep{ECB_USD_EUR_2025}. This is nearly twice the annual salary of a PhD student in France \citep{MESR_financement_doctoral}.

    Throughout this thesis, I have access to the Pairs observatoy's super computer \citep{DIO_OBSPM}, which currently has about 85 nodes. However, as a humorous aside, consider attempting this computation on a personal research laptop, such as a MacBook Air with an M2 processor. Its estimated speed of $\sim$0.1 TFLOPS (i.e., $10^{11}$ FLOPS) \citep{hubner2025apple} is approximately 10 million times slower than Frontier. A single time step would thus take over 1,500 years.

    Clearly, we require a more tractable modeling approach for studying the evolution of globular clusters and the formation of stellar streams. Fortunately, it is well justified to approximate all the stars in the galaxy as a smooth background density field rather than as a collection of point masses. This argument, presented in Section 1.2 of \citet{2008gady.book.....B}, shows that the inaccuracy of such a model becomes significant only on timescales far exceeding the age of the Universe. This is known as two-body relaxation. This process compares two hypothetical orbits for the same star. One within a smooth medium, and a second within a medium composed of point masses. The two-body relaxation time is how long it takes for these two trajectories to deviate significantly. We revist this in Section~XXX.
        
    In globular clusters, where stars are densely packed into a relatively small volume, stellar encounters become significant over the system's lifetime. According to Chapter 5.1 of \citet{bovy_inprep}, the relaxation timescale within globular clusters is typically $\sim1$~Gyr. To contextualize this, we must consider the orbital period of a typical star within a cluster, and the age of a cluster. First, the characteristic time for a star to traverse the system—is roughly 1 Myr \citep[which can be estimated as the size of a system divided by its internal velocity dispersion][]{2018MNRAS.478.1520B}. Secondly, globular clusters are old systems, with ages ranging from several billion to over ten billion years \citep{2013ApJ...775..134V}. These timescales establish the following hierarchy for globular clusters:
    \begin{equation*}
        t_\mathrm{cross} << t_\mathrm{relax} < t_\mathrm{age}
    \end{equation*}

    This has two major consequences. First, because the crossing time is much shorter than the relaxation time, the cluster can be considered in dynamical equilibrium at any given moment. The stars thus follow orbits determined by a smooth potential. Second, since the age of the cluster exceeds the relaxation time, cumulative stellar encounters (i.e., ``collisions'') significantly influence the system's long-term evolution.

    Despite the importance of collisional effects for the long-term evolution of globular clusters, explicitly modeling them via direct $N$-body simulations remains prohibitively expensive for our purposes. We return to this point in more detail in Section~\ref{sec:ignoredphysics}. In this thesis, we therefore adopt the collisionless approximation for both the Galactic background and the globular cluster. While this is well-justified for the Galaxy, treating the cluster as collisionless is a simplification that limits the generality of our results. Nonetheless, our goal is to accurately model the stellar streams, and the internal dynamics of the globular clusters are beyond the scope of this work.

    To make the problem computationally tractable, we model the stars as test particles—massless bodies that feel the gravitational potential but do not contribute to it. The Galaxy and the globular cluster are each represented as smooth, time-independent density distributions. In this approximation, the stars do not interact with one another or with the cluster, and their trajectories are determined solely by the external potentials. This setup is a version of the \textit{restricted three-body problem}, in which the Galaxy acts as the primary body (fixed at the origin), the globular cluster as the secondary (modeled by its center of mass), and the star as the tertiary object. This is shown in Fig.~\ref{fig:restricted_three_body_set_up}.

    \begin{figure}
        \centering
        \includegraphics[width=.8\linewidth]{images/restricted_three_body_set_up.png}
        \caption{Schematic illustration of the restricted three-body setup. The system is reduced to three interacting components: the Galaxy (primary), the globular cluster (secondary), and a single star (tertiary). Red vectors indicate forces that are neglected, while the black vectors are those that we model.}
        \label{fig:restricted_three_body_set_up}
    \end{figure}

    This formulation dramatically simplifies the dynamics: the full $N$-body problem is approximated by \(N\) independent realizations of the restricted three-body problem. This reduction allows us to efficiently simulate the formation and evolution of tidal features without incurring the computational cost of tracking all mutual interactions among stars. This approximation has been shown to work very well for certain stellar streams and this thesis is not the first time such methods have been employed. For instance, consider the work of \citet{2012A&A...546L...7M}, who modeled the evolution of the Palomar~5's stellar stream with both an $N$-body code as well as the restricted~three body problem. The differences between the two models for reproducing many characteristics of the stream are minimal. 

    In the remainder of this section, I present the theoretical framework underlying the simulations. At their core, the simulations involve solving \(N\) sets of six coupled ordinary differential equations—one set for each star particle. Section~\ref{subsec:myEquationsOfMotion} introduces the equations of motion. Section~\ref{subsec:gravfield} describes the gravitational field used to compute the forces acting on the particles. Finally, Section~\ref{subsec:initialconditions} details the procedure used to generate initial conditions that faithfully represent a globular cluster.


    
    \subsection{Equations of Motion} \label{subsec:myEquationsOfMotion}

        In galactic dynamics, we can well describe the motion of stars through Poisson's equation, which extends Newton's law of gravity from point masses to smooth density distributions: 
        \begin{equation} 
            \nabla^2 \Phi = 4\pi G \rho,
        \end{equation}
        where $\rho$ is the mass density and $\Phi$ is the gravitational potential. We use Newtonian mechanics throughout this work because $\Phi << c^2$ in all relevant regimes, meaning General Relativity is not required \citep[see Appendix C of][]{bovy_inprep}. Poisson's equation assumes Newton's postulate that gravitational forces act instantaneously and that masses attract each other along the line connecting them. \footnote{Galaxies span millions of light-years, so how can Newton's assumption of instantaneous gravity be valid, given that gravitational waves travel at the speed of light? As shown by \citet{2000PhLA..267...81C}, this is well understood in General Relativity: the gravitational field encodes not only the position but also the motion of massive bodies. This ensures that the resulting spacetime curvature produces a force equivalent to attraction toward the body's instantaneous position, rather than where it was when the gravitational influence was emitted.}

        Instead of writing down the gravitational forces through Newton's second law, I will write down Hamilton's equations derived from the variational principle of Lagrangian mechanics. The full interacting system includes the kinetic energy of the globular cluster and the star particle, as well as three gravitational potential energy terms: the cluster-galaxy interaction, the particle-galaxy interaction, and the particle-cluster interaction:
        \begin{equation}
            \mathcal{L} = \frac{1}{2} M_{\rm gc} \dot{\mathbf{R}}_{\rm gc}^2 
                        + \frac{1}{2} m \dot{\mathbf{r}}^2 
                        - M_{\rm gc} \Phi_{\rm gal}(\mathbf{R}_{\rm gc}) 
                        - m \Phi_{\rm gal}(\mathbf{r}) 
                        - m \Phi_{\rm gc}(\mathbf{r} - \mathbf{R}_{\rm gc}).
        \end{equation}  
        \textbf{annotate the equations}. We decouple the equations of motion of the globular cluster and the star. The justification for this approximation is evident when we normalize the Lagrangian by the cluster mass, \( M_{\rm gc} \). This yields:

        \begin{equation}
            \frac{\mathcal{L}}{M_{\rm gc}} = \frac{1}{2} \dot{\mathbf{R}}_{\rm gc}^2 
                                        - \Phi_{\rm gal}(\mathbf{R}_{\rm gc}) 
                                        + \underbrace{\frac{m}{M_{\rm gc}} \left[ \frac{1}{2} \dot{\mathbf{r}}^2 
                                        - \Phi_{\rm gal}(\mathbf{r}) 
                                        - \Phi_{\rm gc}(\mathbf{r} - \mathbf{R}_{\rm gc}) \right]}_{\text{negligible correction to GC's motion}}
        \end{equation}

        In the limit where \( m \ll M_{\rm gc} \), the terms in brackets become negligible, and the star's motion has no influence on the cluster. The Lagrangian for the cluster's orbit thus becomes:

        \begin{equation}
            \frac{\mathcal{L}_{\mathrm{gc}}}{M_{\mathrm{gc}}} = \frac{1}{2} \dot{\mathbf{R}}_{\rm gc}^2 
                                - \Phi_{\rm gal}(\mathbf{R}_{\rm gc})
        \end{equation}

        Switching to the star's perspective, we normalize the Lagrangian by the particle mass \( m \), obtaining:

        \begin{equation}
            \frac{\mathcal{L}_{\rm star}}{m} = \frac{1}{2} \dot{\mathbf{r}}^2 
                                - \Phi_{\rm gal}(\mathbf{r}) 
                                - \Phi_{\rm gc}(\mathbf{r} - \mathbf{R}_{\rm gc}(t))
        \end{equation}

        Here, the cluster's influence on the particle is retained through its time-dependent position \( \mathbf{R}_{\rm gc}(t) \). This influence becomes important when the gravitational forces from the cluster and the galaxy on the particle are comparable. For a quick estimate, we treat both the galaxy and the cluster as point masses. The two forces become comparable when the particle is sufficiently close to the cluster, which occurs when: $|\mathbf{r} - \mathbf{R}_\mathrm{GC}| < \sqrt{\frac{M_{\mathrm{gc}}}{M_{\mathrm{ gal}}}} |\mathbf{r}|.$ In other words, the cluster's gravitational field dominates over the galaxy's on sufficiently small scales around the cluster — as expected. For a quick sanity check: a typical globular cluster may have a mass of \(10^5\, M_\odot\), while the galaxy may be around \(10^{11}\, M_\odot\). If the cluster is a few kiloparsecs from the galactic center, then the cluster's influence dominates within a region of order a few parsecs — which checks out. A better limit is the tidal radius and is presented in the next section.

        Next, working in the Hamiltonian formalism often provides deeper insight into the physics of the system. Additionally, Hamiltonian mechanics reduces the equations of motion to a set of \(2N\) first-order differential equations, rather than \(N\) second-order ones, which is more convenient for computational integration.

        The Hamiltonian can then be obtained via a Legendre transform: \(\mathcal{H} = \sum p_i \dot{q}_i - \mathcal{L},\) where $q$ is a generic position coordinate, $p$ is its conjugate momentum, while $i$ indicates the coordinate of interest. If we use the \textit{specific} Lagrangian—normalized by the mass of the body of interest—then, in Cartesian coordinates, the conjugate momenta reduce to the velocities: $p_i=\frac{\partial \mathcal{L}}{\partial q} \rightarrow p_i = v_i$. Another useful result comes from Noether's theorem: if a coordinate does not explicitly appear in the Lagrangian, its conjugate momentum is conserved. All galactic potentials considered in our simulation are axis-symmetric meaning they have cylindrical symmetry (see next section). Since the Lagrangian does not depend on the azimuthal angle \( \theta \) in the \(x\text{-}y\) plane for the cluster's motion, the conservation of the \(z\)-component of angular momentum is conserved: $p_\theta = L_z = \frac{\partial \mathcal{L}}{\partial \dot{\theta}} = R^2 \dot{\theta} = \mathrm{constant.}$

        This conservation generally does not hold for star particles within the globular cluster, as their dynamics are significantly influenced by the cluster's gravitational potential. However, once a star-particle escapes the cluster the conservation of the z-component of the angular momentum is recovered in the regime where their motion becomes dominated by the galaxy's potential. 

        Hamilton's equations thus provide the time evolution of the momenta and the positions. In cartesian coordaintes, the equations of motion for the cluster are: 

        \begin{align}\label{eq:equations_of_motion_cluster}
            \dot{p}_{\mathrm{gc},x} &= -\frac{\partial \Phi_\mathrm{gal}\left(\mathbf{R}_{\mathrm{gc}}\right)}{\partial x} \\
            \dot{p}_{\mathrm{gc},y} &= -\frac{\partial \Phi_\mathrm{gal}\left(\mathbf{R}_{\mathrm{gc}}\right)}{\partial y} \\
            \dot{p}_{\mathrm{gc},z} &= -\frac{\partial \Phi_\mathrm{gal}\left(\mathbf{R}_{\mathrm{gc}}\right)}{\partial z} \\
            \dot{x}_{\mathrm{gc}} &= p_{\mathrm{gc},x} \\
            \dot{y}_{\mathrm{gc}} &= p_{\mathrm{gc},y} \\
            \dot{z}_{\mathrm{gc}} &= p_{\mathrm{gc},z}.
        \end{align}
        For the star particle, the equations of motion become: 
        \begin{align}\label{eq:equations_of_motion_stream}
            \dot{p}_{x} &= -\frac{\partial \Phi_\mathrm{gal}\left(\mathbf{r}\right)}{\partial x} - \frac{\partial \Phi_\mathrm{gc}\left(\mathbf{r} - \mathbf{R}_{\mathrm{gc}}(t)\right)}{\partial x}\\
            \dot{p}_{y} &= -\frac{\partial \Phi_\mathrm{gal}\left(\mathbf{r}\right)}{\partial y} - \frac{\partial \Phi_\mathrm{gc}\left(\mathbf{r} - \mathbf{R}_{\mathrm{gc}}(t)\right)}{\partial y}\\
            \dot{p}_{z} &= -\frac{\partial \Phi_\mathrm{gal}\left(\mathbf{r}\right)}{\partial z} - \frac{\partial \Phi_\mathrm{gc}\left(\mathbf{r} - \mathbf{R}_{\mathrm{gc}}(t)\right)}{\partial z}\\
            \dot{x} &= p_x \\
            \dot{y} &= p_y \\
            \dot{z} &= p_z,
        \end{align} 
        where $\mathbf{r} = \left(x,y,z\right)$. Note that equation~\ref{eq:equations_of_motion_stream} describes the motion of a single particle. In general, if we use 100,000 particles, the motion of each is described with this same equation, but has different initial conditions. 
        
        In Chapter~5, we modify Equation~\ref{eq:equations_of_motion_cluster} to compute the $N$~body forces between all the clusters. While we created these equations to avoid $N$-body computations, it is not too computationally expensive to implement this for the cluster's themselves since we only have about 160 clusters. These equations are modified by:  
        \begin{align}\label{eq:equations_of_motion_cluster_n_body}
            \dot{p}_{x,i} = - \frac{\partial \Phi_\mathrm{gal}}{\partial x} - \sum_{j\neq i}^N \left[\frac{\partial \Phi_\mathrm{gc}\left(\mathbf{R}_i-\mathbf{R}_j|M_j,b_j\right)}{\partial x}\right], \\ 
            \dot{p}_{y,i} = - \frac{\partial \Phi_\mathrm{gal}}{\partial y} - \sum_{j\neq i}^N \left[\frac{\partial \Phi_\mathrm{gc}\left(\mathbf{R}_i-\mathbf{R}_j|M_j,b_j\right)}{\partial y}\right], \\ 
            \dot{p}_{z,i} = - \frac{\partial \Phi_\mathrm{gal}}{\partial z} - \sum_{j\neq i}^N \left[\frac{\partial \Phi_\mathrm{gc}\left(\mathbf{R}_i-\mathbf{R}_j|M_j,b_j\right)}{\partial z}\right],
        \end{align}
        where $i$ is the index of a target cluster and $j$ iterates over the others. Note that in $\Phi_{\mathrm{gc}}$, I explicitly write $M_j,b_j$ to indicate the mass and scale length of the $j^{\mathrm{th}}$ cluster. Each cluster has it's own scale-length, which were computed from the half-mass radii reported in \citet{2018MNRAS.478.1520B}. Since the scale lengths differ from each cluster, we violate Newton's third law: $\dot{p}_{ij} \neq - \dot{p}_{ji}$. However, this is a minor consequence. Since I model each cluster as a Plummer sphere (as explained in the next section), at distances much greater than the scale radius, the functional form is the same as a point mass. 

        Lastly, in terms of the stream generation for Chapter~5, equation~\ref{eq:equations_of_motion_stream} are also modified to include the force from all the clusters, instead of just the host cluster. This becomes: 
        \begin{align}
            \dot{p}_{x} &= -\frac{\partial \Phi_\mathrm{gal}\left(\mathbf{r}\right)}{\partial x} - \sum_i^N \frac{\partial \Phi_\mathrm{gc}\left(\mathbf{r} - \mathbf{R}_i(t)\right)}{\partial x}\\
            \dot{p}_{y} &= -\frac{\partial \Phi_\mathrm{gal}\left(\mathbf{r}\right)}{\partial y} - \sum_i^N \frac{\partial \Phi_\mathrm{gc}\left(\mathbf{r} - \mathbf{R}_i(t)\right)}{\partial y}\\
            \dot{p}_{z} &= -\frac{\partial \Phi_\mathrm{gal}\left(\mathbf{r}\right)}{\partial z} - \sum_i^N \frac{\partial \Phi_\mathrm{gc}\left(\mathbf{r} - \mathbf{R}_i(t)\right)}{\partial z}
        \end{align}


    \subsection{The Gravitational Field} \label{subsec:gravfield}

        \textbf{what is a galaxy? an auto-gravitational system. We can say that a galaxy sadisfies the equations of possion. siccome the poisson's has this properties. I can decompse thegalaxy as a composition of lienar components. Each with its own distribution of mass. }

        The fundamental equation governing Newtonian gravity is Poisson's equation:
        \begin{equation}
            \nabla^2 \Phi = 4\pi G \rho,
        \end{equation}
        which generalizes Newton's law of gravity from point masses to continuous mass distributions. One of the powerful properties of this equation is its linearity, allowing arbitrary mass distributions to be decomposed into individual components whose contributions to the gravitational potential can be summed:
        \begin{equation}
            \nabla^2 \left(a_0\Phi_0 + a_1\Phi_1 \right) = a_0 \nabla^2 \Phi_0 + a_1 \nabla^2 \Phi_1 = 4\pi G \left(a_0\rho_0 +a_1\rho_1\right).
        \end{equation}

        It is instructive to compare stellar dynamics—the motion of stars in galaxies—with celestial mechanics—the motion of planets and spacecraft within the Solar System. In celestial mechanics, the gravitational field is known to high precision. Typically, one begins by considering the dominant mass (the Sun) and includes additional bodies only as needed. For instance, to place the James Webb Space Telescope at the Sun-Earth L2 Lagrange point, first-order calculations involve only the Sun and Earth. Likewise, when studying the long-term evolution of comet orbits, it may suffice to include only the Sun, Jupiter, and possibly Saturn. In these cases, the restricted three- or four-body problem provides a good approximation.

        \textbf{Make this simpler the point is that. In the solar system, you know the mass distirbution really well. The sun and some big planets. However, in the galaxy, we don't know the distribution of mass. So there's a ton a of models that are available. We don't know the exact distirbution. And the exact distirbution is more complex than just Jupiter and the sun. So we have a lot of freedom and lots of people have different models. Why did we pick this one? Why do we use pouliasis?}


        \textbf{The galaxy can be unsdertood of a combination of gas, dark matter, and stars. Gas is density. Stars are point masses. dark matter are large density profiles that onlt interact gravitationally. However, using poisson's equations we can just model each component as density profiles, even the dsitribution of stars we can consider to be in the fluid limit. The jsutification fo rwhich is the 2 body relaxation time which is in the next section. }


        \textbf{So yes the galaxy is this thing here and it follows poisson's equation. Poisson's equation has this linear quality. So all these different models pick different componets to satisfy a series of different observational constraints. Look at McMillan, Bovy, Pouliasis, and Ibata. We picked Pouliasis because it was a available to us made in our group. This model has these properties. These are the potentials. }
        
        In contrast, the Milky Way's gravitational potential is not precisely known and is far more complex. The Galaxy is not merely a collection of point masses; instead, it consists of billions of stars, gas, and dark matter, which collectively form structured components. Each of these components is represented with its own potential-density pair. While various methods exist to model the Galactic potential, most approaches adopt parameterized functions for each component. Notable examples include the widely used models of McMillan (2017) and Bovy (2015).

        The Galaxy is typically modeled as a superposition of several distinct components:

        \begin{itemize}
            \item the bulge a dense, roughly spherical central concentration of stars;
            \item the disk, where most of the stars and gas lie, with an approximately axisymmetric, flattened distribution;
            \item the halo which includes both the stellar halo and the dark matter halo.
        \end{itemize}

        More elaborate models further subdivide these components and can include time-dependent features. For example, the Galactic bar can be modeled as a rotating triaxial or prolate ellipsoid with a time-varying orientation. Moving substructures such as molecular clouds, dwarf galaxies, or dark matter subhalos can also be included. In my simulations, the globular cluster is treated as one such additional component, introduced on top of the smooth background.

        When simple analytic potentials are insufficient—especially for capturing non-axisymmetric or irregular features—basis function expansions can be employed. For instance, in spherical coordinates, one may use eigenfunctions of the Laplacian (e.g., spherical harmonics) to build the potential. Similarly, in cylindrical coordinates, Bessel functions naturally arise as solutions to the Laplacian and are well suited for modeling disks. (Cite Arpit's recent paper)


        \textbf{try to separate the plummer from the galactic components. Because it's better logically. Explain the martos halo. Say that the similar properties is that it has a characteristic radius and thus two different slopes inside and outside. }

        A common starting point when constructing gravitational potential models is to choose either the functional form of the density $\rho(\vec{x})$ and solve Poisson's equation to obtain the potential $\Phi(\vec{x})$, or vice versa. In practice, it is often simpler to specify a potential and then derive the corresponding density by differentiation, since solving Poisson's equation analytically is usually more tractable than performing the required integrals in the reverse direction.

        In this thesis, I adopt the Galactic potential model of Pouliasis et al. (2017), which builds upon the earlier model by Allen \& Santillan (1991) and incorporates improved fits to more recent observational constraints. The Galactic disk is modeled using the Miyamoto-Nagai potential:
        \begin{equation}
            \Phi(R, z \mid M, a, b) = -\frac{G M}{\sqrt{R^2 + \left(a + \sqrt{z^2 + b^2}\right)^2}},
        \end{equation}
        where $M$ is the total mass of the disk, $a$ is the radial scale length, and $b$ is the vertical scale height. This form smoothly interpolates between a flattened disk and a spherical distribution depending on the values of $a$ and $b$. In the limit $a \to 0$, the Miyamoto-Nagai potential reduces to the Plummer potential:
        \begin{equation}
            \Phi(r \mid M, b) = -\frac{G M}{\sqrt{r^2 + b^2}},
        \end{equation}
        which I use to model globular clusters in this work. The Plummer model provides a simple spherically symmetric mass distribution with a finite central density and a total mass that asymptotes at large radii. Interestingly, the Plummer potential is mathematically equivalent to a softened gravitational potential, commonly used in $N$-body simulations to regularize close encounters between particles.

        The dark matter halo is modeled separately and selected to reproduce the observed flat rotation curve of the Milky Way at large radii. This requirement translates into a mass profile that grows linearly with radius ($M(r) \propto r$) in the outer regions. A class of density profiles known as double power-law models satisfies this condition:
        \begin{equation}
            \rho(r \mid \alpha, \beta, r_s) = \rho_0 \left( \frac{r}{r_s} \right)^{-\alpha} \left(1 + \frac{r}{r_s} \right)^{\alpha - \beta},
        \end{equation}
        where $r_s$ is a scale radius, $\alpha$ controls the inner slope, and $\beta$ the outer slope. Common choices include the NFW profile ($\alpha = 1$, $\beta = 3$) and the Hernquist profile $(\alpha = 1$, $\beta = 4)$.

        The potential presented in Allen Santillan uses the Martos halo, which is not of the same functional family of the two power-law density model, but nonetheless has similar properties. The Martos halo is defined by a parameterized enclosed mass function:
        \begin{equation} 
            M_{\mathrm{enc}}(r|M_0,\gamma,r_0,r_c) = M_0
            \begin{cases}
             \frac{\left(r/r_0\right)^\gamma}{1 + \left(r/r_0\right)^{\gamma - 1}} & r<r_c,\\
             \frac{\left(r_c/r_0\right)^\gamma}{1 + \left(r_c/r_0\right)^{\gamma - 1}} & r> r_c,\\
            \end{cases} 
            \label{eq:martos_enclosed_mass}
        \end{equation}
        where $M_0$ is a mass scaling parameter (not the total mass), $r_0$ is the characteristic radius, $\gamma$ is the exponental parameter, and $r_c$ is a cutoff radius beyond which the mass is held constant. Thus, for $r > r_c$ Eq.~\ref{eq:martos_enclosed_mass} reports the total mass. In the outer regime $(r/r_0) >> 1$, the enclosed mass scales as $M(r) \propto r$, producing a flat rotation curve. In the inner regime $(r/r_0) << 1$, the mass grows as a power law, $M(r) \propto r^{\gamma - 1}$, which provides flexibility in shaping the central density slope. Since the unmodified profile leads to an unphysical divergence in total mass, a truncation at $r_c$ is imposed. When the potential has spherical symmetry we may use the relation $\nabla \Phi = \frac{G M_{\mathrm{enc}}(r)}{r^2}$, and requiring that the potential vanish at infinity, the corresponding gravitational potential can be obtained by integrating:


        \begin{equation}
            \Phi(r|M_0,\gamma,r_0,r_c) = 
            \begin{cases}
                \frac{GM_0}{r_0\left(\gamma-1\right)}\ln\left|\frac{1+(r/r_0)^{\gamma-1}}{1+(r_c/r_0)^{\gamma-1}}\right| -\frac{GM_t}{r_c}, & r<r_c\\
                -GM_t/r & r>r_c.
            \end{cases}
        \end{equation}
        Two Miyamoto-Nagai disks plus the Martos halo compose the second potential model presented in Pouliasis2017pii and is presented in Fig.~\ref{fig:figure_pouliasis2017pii_potential}.
        
        \begin{figure}
            \centering
            \includegraphics[width=\linewidth]{images/figure_pouliasis2017pii_potential_-8_8.png}
            \caption{Individual components of the second Galactic potential model from Pouliasis et al. (2017). The parameters used are listed in Table XXX. The vector field illustrates the normalized gravitational force at each position, corresponding to $\vec{g} = -\nabla\Phi$. Equipotential contour lines are overlaid to guide the eye.}
            \label{fig:figure_pouliasis2017pii_potential}
        \end{figure}        

        \textbf{Describe }
    
    \subsection{The initial conditions}\label{subsec:initialconditions}

        \textbf{Make this less generic and more specific. Why are we talking about CBE? because I need to generate the initial conditions. Perhaps discuss the initial conditions here.  Density -> positions -> velocity. Be explicit that the reason i'm writing about this is to get the initial conditions.}

        In this section, I want to show how we obtain the distribution of stars that populate the globular clusters used in these models. Specifically, I'll walk through the assumptions that reduce the general phase-space distribution to the simpler, tractable case relevant to our needs.

        In galactic dynamics, a fundamental challenge is the \textit{self-consistency problem}. This involves a loop of three steps:  
        (1) Given a density distribution and a gravitational potential (linked via Poisson's equation), one samples the positions of stars.  
        (2) One then assigns initial velocities and evolves the system forward in time.  
        (3) The potential is updated based on the evolved mass distribution.  

        If the system is in equilibrium and self-consistent, then the mass distribution remains stable under its own gravity. That is, the same density that generated the potential continues to persist as the system evolves. Crucially, while the mass density may be known, the velocities are not given a priori. To determine the velocity distribution, we turn to the Collisionless Boltzmann Equation (CBE).

        Our starting point is to characterize the system statistically, with a distribution function (DF) over phase space. That is, what is the probability density of finding a particle at a given position and velocity? This is encoded in the function \( f(\vec{x}, \vec{v}, t) \). If we treat this as a true probability density function that integrates to 1 over all phase space, we are implicitly assuming a closed system: no stars enter, leave, are born, or die.

        Already, by writing \( f(\vec{x}, \vec{v}, t) \), we are assuming that a particle's phase-space position is independent of any other attributes — such as mass. Let us define \( \mathbf{w} = (\vec{x}, \vec{v}) \) as the 6D phase-space coordinate. Then, for a system of \( N \) particles, the full distribution function lives in \(6N\)-dimensional space:  
        \[
        f(\mathbf{w}_1, \dots, \mathbf{w}_N)
        \]  
        This expression represents the joint probability density of finding particle 1 at \( \mathbf{w}_1 \), particle 2 at \( \mathbf{w}_2 \), and so on. In general, this object can be extremely complicated. But we can simplify it with two strong assumptions: that particles are independent and identically distributed (i.i.d.). This leads to:

        \[
        \begin{array}{rl}
        f 
        & \stackrel{\text{(1) $N$-body DF}}{\longrightarrow} 
        f(\mathbf{w}_1, \dots, \mathbf{w}_N) \\[2ex]
        & \stackrel{\text{(2) independence}}{\longrightarrow} 
        \prod_{i=1}^N f^i(\mathbf{w}_i) \\[2ex]
        & \stackrel{\text{(3) identical distribution}}{\longrightarrow} 
        \left[ f(\mathbf{w}) \right]^N
        \end{array}
        \]

        The consequence of this factorization is that the total DF no longer accounts for correlations in other variables — such as mass. Therefore, this model cannot represent mass segregation, multiple populations, or any other effects that differentiate stars. Instead, we describe a single homogeneous population of stars, each drawn independently from the same distribution.

        Because of this, we can focus our attention entirely on the single-particle DF \( f(\mathbf{w}) \), which encodes all the statistical information we need about the system.

        The next key assumption is that particles are uncorrelated not just in identity, but dynamically — they do not scatter off one another. This means the system is \textit{collisionless}. The DF then evolves under the Collisionless Boltzmann Equation:

        \begin{equation}
        \frac{Df}{Dt} = \frac{\partial f}{\partial t} + \dot{\mathbf{x}} \cdot \nabla_{\mathbf{x}} f + \dot{\mathbf{v}} \cdot \nabla_{\mathbf{v}} f = 0
        \end{equation}

        This is a material (Lagrangian) derivative following a phase-space trajectory. The physical interpretation is that the DF is conserved along the orbits of stars in phase space. No bunching up, no spreading out — stars simply move under the influence of a smooth, mean gravitational potential.

        Next, we often impose the equilibrium condition:
        \[
        \frac{\partial f}{\partial t} = 0
        \]
        This implies that the DF is time-independent: the number of stars at any given phase-space location remains constant. If some leave a region, others must replace them. This assumption is only valid in isolation — for instance, in an isolated galaxy or globular cluster. It is violated during mergers or tidal disruptions. In my simulations, I model such disruptions, so this assumption is not globally valid. However, I begin with an equilibrium system and study how it departs from equilibrium numerically.

        Now, let us focus on globular clusters. These are often modeled as spherically symmetric systems. However, spherical symmetry in density does not imply isotropy in velocity. A system can be spherically symmetric but have velocity anisotropy, meaning that orbits are preferentially radial or circular. This is quantified by the anisotropy parameter \( \beta \). In my simulations, I assume isotropy, meaning that the DF depends only on the energy \( E \). Thus, we write:

        \[
        f(\mathbf{w}) = f(E)
        \]

        At this point, it is useful to mention Jeans equations, which relate the moments of the DF to observable quantities. By integrating the DF over all velocities, we recover the spatial mass density:

        \[
        \rho(\mathbf{r}) = M \int f(\mathbf{x}, \mathbf{v}) \, d^3\mathbf{v}
        \]

        In fact, for isotropic spherical systems, an inversion formula exists — known as the Abel transform — that allows one to reconstruct \( f(E) \) from a known \( \rho(r) \). This technique is presented in Binney \& Tremaine and also in Bovy's online book. At this point, one can obtain a complete analytic expression for the distribution function. With this in mind, a discrete set of positions and velocities can be sampled and is presented in the next chapter.

        



\section{The Implicit Physics}

    \subsection{The Planar Circular Restricted Three-Body Problem}
        
        Even if we simplify the simulation by modeling the cluster as $\mathcal{N}$ independent three-body problems, the problem remains challenging. Indeed, if we consider all three bodies to have mass, we can write down a Hamiltonian with 18 dimensions: three positions and three momenta in $\mathcal{R}^3$ for each particle. The dimensionality of the problem can be reduced by using the conservation of total linear momentum and total angular momentum, and by expressing the dynamics in terms of relative coordinates about the system's center of mass. This reduces the total dimensionality to 9. Nevertheless, the problem remains analytically intractable due to its high dimensionality. To gain analytical insight, we simplify the system further.
        
        Instead of writing the full Lagrangian for the three-body motion, we focus on the third particle subject to the gravitational influence of the two massive bodies, in an inertial reference frame centered at the system's center of mass. The Lagrangian then contains the kinetic energy of the third particle and two gravitational potential energy terms from the primary and secondary. However, since the positions and momenta of the primaries are not treated as dynamical variables in our system, they appear as explicit functions of time, making the Lagrangian non-autonomous. This implies that the corresponding Hamiltonian is time-dependent and the total energy of the third particle is not conserved, as the primaries can exchange energy with it.

        We introduce a further simplifying assumption: the two primaries move on circular orbits. Under this assumption, we transform to a reference frame rotating with the primaries, placing them along the $x-axis$. In this rotating frame, the Lagrangian becomes autonomous; it no longer depends explicitly on time and requires no external information to determine the particle's subsequent motion.

        Two effects make this possible. First, by moving to the rotating frame, we introduce non-inertial forces: the Coriolis force and the centrifugal force. The centrifugal force is conservative, associated with a scalar potential. The Coriolis force depends on the particle's velocity as $2\omega\times v$, but because it is always perpendicular to the velocity, it does no work and thus does not change the particle's kinetic energy. After performing the coordinate transformation, the canonical momenta—now position-dependent—can be derived. The resulting Hamiltonian is:
        \begin{equation}
            \mathcal{H} = \frac{1}{2}\left(\left(p_x + \omega y\right)^2 + \left(p_y - \omega x\right)^2 \right) + \Phi_\mathrm{eff}(x,y),
        \end{equation}
        where
        \begin{equation}
            \Phi_\mathrm{eff}(x,y) = -\frac{1}{2} \omega^2 (x^2 + y^2) - \frac{G m_1}{|r_1|} - \frac{G m_2}{|r_2|}.
        \end{equation}

        The potential can be normalized by noting that the orbital angular velocity from the two-body problem satisfies \(\omega^2 = \frac{G M}{a^3}\), where \(a\) is the separation between the primaries and \(M = m_1 + m_2\) is the total mass of the system. With this normalization, the system depends on a single dimensionless parameter \(\mu\), the relative mass ratio defined as \(\mu = \frac{m_2}{m_1 + m_2}\).

        At this point, the system can be studied qualitatively. Unfortunately, no general closed-form solution exists for the circular restricted three-body problem that describes the subsequent motion as a function of time. However, by studying the effective potential \(\Phi_\mathrm{eff}\), we gain valuable insights.

        Our Hamiltonian depends on the four variables \((x, y, p_x, p_y)\) and has one integral of motion.\footnote{The term "integral of motion" is often used interchangeably with "constant of motion," but strictly speaking, the latter is preferable since it refers to a quantity that remains constant throughout the orbital evolution, not to an integral in the mathematical sense of the word (area under a curve).} A famous quantity in this context is the Jacobi integral (or Jacobi constant), often written as \(C_j = -2E\). Defining this constant is somewhat arbitrary since the total mechanical energy \(E\) is itself conserved, and any scalar multiple of it is also conserved. The utility of the Jacobi constant is mostly conventional: it is often defined to be positive and multiplied by 2, which simplifies the expression for forbidden regions and zero-velocity curves. For example, one can write
        \[
        \dot{x}^2 + \dot{y}^2 = 2 \Phi_\mathrm{eff}(x,y) + C_j,
        \]
        instead of the equivalent
        \[
        \dot{x}^2 + \dot{y}^2 = -2 \Phi_\mathrm{eff}(x,y) + 2E.
        \]
        

        Since we have four variables and one constraint, the motion is restricted to a three-dimensional hypersurface (or manifold) embedded in the four-dimensional phase space.

        At this stage, we find the points where \(\nabla \Phi_\mathrm{eff} = 0\), which correspond to the Lagrange points—locations where all effective forces balance. Of particular importance are the first two Lagrange points \(L_1\) and \(L_2\), which lie along the line connecting the primary and secondary. The effective potential at \(L_1\) is lower than at \(L_2\).

        Given a certain energy level, setting the kinetic energy to zero defines boundaries between regions where the particle can and cannot move. Regions where the kinetic energy would have to be negative (which is physically impossible) are forbidden, since that would require imaginary velocities.

        The points \(L_1\) and \(L_2\) are especially important for our study of a globular cluster with stars initially bound to it. Stars can escape through these Lagrange points: lower-energy particles tend to escape through \(L_1\), while higher-energy particles can also escape through \(L_2\). Figure~\ref{fig:CR3BP_forbidden_region} illustrates these forbidden and allowed regions clearly.




        \begin{figure}
            \centering
            \includegraphics[width=.32\linewidth]{images/CR3BP_forbidden_region_0.png}
            \includegraphics[width=.32\linewidth]{images/CR3BP_forbidden_region_1.png}
            \includegraphics[width=.32\linewidth]{images/CR3BP_forbidden_region_2.png}
            
            \includegraphics[width=.32\linewidth]{images/CR3BP_forbidden_region_3.png}
            \includegraphics[width=.32\linewidth]{images/CR3BP_forbidden_region_4.png}
            \caption{A reproduction of Fig.~2.4.2 from \citet{koon2000dynamical}. Each plot is shown in the non-inertial reference frame rotating with the orbital frequency of the primary and secondary about their common center of mass. In this example, the mass ratio is \(m_1 = 10 m_2\). The gray regions represent areas inaccessible to the test particle for a given energy. In the first case, where \(E < E_1\), the particle remains confined to orbit either around the primary or the secondary, depending on its initial position. As the particle's energy increases, more of the \(xy\)-plane becomes accessible. The narrow passage between \(m_1\) and \(m_2\) corresponds to the \(L_1\) Lagrange point, while the choke point allowing escape to the exterior when \(E > E_2\) is the \(L_2\) point.}
            \label{fig:CR3BP_forbidden_region}
        \end{figure}

        The distance from the secondary to the first and second Lagrange points is approximately equal, defining a region around the secondary known as the Hill sphere. Within this zone, the gravitational influence of the secondary dominates over that of the primary. The Hill radius is commonly approximated as \( r_h \approx r\sqrt[3]{m_2/m_1} \), where \( r \) is the separation between the primaries. Interestingly, the expression for the Hill radius arises from a fifth-order polynomial expansion of the effective potential—a detail that might seem minor, but is actually quite delightful. In most areas of physics, approximations typically stem from Taylor expansions, basis function decompositions, or limits where the governing equations simplify. 


        It is tempting to wonder whether a particle that escapes from the vicinity of the secondary can be recaptured. Indeed, within the simplified framework of the circular restricted three-body problem, recapture is possible. Once the initial conditions—position and momentum—are specified, the subsequent motion is fully determined by Hamilton's equations. While no general analytical solution exists, the uniqueness theorem ensures that the system's evolution is deterministic. Therefore, for a given total energy, we can identify the accessible regions in configuration space, bounded by the zero-velocity surfaces. This implies that the fate of the particle—whether it remains bound or escapes—is encoded in the initial conditions.

        In the context of celestial mechanics, particularly for applications such as spacecraft trajectory design, more sophisticated tools exist to analyze escape conditions and transfer orbits, including invariant manifolds and dynamical systems techniques. These are invaluable in the solar system regime, where the mass ratio is extreme and the test particle is typically a spacecraft or small body. However, in stellar and galactic dynamics, the situation differs substantially. The assumptions underpinning those techniques often break down, and the complexity of the gravitational potential increases. In this regime, it is more insightful to think of tidal forces as the underlying mechanism that gradually transfers energy and angular momentum, nudging stars across the Lagrange boundaries and ultimately driving escape. It is to this process—tidal stripping—that we now turn.


        \textbf{NOTE from Ferrone. I wonder if the $L_4$ and $L_5$ exist in the galactic context... are there Tojan stars? or does the complexity of the galactic potential paired with non circular orbits render such a phenomena unnatural?}

    \subsection{The tidal tensor}
        Tidal forces arise due to spatial variations in the gravitational field and are especially apparent when comparing the accelerations experienced by nearby particles. To explore this, consider a Taylor expansion of the gravitational potential of the primary, \(\Phi_g\), evaluated at the star's position \(\vec{x}_s\), relative to the secondary's position \(\vec{x}_c\):
        \begin{equation}
            \Phi_g\left(\vec{x}_s\right) \approx \Phi_g\left(\vec{x}_c\right) + \left[\nabla \Phi_g (\vec{x}_c)\cdot \Delta \vec{x}\right] + \left[\Delta \vec{x} \cdot \mathcal{D}^2\left(\Phi_g\right) \cdot \Delta\vec{x}\right],
        \end{equation}
        where \(\Delta \vec{x} = \vec{x}_s - \vec{x}_c\), and \(\mathcal{D}^2 \Phi_g\) is the Hessian matrix of second derivatives of the potential: \(\partial^2 \Phi/\partial x_i \partial x_j\).

        An equivalent expression can be derived by linearizing the gravitational force in a non-inertial frame co-moving with the secondary. Let us write Newton's second law for the star-particle and the secondary in an inertial frame:
        \begin{eqnarray}
            \vec{F}_s &= \nabla \Phi_c\left(\Delta \vec{x}\right) + \nabla \Phi_g\left(\vec{x}_s\right),\\
            \vec{F}_c &= \nabla \Phi_g\left(\vec{x}_c\right).
        \end{eqnarray}
        Then the relative acceleration of the star in the non-inertial frame is:
        \begin{eqnarray}
            \vec{f}_s &= \vec{F}_s - \vec{F}_c + \vec{F}_\mathrm{fictitious} \\
                    &= \nabla \Phi_c\left(\Delta \vec{x}\right) + \nabla \Phi_g\left(\vec{x}_s\right) - \nabla \Phi_g\left(\vec{x}_c\right) + \vec{F}_\mathrm{fictitious} \\
                    &\approx \nabla \Phi_c\left(\Delta \vec{x}\right) + \mathrm{Jac}\left(\nabla \Phi_g(\vec{x}_c)\right) \cdot \Delta \vec{x} + \vec{F}_\mathrm{fictitious},
        \end{eqnarray}
        where the last line uses a first-order Taylor expansion of the gravitational force field, valid under the assumption that \(|\Delta \vec{x}| \ll |\vec{x}_c|\). 

        The Jacobian of the gravitational field is equal to the Hessian of the potential, owing to the symmetry of second derivatives and the fact that \(\vec{g} = -\nabla \Phi_g\). This matrix, known as the \textit{tidal tensor} \(\mathcal{T}\), describes the linearized spatial variation of the gravitational field:
        \begin{equation}
            \mathcal{T} = -\mathcal{D}^2\Phi_g = \mathrm{Jac}(\nabla \Phi_g) = \left(\begin{matrix}
                \partial_x g_x & \partial_y g_x & \partial_z g_x \\
                \partial_x g_y & \partial_y g_y & \partial_z g_y \\
                \partial_x g_z & \partial_y g_z & \partial_z g_z 
            \end{matrix}\right).
        \end{equation}

        While the Hessian and Jacobian are formally equivalent, the Jacobian viewpoint offers a more geometric interpretation: it acts as a linear transformation on nearby displacements, mapping them to differences in acceleration. Diagonalizing the tidal tensor reveals the principal axes of tidal deformation. A positive eigenvalue corresponds to stretching along the associated eigenvector; a negative eigenvalue indicates compression. The magnitude gives the rate of stretching or compression.

        Finally, we note that although many relevant potentials exhibit spherical or cylindrical symmetry, Cartesian coordinates are preferred here. In curvilinear systems, computing the Jacobian or Hessian requires accounting for Christoffel symbols, which complicates the interpretation and computation.


        
        \subsubsection*{The Moon}
            Nothing clarifies the concept of tides like the most familiar example: the Moon. Tidal forces are invoked to explain a wide range of phenomena in the Earth-Moon system. The most relatable effect is, of course, the periodic variation in sea level on Earth. While accurately modeling these changes requires fluid dynamics—beyond the scope of this thesis—NASA provides several accessible explanations and visualizations at \href{https://science.nasa.gov/moon/tides/}{https://science.nasa.gov/moon/tides/}, including daily high and low tides, as well as spring and neap tides.

            Another key example is the tidal deformation of the Moon, which ultimately led to its tidal locking—explaining why we always see the same side of the Moon from Earth.

            A particularly insightful illustration is the angular offset between the Earth's tidal bulge and the Moon's position, caused by the Earth's rotation. This offset results in a torque that transfers angular momentum from the Earth's rotation to the Moon's orbit. As a consequence, Earth's rotation gradually slows while the Moon slowly recedes from Earth. Much of this behavior can be understood qualitatively using the tidal tensor for a Keplerian potential:
            \begin{equation}
                \mathcal{T}= -\frac{GM}{r^3}\left(\begin{matrix}
                    1-\frac{3x^2}{r^2} & -\frac{3xy}{r^2} & -\frac{3xz}{r^2} \\
                    -\frac{3yx}{r^2} & 1-\frac{3y^2}{r^2} & -\frac{3yz}{r^2} \\
                    -\frac{3zx}{r^2} & -\frac{3zy}{r^2} & 1-\frac{3z^2}{r^2}
                \end{matrix}\right),
            \end{equation}
            which has eigenvalues $2\frac{GM}{r^3}$, $-\frac{GM}{r^3}$, and $-\frac{GM}{r^3}$, with corresponding eigenvectors:
            \begin{equation}
                \vec{v}_1,\vec{v}_2,\vec{v}_3 = \dfrac{1}{r}\begin{bmatrix} x \\ y \\ z \end{bmatrix}, \dfrac{1}{r}\begin{bmatrix} -x \\ y \\ 0 \end{bmatrix}, \dfrac{1}{r}\begin{bmatrix} -x \\ 0 \\ z \end{bmatrix}.
            \end{equation}

            Notably, the first eigenvalue is positive and corresponds to a stretching deformation along the position vector. The other two are negative, representing compression in directions perpendicular to the stretching axis. These directions define a plane orthogonal to the Earth-Moon line. From this, several tidal effects become evident. For instance, the Earth's oceans stretch along the Earth-Moon axis due to the Moon's tidal forces. While the Sun also exerts tidal forces on Earth, their magnitude is weaker due to the $r^{-3}$ scaling with distance.

            When the Moon is either full or new, the Sun and Moon's tidal forces act constructively, leading to spring tides. At first and third quarters, they interfere destructively, causing neap tides. Additionally, Earth's tidal influence distorts the Moon from spherical symmetry into an ellipsoid. The Moon's most stable orientation is one where its longest axis aligns with the Earth-Moon line—resulting in tidal locking.

            A more quantitative treatment of these phenomena would require modeling the Moon's internal structure and Earth's ocean dynamics—well beyond the gravity-only scope of this thesis. However, we can still explore one instructive effect: how solar tidal forces \textit{perturb} the Moon's orbit away from the idealized two-body Earth-Moon configuration. Figure~\ref{fig:moon_tidal_simulation} shows a toy model comparing two scenarios. In both, I used initial conditions based on JPL NASA ephemerides (citation needed) and integrated two sets of equations of motion.

            In the first scenario, the Moon's motion is governed by the two-body Earth-Moon problem with a rotating reference frame correction:
            \begin{equation}
                \ddot{\vec{r}} = -\frac{GM_\oplus}{r^3}\vec{r} - \omega_\oplus \times \left(\omega_\oplus \times \vec{r}_\oplus\right),
            \end{equation}
            while in the second, we include the effect of solar tidal forces:
            \begin{equation}
                \ddot{\vec{r}} = -\frac{GM_\oplus}{r^3}\vec{r} - \omega_\oplus \times \left(\omega_\oplus \times \vec{r}_\oplus\right) -\frac{GM_\odot}{r_\oplus^3}
                \left(\begin{matrix}
                    1-\frac{3x^2}{r_\oplus^2} & -\frac{3xy}{r_\oplus^2} & -\frac{3xz}{r_\oplus^2} \\
                    -\frac{3yx}{r_\oplus^2} & 1-\frac{3y^2}{r_\oplus^2} & -\frac{3yz}{r_\oplus^2} \\
                    -\frac{3zx}{r_\oplus^2} & -\frac{3zy}{r_\oplus^2} & 1-\frac{3z^2}{r_\oplus^2}
                \end{matrix}  \right) \cdot \vec{r},
            \end{equation}
            where $r_\oplus$ is the Earth's position relative to the Sun, $\vec{r}$ is the Moon's position relative to Earth, $M_\odot$ is the mass of the Sun, and $M_\oplus$ is the mass of the Earth. The coordinates $x, y, z$ refer to the components of Earth's heliocentric position.

            
            
            \begin{verbatim}
            VIDEO: moon_tidal_simulation.mp4
            \end{verbatim}

            \begin{figure}
                \centering
                \includegraphics[width=\linewidth]{images/moon_tidal_simulation.png}
                \caption{An illustrative experiment demonstrating the effect of the Sun's tidal field. The left panels show the tidal field, and the right panels show two snapshots of the Moon's orbital trajectory. The green curve corresponds to the solution of Eq.~\textbf{XXX}, which includes the Sun's tidal effects, while the orange curve corresponds to the simpler two-body problem that neglects them. The black ellipse represents the tidal ellipsoid, whose major axis remains aligned with the Sun's position vector relative to the Earth. The bottom panel shows the accumulated phase difference between the two solutions. Neglecting solar tides causes the predicted Moon orbit to drift ahead of the more accurate trajectory. With about three to four years, the two body predicted solution would off by half a moon phase.
                }\label{fig:moon_tidal_simulation}
            \end{figure}

        \subsubsection*{Tides in the Galaxy}
            
            How the tidal field varies throughout the Galaxy by evaluating the tidal tensor along different orbits and in different mass models. In the Milky Way, the disk, bulge, and halo each contribute differently to the tidal forces experienced by a globular cluster. The Miyamoto-Nagai disk produces strong, rapidly varying tidal fields near the Galactic plane, leading to phenomena such as disk shocking when clusters cross the disk. The Martos halo, on the other hand, provides a more slowly varying, generally weaker tidal field at large Galactocentric radii.

            The strength and orientation of the tidal field at a cluster's location determine both the rate at which stars are stripped and the geometry of the resulting stellar streams. For example, clusters on eccentric or inclined orbits experience time-dependent tidal forces, with strong compressive shocks during disk crossings and enhanced stretching near pericenter. The eigenvalues and eigenvectors of the tidal tensor at each point along the orbit reveal the principal axes of stretching and compression, which in turn set the directions along which stars are most likely to escape.

            By computing the tidal tensor for the Miyamoto-Nagai disk and Martos halo potentials, as shown below, we can visualize and quantify these effects. The following figures illustrate how the tidal field evolves for representative orbits, highlighting the interplay between the cluster's trajectory and the Galactic mass distribution. This analysis underpins our understanding of stream formation and the morphological diversity of observed tidal tails.

            We can construct the tidal tensor for the Miyamoto-Nagai potential. First, it is convenient to non-dimensionalize the potential. Below, we normalize the potential by the total mass and gravitational constant, $\Phi\prime = \Phi / (GM)$, and each distance by the characteristic length of the disk, $x\prime = x/a$, $b\prime = b/a$. For clarity, we omit the prime notation in what follows. The dimensionless potential then becomes:            
            \begin{eqnarray}
                \Phi   &= \frac{1}{D},\\
                D       &= \sqrt{x^2 + y^2 + \beta^2(z)},\\
                \beta(z)   &= 1 + \sqrt{z^2 + b^2}.
            \end{eqnarray}
            
            % \begin{eqnarray}
                % \beta^{\prime}(z) &= \frac{z}{\sqrt{z^2 + b^2}}\\
                % \beta^{\prime\prime}(z)  &= \frac{b^2}{\left(z^2 + b^2\right)^{3/2}}
            % \end{eqnarray}
            The dimensionless tidal tensor is then: 
            \begin{equation}
                \mathcal{T}=-\frac{1}{D^3}\left(\begin{matrix}
                    1-\frac{3x^2}{D^2} & -\frac{3xy}{D^2} & -\frac{3x\beta \beta'}{D^2} \\
                    \dots & 1-\frac{3y^2}{D^2} & -\frac{3y\beta \beta'}{D^2} \\
                    \dots & \dots & \beta'^2 + \beta \beta'' -\frac{3\left(\beta\beta'\right)^2}{D^2}
                \end{matrix}\right).
            \end{equation}             
            We immediately notice that, due to the cylindrical symmetry, the eigenvectors are not as simple as in the spherical case. As long as $\beta'^2 + \beta \beta'' -\frac{3\left(\beta\beta'\right)^2}{D^2} \neq 1-\frac{3z^2}{D^2}$, the three eigenvectors are no longer simply (1) parallel to the position vector and (2) the other two spanning the plane perpendicular to it. Instead, the exact orientation of all three eigenvectors depends on the cluster's position in the galaxy. Note that if $z=0$, we recover the stretching eigenvector being parallel to the position vector, although the orientations of the eigen vector are fixed as the compression axes are not the same in magnitude. If, we $b=0$, then we recover spherical symmetry and the compression eigenvalues are the same in magnitude.

            \begin{verbatim}
            VIDEO: tidal_deformation_ellipsoid.mp4
            \end{verbatim}

            Below I have prepared some selected orbits to demonstrate the variety of shocks and tidal forces than can influence a globular cluster system. 
            
            \begin{figure}
                \includegraphics[width=\linewidth]{images/miyamoto_disc_shocks_ab_rp_e_i_0.20_4.0_0.01_50.0.png}
                \caption{Tidal forces on an inclined yet non-eccentric orbit. Disk shocks are present, yet there is no tidal stretching from pericenter passages. \textbf{Left}: The three eigenvalues of the tidal tensor matrix are plotted against time. Both the forces and time are normalized to the characteristic values of the Miyamoto-Nagai model, where $a$ is the radial scale length and $M$ is the total mass of the system. The red and green curves correspond to the two compressive axes, while the blue curve shows the magnitude of the stretching axis. The parameters listed at the top describe the orbit: the ratio of cylindrical to vertical scale lengths, the apocenter distance, the eccentricity, and the initial orbital inclination. \textbf{Right}: The orbit is shown in the meridional plane. The purple stars indicate disk crossing events and correspond to the peaks in the magnitude of the eigenvalues. }
                \label{fig:miyamoto_disc_shocks_circular_inclined_orbit}
            \end{figure}


            \begin{figure}
                \includegraphics[width=\linewidth]{images/miyamoto_disc_shocks_ab_rp_e_i_0.20_4.0_0.50_0.0.png}
                \caption{Evolution of the tidal eigenvectors for an eccentric, non-inclined orbit, resulting in a compressed meridional plane. Orange dots mark the pericenter passages. Since the cluster remains confined to the plane, no disk shocks occur.}
                \label{fig:miyamoto_disc_shocks_planar_eccentric_orbit}
            \end{figure}

            \begin{figure}
                \includegraphics[width=\linewidth]{images/miyamoto_disc_shocks_ab_rp_e_i_0.20_4.0_0.50_30.0.png}
                \caption{An inclined, eccentric orbit in which the frequencies in both the $R, p_R$ and $z, p_z$ planes are nearly resonant. The cluster crosses the disk just before and just after each pericenter passage. This is the same orbit as the video presented in this section that is available in the online version.}
                \label{fig:miyamoto_disc_shocks_responant_R_z}
            \end{figure}
            
            \begin{figure}
                \includegraphics[width=\linewidth]{images/miyamoto_disc_shocks_ab_rp_e_i_0.20_31.6_0.60_50.0.png}
                \caption{An eccentric, inclined orbit with a large apocenter. As the cluster evolves through phase space, disk crossings that occur farther out happen at steeper angles and in lower-density regions, reducing the strength of the resulting disk shocks. In contrast, crossings near pericenter remain strong. Because this orbit has higher energy than the previous cases, the overall magnitude of the tidal forces is lower. }
                \label{fig:miyamoto_disc_shocks_big_apocenter}
            \end{figure}
            
            \begin{figure}
                \includegraphics[width=\linewidth]{images/miyamoto_disc_shocks_ab_rp_e_i_0.88_4.0_0.50_45.0.png}
                \caption{Eccentric and inclined orbit in a weak disk. The ratio of the cylindrical scale length  to characteristic height (a/b) is close to 1. The disk crossings still produce significant tidal compression, but the resulting shocks are broader.}
                \label{fig:miyamoto_disc_shocks_weak_shocks}
            \end{figure}


   
        \subsubsection*{Interesting case}

            The Marto's halo has this mass distribution:
            
            \begin{equation}
                M'_\text{enc}(s) = \frac{s^\gamma}{1+s^{\gamma-1}}
            \end{equation}
            The dimensionless tidal tensor is thus: 
            \begin{equation}
                \mathcal{T'}_{i,j}= -\frac{M'_\text{enc}(s)}{s^3}\left(\begin{matrix}
                    1-\frac{x^2}{s^2}f(s) & -\frac{xy}{s^2}f(s) & -\frac{xz}{s^2}f(s) \\
                    -\frac{yx}{s^2}f(s) & 1-\frac{y^2}{s^2}f(s) & -\frac{yz}{s^2}f(s) \\
                    -\frac{zx}{s^2}f(s) & -\frac{zy}{s^2}f(s) & 1-\frac{z^2}{s^2}f(s)
                \end{matrix}\right)
            \end{equation}  
            where 
            \begin{equation}
                f(s) = 2-\frac{\gamma-1}{1+s^{\gamma-1}}
                \label{eq:martos_f_s}
            \end{equation}

            There is an interesting area in the parameter space where the tidal forces would impede creating stellar streams instead of making them, as shown in Fig.~\ref{fig:martos_tidal_field_small_r}.

            Taking the Martos tidal tensor in Eq.~\ref{eq:martos_f_s}, we can see that for $\gamma > 3$ and $s \ll  1$, then $f(s)< 0$. Physically, this would be a sphere whose density increases with distance. This is not natural, as, in general, gravity sends the more massive objects towards the center. However, it's fun to indulge in this situation to learn some insight about the flexibility of tidal fields. The consequence of $f(s)< 0$ is that all terms in the tidal tensor are negative, which means that the force is compressive everywhere. Consequently, no stars escape from the cluster. 

            In Fig.~\ref{fig:martos_tidal_field_small_r}, I present a small experiment demonstrating the consequence of such a tidal force on a globular cluster, which is that no tidal stream forms. Briefly, I created a plummer sphere of $10^6$~M$_\odot$ and half mass radius 20~pc and evolved it in a Martos halo potential of mass parameter $10^{12}$~M$_\odot$ a characteristic radius of $30$~kpc. Each cluster was placed at the same initial conditions, a distance of 1/4 the scale radius from the center of the potential. The initial velocity was made perpendicular to the position vector with a speed of $(1-e)v_\textrm{c}$. This is a pseudo-eccentricity, which was added to have a non-circular orbit to demonstrate how the trajectories change in the two cases. The top panel uses a $\gamma$ of 2.02, which is the same value in the model where the halo was originally presented, and the value I employ in this thesis. Next, the bottom panel uses $\gamma$ of 4.5, which corresponds to a density profile where $\rho(r) \propto r^{1.5}$. 

            To get a feel for the strength of the tidal stretching and compression, I show a circle and the resulting ellipse after applying the tidal deformation. I computed the coordinates of the ellipse by adding $\vec{Ell} = \vec{C} + \frac{1}{2} t_\textrm{char}^2 \mathcal{T}\cdot \left(\vec{C} - \vec{r_o}\right)$. This way, force can be mapped to position space, and the strengths of the tidal forces can be seen visually. The characteristic time, $t_\textrm{char}$, was set to $\frac{1}{10} 2\pi r_\textrm{halo} / v_\textrm{c}$. 
            
            \begin{figure}
                \includegraphics[width=\linewidth]{images/martos_tidal_field_202_10_25.png}
                \includegraphics[width=\linewidth]{images/martos_tidal_field_450_10_25.png}
                \caption{The plots show two low-resolution streams (N = 1000) created by dissolving a Plummer sphere in the Martos halo potential. The units are scaled to the halo's characteristic radius. Gamma is the mass exponent and is the sole variable between the two simulations. The panels on the left show the orbit in gray and the stars in black. The black arrow points towards the center of the potential. The panels on the right show the tidal field, which is the tidal tensor evaluated at each position in space. The gray dotted circle is plotted with an arbitrary radius and is deformed by the tidal field into a black ellipse.}
                \label{fig:martos_tidal_field_small_r}
            \end{figure}

            \begin{figure}
                \includegraphics[width=\linewidth]{images/martos_tidal_field_202_10_400.png}
                \includegraphics[width=\linewidth]{images/martos_tidal_field_450_10_400.png}
                \caption{The same experiment as Fig.~\ref{fig:martos_tidal_field_small_r}, but the cluster was placed at a larger distance of 4 $r_\textrm{halo}$, since we are beyond the characteristic radius, the tidal fields are the same, despite the different exponents $\gamma$.}
                \label{fig:martos_tidal_field_big_r}
            \end{figure}

            In the case of Fig.~\ref{fig:martos_tidal_field_big_r}, the tidal field returns to the typical situation where one axis is compressive and the other stretches. Notice how the deformations are similar in magnitude, while in the case of $\gamma=2.02$ for the top panel of Fig.~\ref{fig:martos_tidal_field_small_r}, the compression is stronger than the expansion. Both of these are different than the keplerian tidal deformation where the stronger deformation is stretching and whose axis is parallel to the position vector. 






    \subsection{Phase mixing}
        \textbf{NOTE: I want to add citations and discussions to Vasiliev Gaia DR3's look on GCs}

        
        \textit{Phase mixing} is a direct consequence of Liouville's theorem, which itself follows from the collisionless Boltzmann equation. Liouville's theorem states that the infinitesimal volume element in phase space is preserved under Hamiltonian evolution. In other words, as a system evolves in time, the phase-space density \( f(\mathbf{q}, \mathbf{p}, t) \) is conserved along particle trajectories. Since the number of particles is conserved, and the phase-space volume they occupy does not change, any spatial spreading must be accompanied by a drop in phase-space density.

        This effect causes ensembles of nearby orbits to stretch and fold through phase space, spreading out and becoming more finely interleaved over time, even though the total volume is constant. In physical space, this corresponds to a decrease in local density as the particles become more dispersed—this is the essence of phase mixing.

        We can estimate the characteristic time scale for phase mixing by considering how long it takes for orbits with slightly different energies to drift apart in phase. For a Keplerian orbit, the orbital period is given by:

        \begin{equation}
        T^2 = \frac{4\pi^2 a^3}{GM},
        \end{equation}

        which implies that the period depends on the semi-major axis and thus on the orbital energy. In general, most orbits are not closed, so rather than returning to the exact same phase-space point, particles precess and fill out invariant tori over time. It is therefore useful to define a characteristic orbital timescale in terms of energy. From dimensional analysis, we argue that the characteristic time for a body to complete a phase space cycle is:

        \begin{equation}
        T_\mathrm{char} = C \frac{GM}{E^{3/2}},
        \end{equation}

        where $ C$ is a constant and $E$ is the specific orbital energy.

        Consider now two nearby particles with energies $E - \Delta E$ and $ E + \Delta E $. Their orbital frequencies differ by:

        \begin{equation}
        \Delta f = \frac{1}{T( E - \Delta E)} - \frac{1}{T(E + \Delta E)}.
        \end{equation}

        The time required for the faster particle to lap the slower one in phase is the \textit{phase mixing time}, which is the inverse of the difference $2\pi / \Delta f$ and is given by:

        \begin{equation}
        T_\mathrm{mix} = \frac{2\pi}{\Delta f} = \frac{2\pi T_1 T_2}{T_2 - T_1}.
        \end{equation}

        For small energy differences \( \Delta E << E \), we can expand the period in a Taylor series to first order. This leads to:

        \begin{equation}
            T_\mathrm{mix} \approx T_\mathrm{char} \cdot 2\pi \left( \frac{E}{3 \Delta E} \right),
            \label{EQ:phase_mixing}
        \end{equation}
        Thus, the phase mixing time grows inversely with the relative energy spread in the population.

        \subsubsection{Orbital drift}
            The first manifestation of phase mixing is the growth of uncertainty in orbital solutions for globular clusters over time. This originates from the initial spread in orbital energies caused by observational uncertainties. In Fig.\ref{fig:phase_mixing_palomar_5_orbital_solutions}, I present an example based on Palomar~5, for which I compute 50 orbital solutions in our potential model. These initial conditions are sampled according to the uncertainties reported in the Baumgardt catalog. The figure shows that while the orbital solutions remain broadly similar, the lower-energy orbits progress through phase space more rapidly. In the bottom panel, the rightmost side includes overplotted dots indicating the final $(t, z)$ coordinates of each solution, illustrating that they span nearly the full range of $z$ values allowed by the initial conditions. In essence, once the integration time exceeds the phase-mixing timescale, any individual solution becomes speculative: the system could plausibly occupy any location in phase space permitted by the initial conditions.


            \begin{figure}
                \centering
                \includegraphics[width=\linewidth]{images/phase_mixing_palomar_5_orbital_solutions.png}
                \caption{Phase mixing of Palomar~5's orbital solutions in the \texttt{Pouliasis2017pii} potential. We sample 50 different initial conditions based on the observational uncertainties in distance, radial velocity, and proper motions. Each orbital solution is color-coded by its initial total orbital energy $E$. The top six panels show snapshots of the positions in the $xy$-plane of the orbital solutions at different times. The bottom panel shows the evolution of the $z$ coordinate as a function of time. Black vertical bars mark the timestamps corresponding to the snapshots in the top panels, which are labeled with matching alphabetical identifiers.}
                \label{fig:phase_mixing_palomar_5_orbital_solutions}
            \end{figure}    

            By inspecting Eq.\ref{EQ:phase_mixing}, we note that if the phase-mixing time is normalized by the characteristic orbital time, the resulting dimensionless mixing time is inversely proportional to the uncertainty in orbital energy. This relationship is general and holds for all orbits, regardless of their periods. However, it is still illustrative to examine this behavior in the context of the globular cluster catalog. In Fig.\ref{fig:phase_mixing_orbital_errors_sample}, I present the phase-mixing times for all Galactic globular clusters, based on current uncertainties from the Gaia DR3 catalog. The top panel shows the distribution for all 165 clusters. The bottom panels highlight the mixing behavior in cylindrical radius for a selection of statistically representative clusters: Gran~1, which has the shortest mixing time; Gran~5, the median case; NGC~6752, whose mixing time is closest to the mean; and NGC~2419, which has the longest mixing time.

            It is important to emphasize that these estimates are highly model-dependent. In some cases, the models predict phase-mixing timescales exceeding the age of the Universe. Unsurprisingly, the phase-mixing time correlates with the orbital period, which itself is related to the cluster's distance from the Galactic center. At large Galactocentric distances, the gravitational potential of the Milky Way becomes increasingly uncertain, as observational constraints on the Galactic mass distribution are weaker. Therefore, these extreme mixing times should not be interpreted as implying that we can predict the phase-space location of certain clusters with high confidence over cosmological timescales. 



            \begin{figure}
                \centering
                \includegraphics[width=.75\linewidth]{images/phase_mixing_time_histogram_MWGCS.png}
                \includegraphics[width=\linewidth]{images/phase_mixing_orbital_errors_sample.png}
                \caption{The top panel shows the distribution of phase-mixing times for globular clusters, computed using Eq.~\ref{EQ:phase_mixing}. The following four rows illustrate the phase-mixing behavior for four selected clusters over the time range considered in this experiment. Orbital solutions are color-coded by their normalized deviation from the mean orbital energy.}
                \label{fig:phase_mixing_orbital_errors_sample}
            \end{figure}
     
            It's also interesting to note how the uncertainties in the orbital energy relates to the uncertainties in each observable. In Fig.~\ref{fig:energy_sensitivity_analysis_MWGCS_to_distance_RV_mu}, I perform this quick calulation. The uncertainties are measured in Galactic coordinates the ICRS coordinate system. This is a direct transformation to Galactic coordinates and thus can be expressed analytically.

            \begin{figure}[p]
                \includegraphics[width=\linewidth]{images/energy_sensitivity_analysis_MWGCS_to_errors.png}
                \caption{An analysis on the spread in orbital energies of the globular cluster population for each of its reported uncertainties. Each panel finds the STD in the Hamiltonain given the uncertainties in the specific variable. So the top left allows the distances to vary but holds the proper motions and radial velocities constant. It is plotted against the distance. The clusters are color-coated by their mean energies.}
                \label{fig:energy_sensitivity_analysis_MWGCS_to_distance_RV_mu}
            \end{figure}

        \subsubsection{Mixing of tidal debris}
            At first glance, Fig.~\ref{fig:phase_mixing_palomar_5_orbital_solutions} might give the impression that we are looking at snapshots of a single stellar stream at different times. In reality, the figure shows different orbital solutions. This visual similarity underscores that both phenomena—stellar streams and diverging orbital solutions—are manifestations of phase mixing. However, there is a crucial distinction: phase mixing of tidal debris is a physical process involving actual particles with slightly different initial conditions, whereas diverging orbital solutions represent the increasing uncertainty in predicted trajectories over time.

            Still, it is useful to use phase mixing as an analogy, though the physical assumptions differ. In particular, the analogy breaks down because globular clusters are self-gravitating systems: stars interact gravitationally and remain bound for a significant time. This self-gravity slows down the mixing process. The time it takes for a globular cluster to dissolve and distribute its stars uniformly in phase space is typically longer than the idealized mixing time given by Eq.~\ref{EQ:phase_mixing}.

            Let us briefly illustrate this with a conceptual calculation:

            \begin{itemize}
                \item \textbf{Case 1}: Particles drift apart under phase mixing with no mutual gravitational influence.
                \item \textbf{Case 2}: Particles are bound by a central potential, representing the cluster's self-gravity.
            \end{itemize}
            From this thought experiment, we see that phase mixing proceeds more slowly when self-gravity is included. Nevertheless, phase mixing still governs the long-term behavior of tidally escaped stars.

            Importantly, we need not consider all particles within the cluster to study phase mixing. For clusters with short orbital mixing times compared to the simulation duration, stars that escape during the simulation can redistribute throughout the available phase space. While the resulting debris will not be uniformly distributed—an overdensity will remain at the cluster location—the tidal debris will broadly fill the phase space accessible to it. This point is crucial for the results discussed in Chapter 4.

            
    \subsection{Collisions}

        Collisions are essential in this thesis for two reasons. Firstly, one of the main research questions is how the passage of a subhalo perturbs a stellar stream. This interaction can be treated as a gravitational collision.

        In general, we are interested in how much the momentum of a particle changes before and after the encounter. We can simplify the scenario by assuming that the target particle is initially at rest, and that the perturber (the subhalo) flies by with a minimum distance of approach $b$—the \emph{impact parameter}.

        Since both objects exert gravitational forces on each other, the target will gain energy and begin to move, while the perturber will be slightly deflected and slow down. However, to simplify the calculation, we assume the perturber is unperturbed by the target—i.e., it moves on a straight-line trajectory with constant speed $v$.

        In this approximation, we compute the momentum change of the target by integrating the gravitational force over time. Gravity is always attractive: the perturber pulls the target particle to the left before the closest approach and to the right afterward. If the trajectory is symmetric and unperturbed, the components of the force parallel to the motion cancel out. Thus, the net momentum transfer is perpendicular to the motion of the perturber.

        Assume the perturber moves along the $x$-axis with impact parameter $b$, and the target lies at the origin. Then the net force acts in the $+y$ direction. The gravitational force scales as $1/d^2$, so for distances $d \gg b$, the force becomes negligible. We can then approximate the force as roughly constant over a short time interval during the closest approach.

        We estimate the duration of the interaction as the time it takes the perturber to travel a distance $2b$ (from $x = -b$ to $x = +b$), giving $\Delta t = 2b/v$. Assuming a constant perpendicular force $F_\perp \approx GMm/b^2$, the change in momentum is:

        \begin{eqnarray}
        m\,\delta v_y &=& \int F_\perp \, dt \\
                    &\approx& F_\perp \cdot \Delta t \\
                    &=& \frac{GMm}{b^2} \cdot \frac{2b}{v}
        \end{eqnarray}

        Dividing both sides by $m$, the change in velocity is:
        \begin{equation}
        \delta v_y = \frac{2GM}{b v}
        \end{equation}

        It is interesting to note that this result is inversely proportional to the velocity of the perturber. This behavior contrasts with contact collisions, where a higher speed typically results in a greater momentum transfer. In gravitational encounters, a faster flyby leads to a shorter interaction time and thus less momentum exchange.

        \subsubsection{Gapology}
            Gaps in stellar streams are underdensities that appear after encounters with dark matter subhalos or globular clusters. These can be understood as resulting from collisions. However, modeling the interaction as one point mass hitting another is too simplistic. What happens when an extended object, like a subhalo, interacts with a continuous stellar stream? Before delving into the technical details, it is worth clarifying a misconception--one that I myself held when first approaching this problem. Encounters between stellar streams and perturbers are not violent collisions in which stars are physically ejected or stripped from the stream. Rather, the perturber imparts a small, localized change in velocity to nearby stars, subtly altering their orbital periods. Over time, these tiny shifts accumulate, causing stars to drift apart and gradually sculpting a visible underdensity: the gap. 
            
            To address this, we turn to the work of \citet{2015MNRAS.450.1136E}, who studied a simplified model: a stream with uniform density on a circular orbit. This idealized setup allows for an analytical treatment of the gap's formation and evolution. A gap can be characterized by two main quantities: its angular width, $\Delta \theta$, and the density contrast, $\rho_{\mathrm{peak}}/\rho_0$, between the overdensities at the gap's edges and the central underdensity.


            To understand how these evolve, the authors derived a parameterization of the change in velocity $\Delta \vec{v}$ for stars along the stream after a perturbation. This vector function depends on the position $y$ along the stream relative to the impact point and on several parameters describing the perturber and the impact geometry:
            \[
            \Delta \vec{v} = f(y \,|\, M, r_s, b, w_\parallel, w_\perp, \alpha),
            \]
            where $M$ is the impactor's mass, $r_s$ its scale radius, $b$ the impact parameter, $w_\parallel$ and $w_\perp$ the components of its velocity parallel and perpendicular to the stream, and $\alpha$ the angle between the impactor's trajectory and the $(x,z)$-axes.

            To set up the problem, a coordinate system is defined where the stream lies along the $y$-axis, and the impact occurs at $y=0$. Because the stream has spatial extent, the impactor's motion must be decomposed into parallel and perpendicular components with respect to the stream. Unlike the simpler point-mass approximation, here $\Delta \vec{v}$ generally has nonzero components in all directions. In particular, $\Delta v_y$ alters the speed along the stream, while $\Delta v_x$ and $\Delta v_z$ displace stars out of the stream's plane. $\Delta v_x$ or $\Delta v_z$ can be zero if the trajectory of the impactor is parallel to the $x$ axis or $z$ axis, respectively. 

            The expression for $\Delta v_y$ is:
            \[
            \Delta v_y\left(y\,|\, M, r_s, b, w_\parallel, w_\perp\right) = - \frac{2GM w_\perp^2 y}{w\left[\left(b^2 + r_s^2\right)w^2 + w_\perp^2 y^2\right]},
            \]
            which is an odd function of $y$, as expected for a gravitationally attractive force: stars ahead of the impact point ($y>0$) are slowed down, while those behind ($y<0$) are sped up.

            \begin{figure}
                \centering
                \includegraphics[width=\linewidth]{images/erkal_et_al_2015_fig_3.png}
                \caption{The change in velocity, $\Delta v_y$, of stars along the stream after an impact. The stream lies along the $y$-axis, with the point of impact at the origin. Key features are labeled: the maximum value of $\Delta v_y$, its location, and the full width at half maximum. Taken from Fig.~3 of \citet{2015MNRAS.450.1136E}.}
                \label{fig:erkal_2015_fig_3}
            \end{figure}

            Several assumptions underlie this formulation. First, the stream is approximated as a straight line, which requires the size of the impacted region to be much smaller than the orbital circumference. Second, the velocities of both the stream and the perturber are assumed constant, implying that the duration of the impact is much shorter than the system's orbital period. These assumptions also imply that the impactor must pass close to the stream and be compact relative to the stream's orbital radius.

            The authors then developed the time evolution of the stream's density following the impact. To study these, the authors employed three complementary approaches:
            \begin{enumerate}
                \item A numerical solution of a transcendental equation governing $\Delta \theta(t)$;
                \item A fully analytic solution for a low-order analytical approximation valid when the gap opening $\Delta \theta$ is still small;
                \item An $N$-body simulation of the full system.
            \end{enumerate}

            Together, these revealed three distinct phases of gap evolution:
            \begin{enumerate}
                \item \textbf{Compression phase} --  Shortly after the impact, stars on opposite sides swap places, briefly compressing the stream until the right and left sides overtake one another. 
                \item \textbf{Linear growth phase} -- The gap grows roughly linearly in time, as stars move apart under their velocity differences.
                \item \textbf{Caustic phase} -- Eventually, stars bunch up at different locations at the edges of the gap, creating sharp overdensity features (``caustics''). Additionally, the gap's growth to is slowed to a sublinear rate, scaling roughly as $\propto \sqrt{t}$.
            \end{enumerate}


            The author's make an important realization. Even in a scenario where the galactic potential and the stream's orbit are known, with just the gap's density profile there are still seven parameters that describe it's shape: $M,r_s,b,w_\perp,w_\parallel,\alpha,t$. The authors state that the gap's profile provide two more piece of information and with a constraint on the perturber's density, we would still be left with four degrees of freedom. The problem is thus underdetermined and it is thus impossible to uniquely determine the properties of the perturber. A large statistical sample of gaps would be required to constrain a population of dark matter subhalos. With the ever increasing data quality, perhaps this could be possible someday.  

            While powerful, the authors describe how the limiting assumptions impact the interpretations of this work. First, the framework is limited to circular orbits which poses a significant constraint especially for globular clusters which often have inclined and eccentric trajectories. Indeed, as they discussed in their work, streams do not have uniform densities. Stars that populate the streams come from tidally disrupting stellar systems. As they leave the cluster and enter the stream, they come with a range of energies and and angular momenta that in essence eliminate any casutic features. 

            \citet{2016MNRAS.457.3817S} expanded on this foundation and provided a truly exchaustive and comprehensive exploration of stream-subhalo interactions. Their primary methodological innovation was the development of a framework for modeling the impact of dark matter subhaloes on cold, thin streams in angle-frequency space. This space significantly simplifies stream dynamics, allowing for the rapid generation of general stream models and proving ideal for incorporating velocity perturbations from subhaloes. They developed methods to compute velocity, angle, and frequency kicks for general subhaloes and impact geometries, then translated these into angle and frequency perturbations. This framework allowed for an extensive range of experiments:

            \begin{itemize}
                \item \textbf{Various perturber models}: \citet{2016MNRAS.457.3817S} compared velocity kicks produced by various astrophysically relevant subhalo profiles, including Plummer, Hernquist, truncated Navarro-Frenk-White (NFW), and non-truncated NFW profiles. They found that while there are differences in kick amplitudes depending on the profile, the absolute differences between these methods were generally small, especially for lower-mass impacts, suggesting the simpler "curved approximation" (which accounts for stream curvature but assumes fixed relative velocity during flyby) is often sufficient
                \item \textbf{Different Orbits for the Progenitor}: key advancement was the focus on streams formed from progenitors on eccentric orbits. This directly addressed a limitation of \citet{2015MNRAS.450.1136E}, which primarily restricted its analysis to streams on circular orbits. The angle-frequency space formalism is particularly well-suited for modeling these more complex, realistic eccentric stream dynamics.
                \item \textbf{Different Impact Points}: The study thoroughly investigated how the stream changes depending on where along the stream the impact occurs. They simulated impacts both close to the progenitor (where particles are more mixed in energy) and further downstream (where particles are better ordered by energy), finding that the growth rate of the gap depends on the impact location. 
            \end{itemize}

            Drawing from these comprehensive investigations, \citet{2016MNRAS.457.3817S} reached several essential conclusions and insights into ``gapology'':
            \begin{itemize}
                \item \textbf{Dominance of Frequency Kicks}: They found that angle perturbations from a flyby are generally only significant on short timescales (less than a radial period). At later times, the frequency perturbations become much more important, controlling the future structure of the gap. These frequency perturbations are simply related to the velocity perturbations (specifically, the change in energy, $v \cdot \delta v_g$) in scale-free potentials.
                \item \textbf{Gap Growth and Density Plateau}: While \citet{2015MNRAS.450.1136E} predicted gap growth slowing to a $t^{1/2}$ rate in the caustic phase and the central density decreasing as $t^{-1}$, \citet{2016MNRAS.457.3817S} observed that the minimum gap density eventually plateaus in time. This plateau value decreases with increasing subhalo mass and is attributed to the stream's non-zero velocity dispersion, allowing upstream material to fill in the gap, and the non-uniform unperturbed stream density.
                \item \textbf{Impact Location Affects Growth Rate}: They discovered that gaps formed far downstream grow more rapidly than those closer to the progenitor. This is because stream members far from the progenitor are more "ordered" by energy, leading to an already growing underlying stream structure that enhances gap formation.
            \end{itemize}
            For many of the experiments, the authors cross validate the semi-analytic formalism against NBody simulations and present truly robust results. Below, I want to show how one can understand a 1D stream. 


            To first order, a stellar stream can be modeled as a \textit{collisionless}, non-accelerating ensemble of stars — i.e., a one-dimensional \textit{streaming} solution to the collisionless Boltzmann equation. In this case, the equation simplifies to:

            \begin{equation}
                \frac{\partial f}{\partial t} + v \frac{\partial f}{\partial x} = 0,
            \end{equation}

            where \( f(x,v,t) \) is the phase-space distribution function. We assume the initial condition \( f(x,v,t=0) = 0 \), and a boundary condition of a constant source at \( x=0 \):

            \begin{equation}
                f(0,v,t) = g(v \,|\, v_0, \sigma_v) = \mathcal{N}(v \,|\, v_0, \sigma_v),
            \end{equation}

            i.e., a Gaussian ejection velocity distribution centered at \( v_0 \) with dispersion \( \sigma_v \). The total flux amplitude is arbitrary here.

            Using the method of characteristics, we find that \( f \) is constant along lines of the form \( x - vt = \text{const} \). This means we are solving a PDE in the \( (x,t) \) plane for fixed values of \( v \). The initial condition implies that \( f=0 \) in regions not yet reached by any particles — that is, wherever \( x/t > v \). Thus, the solution is:

            \begin{equation}
                f(x,v,t) = 
                \begin{cases}
                    \mathcal{N}(v \,|\, v_0, \sigma_v) & \text{if } v > \frac{x}{t}, \\
                    0 & \text{otherwise}.
                \end{cases}
                \label{eq:one_dimensional_collisionless_streaming}
            \end{equation}

            We can now compute the moments of this distribution. The density at each position is given by:

            \begin{equation}
                \rho(x,t) = \int_{x/t}^{\infty} f(x,v,t) \, dv = \frac{1}{2} \, \mathrm{erfc}\left( \frac{x/t - v_0}{\sqrt{2}\sigma_v} \right),
            \end{equation}

            i.e., the integral of a truncated Gaussian. Similarly, the \textit{mean velocity} and \textit{velocity dispersion} at fixed \( (x,t) \) are computed from the conditional distribution \( f(v|x,t) = f(x,v,t)/\rho(x,t) \), via:

            \begin{align}
                \langle v \rangle(x,t) &= \frac{1}{\rho(x,t)} \int_{x/t}^\infty v f(x,v,t) \, dv, \\
                \langle v^2 \rangle(x,t) &= \frac{1}{\rho(x,t)} \int_{x/t}^\infty v^2 f(x,v,t) \, dv, \\
                \sigma_v^2(x,t) &= \langle v^2 \rangle(x,t) - \langle v \rangle(x,t)^2.
            \end{align}

            Each of these integrals can be expressed analytically in terms of the error function and exponential functions, since they are the moments of a truncated Gaussian.


            In Fig.~\ref{fig:collisionless_1D_stream}, I show a plot of the \textit{density}, \textit{mean velocity}, and \textit{velocity dispersion} as a function of position at a given time. The figure illustrates how the leading edge of the stream (larger \( x \)) contains only the fastest particles and thus has a lower density and higher average velocity. Conversely, the trailing regions have a broader mix of velocities and higher local dispersion.
            
            \begin{figure}
                \centering
                \includegraphics[width=\linewidth]{images/collisionless_1D_stream.png}
                \caption{A snapshot at a fixed time of selected moments of the distribution function describing one-dimensional collisionless streaming, used here as an approximation for a stellar stream, as given by Eq.~\ref{eq:one_dimensional_collisionless_streaming}. Particles are continuously injected at the origin with velocities drawn from a Gaussian distribution with mean $v_0 $ and dispersion $ \sigma_v$. Shown are the particle density, local mean velocity, and local velocity dispersion, all normalized. }
                \label{fig:collisionless_1D_stream}
            \end{figure}


            \textit{Note:} I would love to explore this more rigorously in the context of time-dependent or eccentric mass loss, where the source term becomes periodic. I did begin studying such a model using a time-dependent Gaussian source, but at some point the ``simple approach'' becomes as complex as full numerical simulations. 

            





\section{The Ignored Physics} \label{sec:ignoredphysics}


    \textbf{NOTE. I am trying to construct a narrative by starting with the collisional dynamics to show why it was ignored for the galaxy and show that it has an influence for globular clusters. I want to discuss aspects such as two body relaxation, evaporation, etc. and how the assumptions make above are not always valid, and the type of phenomena our model cannot cover. I want to cite the papers such as Balbinot (the devils in the tails), Baumgardt's NBody library, and others. This section is quite incomlpete, I would appreciate ideas.}
    
    This section could be the longest of the thesis, but I'll try to keep brief as it mostly discusses aspects of the problem that the textbooks, conferences, and the literature, and have shown me to be important. This section certainly serves as a base for more things to include in future experiments that improve on the previous ones. However, maxwells equations are not relevant here, nor schrondiners equation. 
        


    \subsection{Collisional dynamics} \label{sec:collisionalDynamics}

        In my experiment discussed in chapter 5, there are collisions in the sense that other globular clusters can impact a stream. However, this is not what I mean by collisional dynamics. Indeed, collisional dynamics are the effects of gravitational encounters driving the evolution of the system, instead of the orbits being determined by the mean field only. There is a fundamental derivation in gravitational dynamics that justifies when collisionless dynamics are justified and this is called the \textbf{two-body relaxation time}. The name of the game is to consider how long it takes for a particle to have it's momentum vector changed by the same order of mangitude. Bovy and Binney \& Tremaine present this, and any expert in the field is familiar with such a fundamental derivation. I would like to present some aspects of the computation that I find to be insightful. 

        First of all, we would like to find out how long it takes for $|\Delta \vec{v}|/|\vec{v}|\approx 1$, as the particle traverses a realsitic gravitational environment made of many point masses. We would like a general solution, for arbitrary orbits in any galactic environment. However, that's too much to ask for and we can obtain powerful insight with simpler computations. Here is how the problem is simplified:
        \begin{itemize}
            \item assume the star is traveling on a straight line 
            \item assume all field stars are at rest with respect to the field star 
            \item assume that the galaxy has a uniform stellar number density
            \item assume the impule approximation for all encounters
        \end{itemize}
        
        Remember that this computation is calculating how long the it takes for the orbit to deviate from the current trajectory if it were in a perfect fluid and not in a environment made of discrete particles. These assumptions above are great for finding the relaxation time of stars on circular orbits within spherical stellar systems or the disk of galaxies. Eccentricities and inclinations would mean that the number density of the medium would change greatly with time. Also, a corrolary of the impules approximation for each encounter is that no single encounter greatly deviates the particle's trajectory. Thus, there is a limit on the lowest minimum impact parameter. 

        Remember, by definition, all impact parameters are perpendicular to the star's trajectory. By applying the impulse approximation for each encounter, the change in momentum is always perpendicular to star's trajectory. We can compute the number of encounters as a function of the impact parameter. this is the density in the medium multiplied with the number of particles within a ring about b. 
        \begin{align}
            \sigma &= \pi \left[(b+db)^2-(b-db)^2 \right]\\
            \sigma &= 4\pi bdb,
        \end{align}
        We can estimate the number density of the medium by using a sphere: $n=\frac{3N}{4\pi R^3}$. With this, number of encounters as a function of impact parameter, per unit length is: 
        \begin{equation}
            \frac{d N_\mathrm{int}}{db} = \frac{3N}{R^3} b
        \end{equation}
        Remember, this is the number of interactions per unit length. We can change this to the total number of interactions per unit time by swapping $dx=vdt$, assuming the star has constant velocity: 
        \begin{equation}
            \frac{d N_\mathrm{int}}{dtdb} = \frac{3N}{R^3} vb
        \end{equation}
        We can then find the rate at which the energy changes by integrating the 

        Ok. I can finish this later. I just want to be insightful and show that MW va bene, for the disc. Then I want to show that this isn't good for some globular clusters. Perhaps I can get a list where it would be appropraite? why didn't I compute this before!

        
        
        

        
        
        

        Globular clusters are undoubteled collisional systems. The same computating using the columb law, to compute how long it takes for a given particles to change it's momentum on same order of magnitude as it's initial momentum is much less than the age of a given globular cluster. This computation is what justified us for using the fluid limit for galactic dynamics, and it is the same computation that instructs us that we are incorrect for globular clusters. \citet{2023LRCA....9....3S} published a review discussing \textit{Computational methods for collisional stellar systems}, and entitled one of their sections \textit{Nbody the growth of an industry}. 


        \begin{itemize}
            \item not nbody
            \item no mass segregation
            \item no three body encounters 
            \item no soft or hard binaries 
            \item show some results from Corespray 
        \end{itemize}
    
    \subsection{Stellar evolution}
        \begin{itemize}
            \item They're all point masses 
            \item No salpeter's 
            \item No strong initial mass loss 
            \item No accurate model for the colors 
            \item No multiple stellar populations 
        \end{itemize}
    
    \subsection{Time evolution}
        In someways, we take time evolution into account, and in someways, we ignore and this has already been covered in the previous sections. i.e., the orbit of the star-particles depend on the position of the host globular cluster, which I do not solve for simoltaneously but instead opt to load it into the computation, as shown in Section~\ref{subsec:myEquationsOfMotion}. Also, things like mass segregation and stellar evolution are time-dependent which is completely ignored in my simulations. 

