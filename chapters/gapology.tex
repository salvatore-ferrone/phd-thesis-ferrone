\section{Published results}

\section{Velocity distribution within the stream}

    \subsection{Self-Segregation and Stream Chilling}

    The classic collisionless boltzmann equation:
    \begin{equation}
        \frac{df}{dt} = 0 = \frac{\partial f}{\partial x} \frac{dx}{dt} + \frac{\partial f}{\partial v} \frac{dv}{dt}+ \frac{\partial f}{\partial t} 
    \end{equation}

    we are saying that the pusles drift with the same velocities thus $\frac{dv}{dt}=0$. I impart more assumptions, namely that: $\rho(x,t=0)=\delta(x)$ and that velocity is defined as a normal distribution that does not change over time. This means that I can write the initial distribution function as:
    
    \begin{equation}
        f(x,v,t=0) = \delta(x)\frac{1}{\sigma\sqrt{2\pi}}\textrm{exp}\left(-\frac{1}{2}\left(\frac{v-\langle v \rangle}{\sigma}\right)^2\right)
    \end{equation}    

    the solution to the evolution of the density is $\rho = \int f dv$, and you also need to perform this variable substitution $f(x,v,t) = f(x-vt,v,0)$

    \begin{equation}
        \rho(x,t) = \frac{1}{\sigma t \sqrt{2\pi} }\textrm{exp}\left(-\frac{1}{2}\left(\frac{x-\langle v \rangle t}{\sigma_v t}\right)^2\right)
    \end{equation}

    it's also useful to know the relative velocity of the impact site between one group and another. These things drift apart. How much faster does one group go ahead of another? 
    \begin{equation}
        \delta v_{ij} = \frac{x\prime}{t-iT} - \frac{x\prime}{t-jT}
    \end{equation}
    where $i,j$ are the indexes for the packet. Note that this only works for $t > nT$ where $T$ is the orbital period, or spacing between the impacts, $n$ is the number of pericenter passages. $t$ is the total simulation time. $x\prime$ is the position of the impact. It makes sense that the velocity of the particle is the position where the impact occured, divided by the time since it left the origin. 