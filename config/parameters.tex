% Encodage des caractères et langue du document
\usepackage{setspace}
\usepackage[OT1]{fontenc}
\usepackage[utf8]{inputenc}
\usepackage{lmodern}
\usepackage[english,french]{babel}
\usepackage{textgreek}
% \usepackage{config/psl-cover/aas_macros}
\setcounter{tocdepth}{3} % Pour que les subsubsections n'apparaissent pas dans la TOC
\setcounter{secnumdepth}{3} % Pour que les subsubsections ne soient pas numérotées
\usepackage{fixltx2e}

%%%%%%%%%% Gestions des marges %%%%%%%%%% 
\usepackage{geometry} % Si on a besoin d'une configuration plus précise des marges
\geometry{a4paper,                % format de papier
% Définition des marges :
  left= 3cm,right = 2cm,  % marge intérieure extérieure à la page
  top = 3cm,bottom = 3cm,
% En-tête et pied de page :
  headheight=6mm,         % espace réservé à l'en-tête dans la marge top
  %headsep=3mm,            % espace entre le corps et l'en-tête
  %footskip=9mm            % espace entre le corps et le pied de page
  marginparwidth = 16mm
}

\raggedbottom
\reversemarginpar
% \usepackage{showframe} % pour afficher les traits des marges

%%%%%%%%%%%%%%%%%%%%%%%%%%%%%%%%%%%%%%%%%%%%%%%%%%%%%%%%%%%%%%%%%%%%
%%%%%%%%%%% Gestion maths %%%%%%%%%% 
\usepackage{amsmath,amssymb,amsfonts,amsthm}
%\usepackage{mathtools} % version modifiée de amsmath, ajoute des symboles, etc.
\usepackage{mathrsfs}% pour rajouter un format de lettres façon calligraphie en math mode.
\DeclareMathOperator{\sinc}{sinc}
\DeclareMathOperator{\e}{e}

\usepackage[locale = FR]{siunitx} % Pour gérer les unités
\sisetup{inter-unit-product=\ensuremath{{}\cdot{}}} % pour mettre des points médians entre les unités quand il y en a plusieurs
\sisetup{separate-uncertainty=true,multi-part-units=single} % pour faire des incertitudes en écrivant \SI{valeur(incertitude)}{unité}
\DeclareSIUnit\vitesse{\meter\per\second}
\usepackage{eurosym}
\DeclareSIUnit{\octet}{o}

%%%%%%%%%%%%%%%%%%%%%%%%%%%%%%%%%%%%%%%%%%%%%%%%%%%%%%%%%%%%%%%%%%%%
%%%%%%%%%%%  Graphics / Table / List %%%%%%%%%%%
\usepackage{graphicx,array,tikz,multirow}
\usepackage{caption,subcaption} % permet de faire des subfigures (remplace le package subfig)
\usepackage{svg,float}
\usepackage{booktabs,paralist}
\newcolumntype{x}[1]{>{\centering\arraybackslash\hspace{0pt}}p{#1}}
\usepackage[section]{placeins}
% \usepackage{hanging}

%%%%%%%%%%%%%%%%%%%%%%%%%%%%%%%%%%%%%%%%%%%%%%%%%%%%%%%%%%%%%%%%%%%%
%%%%%%%%%%% Header / Foot %%%%%%%%%%% 
\usepackage{fancyhdr,emptypage} % garantit que les pages blanches avant les débuts de chapitres soient vraiment blanches (pas d'en-tête ni de pied de page)
\let\cleardoublepage\clearpage
 
\fancypagestyle{plain}{ %% Page chapitre, toc ...
    \fancyhead{}\fancyfoot[C]{\thepage}
    \renewcommand{\headrulewidth}{0pt}
    \renewcommand{\footrulewidth}{0pt}
}

%%%%%%%%%%% Page normale
\pagestyle{fancy}
    % \renewcommand{\chaptermark}[1]{\markboth{\chaptername \ \thechapter.\ #1}{}} % sert à personnaliser l'affichage de \leftmark (ici : le mot "Chapitre", le numéro, un point, et le titre du chapitre, sans écrire en majuscules)
    % \renewcommand{\chaptermark}[1]{\markleft{\chaptername \ \thechapter.\ #1}{}}
    % \renewcommand{\sectionmark}[1]{\markright{\thesection.\ #1}} % sert à personnaliser l'affichage de \rightmark (ici : le numéro et le titre de la section en cours, sans écrire en majuscules)
    \fancyhf{} % assure que les entête et pieds de page sont vides au départ
    \fancyhead[LE]{\selectfont\nouppercase{\leftmark}}
    \fancyhead[RO]{\selectfont\nouppercase{\rightmark}}
    \fancyfoot[C]{\thepage}
% Explications :
% L = left, R = right, C = center, E = even pages, O = odd pages
%\leftmark : adds name and number of the current top-level structure (for example, Chapter for reports and books classes; Section for articles ) in uppercase letters.
%\rightmark : adds name and number of the current next to top-level structure (Section for reports and books; Subsection for articles) in uppercase letters.

%%%%%%%%%%% Personnaliser les premières pages des chapitres
\usepackage[Lenny]{fncychap}
\ChNameVar{\fontsize{25}{25}\usefont{OT1}{phv}{m}{n}\selectfont}
\ChRuleWidth{0pt}
\ChNumVar{\fontsize{60}{62}\selectfont\textcolor{curcolor}}

\makeatletter
\ChTitleVar{\Huge\rm}
\renewcommand{\DOCH}{%
\setlength{\fboxrule}{\RW} % Let fbox lines be controlled by
\fbox{\CNV\FmN{\@chapapp}\space \CNoV\thechapter}\par\nobreak
\vskip 20\p@}
\renewcommand{\DOTIS}[1]{%
\CTV\bfseries\FmTi{#1}\par\nobreak
\vskip 20\p@}
\makeatother

\renewcommand{\thesection}{\arabic{section}}

% Pour la table des matières
\usepackage[francais,nohints,tight]{minitoc}		% Mini table des matières, en français
\setcounter{minitocdepth}{2} % Mini-toc détaillées (sections/sous-sections)
\setlength{\mtcindent}{-1em} % décalage des minitoc à gauche
\dominitoc

\usepackage[nottoc]{tocbibind} % pour que la bibliographie apparaisse dans la table des matières (avec l'option pour que la table des matières elle-même n'apparaisse pas dans la table des matières).
% \usepackage{tocloft}% pour pouvoir modifier les tailles d'espacement dans la table des matières
\usepackage[titles]{tocloft}

%%%%%%%%%%%%%%%%%%%%%%%%%%%%%%%%%%%%%%%%%%%%%%%%%%%%%%%%%%%%%%%%%%%%
%%%%%%%%%%% Divers %%%%%%%%%%% 
\usepackage{textcomp} % rajoute des symboles 
\usepackage{xcolor} % pour ajouter de la couleur (si besoin)
\usepackage{epigraph} % pour rajouter des citations en début de chapitre  \epigraph{Citation}}{Auteur}
\usepackage{titling}
\usepackage{lipsum} 
\usepackage{csquotes} % added 07/09/21

\usepackage{xspace}
\usepackage{afterpage}
\renewcommand{\baselinestretch}{1.2} % interligne

\usepackage[textsize=footnotesize]{todonotes}

%%%%%%%%%%%%%%%%%%%%%%%%%%%%%%%%%%%%%%%%%%%%%%%%%%%%%%%%%%%%%%%%%%%%
%%%%%%%%%%% Links ref  %%%%%%%%%%% 
\usepackage{bookmark}
\usepackage{acronym}
\usepackage[nameinlink,french]{cleveref} % noabbrev
\Crefname{figure}{Fig.}{Figs.} % traduction des références aux figures/tables/équations
\crefname{figure}{fig.}{figs.}
\Crefname{equation}{Eq.}{Eqs.}
\crefname{equation}{eq.}{eqs.}
\Crefname{table}{Table.}{Tables.}
\crefname{table}{table.}{tables.}

% Configuration de hyperref
\definecolor{color_ref}{rgb}{0.18, 0.31, 0.31} % couleur cite
\definecolor{color_link}{RGB}{36, 56, 141}
\definecolor{curcolor}{RGB}{113,127,184} % couleur des liens (bleu clair)

\hypersetup{
	colorlinks=true, % colore les liens au lieu de les encadrer
	pdfstartview=FitV, % ouvre le PDF de façon à ce qu'il prenne la taille verticale de l'écran
	urlcolor=color_link, % choix de la couleur des liens URL
	linkcolor= color_link, % choix de la couleur des liens internes (table des matières, etc.)
	citecolor=color_ref % choix de la couleur des liens de citations
}

%%%%%%%%%%%%%%%%%%%%%%%%%%%%%%%%%%%%%%%%%%%%%%%%%%%%%%%%%%%%%%%%%%%%
%%%%%%%%%%% Bibliography ref  %%%%%%%%%%%
\usepackage[
    backend=biber,
    style=authoryear,
    sorting=nyt,
    maxcitenames=2,
    mincitenames=1,
    maxbibnames=6,
    minbibnames=6,
    natbib=true,
    hyperref=true,
    backref=true,
    url=false,
    doi=false,
    isbn=false
]{biblatex}

\renewcommand*{\bibfont}{\footnotesize}
\setlength\bibitemsep{\itemsep}
\renewbibmacro{in:}{} % remove "In:"

\renewcommand*{\labelalphaothers}{}
\DeclareLabelalphaTemplate{
  \labelelement{
    \field[final]{shorthand}
    \field{labelname}
    \field{label}
  }
  \labelelement{\literal{,\addhighpenspace}}
  \labelelement{\field{year}}
}

\AtEveryBibitem{%
    \clearfield{note} % Remove note
    \clearlist{language} % Remove doi
}
