\definecolor{sapred}{rgb}{0.5098039,0.1411765,0.2}
\newcommand{\sapred}[1]{\textcolor{sapred}{#1}}

\newcommand{\sujet}[1]{\renewcommand{\sujet}{#1}}
\newcommand{\auteur}[1]{\renewcommand{\auteur}{#1}}
\newcommand{\encadrant}[1]{\renewcommand{\encadrant}{#1}}

\newcommand{\reference}[1]{%
\vspace{.1cm}
\begin{singlespace}
\tikzstyle{titlebox}=[rectangle,inner sep=10pt,inner ysep=10pt,draw=curcolor,draw]%
\tikzstyle{title}=[fill=white]%
\bigskip\noindent\begin{tikzpicture}
\node[titlebox] (box){%
    \begin{minipage}{0.88\textwidth}
#1
    \end{minipage}
};
\node[title] at (box.north west) {\color{curcolor}  reference};
\end{tikzpicture}\bigskip%
\vspace{.1cm}
% \minitoc
\end{singlespace}
\newpage
}

\newcommand{\myparagraph}[1]{\paragraph{#1}\mbox{} \vskip .5\baselineskip \par}

\newcommand{\T}[1]{T\textsubscript{#1}}


%%% Macro pour la mise en forme de la liste des publiations
\newcommand{\Myprod}[3]{ % sans DOI
\begin{singlespace}{#1}. ``\textit{{#2}}''. {#3}.
\end{singlespace}}

\newcommand{\MyprodwithDOI}[4]{ % avec DOI
\begin{singlespace}{#1}. ``\textit{{#2}}''. {#3}. {DOI: \href{https://doi.org/#4}{#4}}
\end{singlespace}}

% Chapitre sans mise en forme particulière (remerciements, conclusion
\newcommand{\addchapnonumber}[1]{
\phantomsection
\addtocounter{chapter}{1}
\chapter*{#1}
\addcontentsline{toc}{chapter}{#1}
\markboth{#1}{#1}
\setcounter{section}{0}
}

%%% Macro pour mise en forme de l'abstract et résumé avec mots clés
\newcommand{\AddResumeAbstract}{
\chapter{Résumé}
% L'avènement de l'astrométrie \textit{Gaia} a permis des études à grande échelle des étoiles et des sous-structures galactiques. Les $\sim 160$ amas globulaires (GCs) de la Voie lactée, chacun contenant de centaines de milliers à plusieurs millions d'étoiles, peuvent désormais être caractérisés par des distances, des mouvements propres, des vitesses radiales, des masses et des tailles précises. Parallèlement, le nombre de courants stellaires connus est passé d'environ 60 au début de cette thèse à plus de 120 aujourd'hui. Comme les courants tracent les orbites de leurs progéniteurs, ils constituent d'excellents sondages du potentiel gravitationnel de la Voie lactée et des perturbations de matière noire. Grâce à ces données, nous avons étudié le système des courants stellaires et des amas globulaires galactiques. J'ai développé le code open-source \texttt{tstrippy}, qui modélise le détachement tidal des étoiles d'amas via le problème restreint des trois corps. Avec ce cadre, nous avons prédit la distribution des débris tidaux de tous les GCs galactiques et réalisé des simulations ciblées du courant de Palomar~5, introduisant des « gaps » dues aux survols d'amas globulaires. Nous présentons les premières prédictions globales des débris tidaux de tous les GCs de la Voie lactée \citep{2023A&A...673A..44F}. Ces simulations sont accessibles publiquement et utilisées par la communauté. Dans une étude de suivi \citep{2025A&A...699A.289F}, nous avons quantifié la fréquence et l'ampleur des rencontres d'GC perturbant Palomar~5, montrant que ces interactions doivent être prises en compte pour éviter les faux positifs dans les recherches de sous-halos de matière noire. Nous avons également montré qu'une dispersion interne de vitesse accrue et des surdensités épicycliques peuvent réduire la persistance des gaps, diminuant la sensibilité des courants aux perturbations externes. Ce travail établit la distribution attendue des courants tidaux des GCs galactiques et quantifie le taux auquel les GCs perturbent Palomar~5. Il fournit une base pour distinguer l'origine des perturbations des courants, étudier la dynamique interne et l'évolution des amas globulaires, et contraindre le potentiel gravitationnel de la Voie lactée en matière visible et noire.
\thefrabstract
\markboth{}{}
\vskip 2em \noindent\makebox[\linewidth]{\rule{.5\linewidth}{0.4pt}}
\vfill
% \noindent \textbf{Mots clés :} Voie Lactée ; Amas globulaires ; courants stellaires ; Matière noire ; Dynamique galactique ; Disruption gravitationnelle
\noindent \textbf{Mots clés :} \thefrkeywords

\chapter{Abstract}
% The advent of \textit{Gaia} astrometry has enabled large-scale studies of stars and Galactic substructures. The Milky Way's $\sim 160$ globular clusters (GCs), each with hundreds of thousands to millions of stars, can now be characterized via accurate distances, proper motions, radial velocities, masses, and sizes. Meanwhile, known stellar streams have grown from about 60 at the start of this thesis to over 120 today. Because streams trace progenitor orbits, they provide excellent probes of the Milky Way's gravitational potential and dark matter perturbations. Enabled by this data, we studied the Galactic stellar stream and globular cluster system. I developed the open-source code \texttt{tstrippy}, which models tidal stripping of cluster stars via the restricted three-body problem. Using this framework, we predicted the distribution of tidal debris from all Galactic GCs and ran targeted simulations of Palomar~5's stream, introducing ``gaps'' from GC flybys. We present the first global predictions of tidal debris from all Milky Way GCs \citep{2023A&A...673A..44F}. These simulations are publicly available and used by the community. In a follow-up study \citep{2025A&A...699A.289F}, we quantified the frequency and range of GC encounters perturbing Palomar~5, showing such interactions must be considered to avoid false positives in dark matter subhalo searches. We also showed that increased internal velocity dispersion and epicyclic overdensities can reduce gap persistence, lowering stream sensitivity to external perturbations. This work establishes the expected distribution of tidal streams from Galactic GCs and quantifies the rate at which GCs perturb Palomar~5. It provides a foundation for disentangling the origins of stream perturbations, studying the internal dynamics and evolution of globular clusters, and constraining the Milky Way's gravitational potential in both visible and dark matter.
\theenabstract
\markboth{}{}
\vskip 2em \noindent\makebox[\linewidth]{\rule{.5\linewidth}{0.4pt}}
\vfill
% \noindent\textbf{Keywords :} Milky Way; Globular Clusters; Stellar Streams; Dark Matter; Galactic Dynamics; Tidal Disruption
\noindent\textbf{Keywords:} \theenkeywords
}